% ****************************************************************************************
% ************************     	RELACIONES Y FUNCIONES   	  ****************************
% ****************************************************************************************


% =======================================================
% =======         HEADER FOR DOCUMENT        ============
% =======================================================
    % *********   DOCUMENT ITSELF   **************
    \documentclass[12pt]{report}                                    %Type of docuemtn and size of font
    \usepackage[margin=1.2in]{geometry}                             %Margins and Geometry pacakge
    \usepackage{ifthen}                                             %Allow simple programming
    \usepackage{hyperref}                                           %Create MetaData for a PDF and LINKS!
    \setlength{\parindent}{0pt}                                     %Eliminate ugly indentation
    \author{Oscar Andrés Rosas}                                     %Who I am

    % *********   LANGUAJE AND UFT-8   *********
    \usepackage[spanish]{babel}                                     %Please use spanish
    \usepackage[utf8]{inputenc}                                     %Please use spanish - UFT
    \usepackage[T1]{fontenc}                                        %Please use spanish

    % *********   MATH AND HIS STYLE  *********
    \usepackage{amsthm, amssymb, amsfonts, mathrsfs}                %Make math beautiful
    \usepackage[fleqn]{amsmath}                                     %Please make equations left
    \usepackage{centernot}                                          %Allow me to negate a symbol
    \decimalpoint                                                   %Use decimal point

    % *********   GRAPHICS AND IMAGES *********
    \usepackage{graphicx}                                           %Allow to create graphics
    \usepackage{wrapfig}                                            %Allow to create images
    \graphicspath{ {Graphics/} }                                    %Where are the images :D

    % *********   LISTS AND TABLES ***********
    \usepackage{listings}                                           %We will be using code here
    \usepackage[inline]{enumitem}                                   %We will need to enumarate
    \usepackage{tasks}                                              %Horizontal lists
    \usepackage{longtable}                                          %Lets make tables awesome
    \usepackage{booktabs}                                           %Lets make tables awesome
    \usepackage{tabularx}                                           %Lets make tables awesome
    \usepackage{multirow}                                           %Lets make tables awesome
    \usepackage{multicol}                                           %Create multicolumns

    % *********   HEADERS AND FOOTERS ********
    \usepackage{fancyhdr}                                           %Lets make awesome headers/footers
    \pagestyle{fancy}                                               %Lets make awesome headers/footers
    \setlength{\headheight}{16pt}                                   %Top line
    \setlength{\parskip}{0.5em}                                     %Top line
    \renewcommand{\footrulewidth}{0.5pt}                            %Bottom line

    \lhead{                                                         %Left Header
        \hyperlink{chapter.\arabic{chapter}}                        %Make a link to the current chapter
        {\normalsize{\textsc{\nouppercase{\leftmark}}}}             %And fot it put the name
    }

    \rhead{                                                         %Right Header
        \hyperlink{section.\arabic{chapter}.\arabic{section}}       %Make a link to the current chapter
            {\footnotesize{\textsc{\nouppercase{\rightmark}}}}      %And fot it put the name
    }

    \rfoot{\textsc{\small{\hyperref[sec:Index]{Ve al Índice}}}}    %This will always be a footer  

    \fancyfoot[L]{                                                  %Algoritm for a changing footer
        \ifthenelse{\isodd{\value{page}}}                           %IF ODD PAGE:
            {\href{https://compilandoconocimiento.com/yo/}          %DO THIS:
                {\footnotesize                                      %Send the page
                    {\textsc{Oscar Andrés Rosas}}}}                 %Send the page
            {\href{https://compilandoconocimiento.com}              %ELSE DO THIS: 
                {\footnotesize                                      %Send the author
                    {\textsc{Compilando Conocimiento}}}}            %Send the author
    }
    
    
    
% ========================================
% ===========   COMMANDS    ==============
% ========================================

    % =====  GENERAL MATH  ==========
    \DeclareMathOperator \Space {\quad}                             %Use: \Space for a cool mega space
    \DeclareMathOperator \MiniSpace {\;}                            %Use: \Space for a cool mini space
    \newcommand \Such {\MiniSpace|\MiniSpace}                       %Use: \Such like in sets

    % =====  LOGIC  ==================
    \DeclareMathOperator \doublearrow {\leftrightarrow}             %Use: \doublearrow for a double arrow
    \newcommand \lequal {\MiniSpace \Leftrightarrow \MiniSpace}     %Use: \lequal for a double arrow
    \newcommand \linfire {\MiniSpace \Rightarrow \MiniSpace}        %Use: \lequal for a double arrow

    % =====  NUMBER THEORY  ==========
    \DeclareMathOperator \Naturals {\mathbb{N}}                     %Use: \Naturals por Notation
    \DeclareMathOperator \Integers {\mathbb{Z}}                     %Use: \Integers por Notation
    \DeclareMathOperator \Racionals{\mathbb{Q}}                     %Use: \Racionals por Notation
    \DeclareMathOperator \Reals {\mathbb{R}}                        %Use: \Reals por Notation
    \DeclareMathOperator \Complexs {\mathbb{C}}                     %Use: \Complex por Notation

    % === LINEAL ALGEBRA & VECTORS ===
    \DeclareMathOperator \LinealTransformation {\mathcal{T}}        %Use: \LinealTransformation for a cool T

    \newcommand{\pVector}[1]{                                       %Use: \pVector {Matrix Notation} use parentesis
        \ensuremath{\begin{pmatrix}#1\end{pmatrix}}                 %Example: \pVector{a\\b\\c} or \pVector{a&b&c} 
    }
    \newcommand{\lVector}[1]{                                       %Use: \lVector {Matrix Notation} use a abs 
        \ensuremath{\begin{vmatrix}#1\end{vmatrix}}                 %Example: \lVector{a\\b\\c} or \lVector{a&b&c} 
    }
    \newcommand{\Vector}[1]{                                        %Use: \Vector {Matrix Notation} no parentesis
        \ensuremath{\begin{matrix}#1\end{matrix}}                   %Example: \Vector{a\\b\\c} or \Vector{a&b&c}
    }



% =====================================================
% ============     	  COVER PAGE	   ================
% =====================================================
\begin{document}
\begin{titlepage}

	\center
	% ============ UNIVERSITY NAME AND DATA =========
	\textbf{\textsc{\Large Proyecto Compilando Conocimiento}}\\[1.0cm] 
	\textsc{\Large Matemáticas Discretas}\\[1.0cm] 

	% ============ NAME OF THE DOCUMENT  ============
	\rule{\linewidth}{0.5mm} \\[1.0cm]
		{ \huge \bfseries Relaciones Binarias}\\[1.0cm] 
	\rule{\linewidth}{0.5mm} \\[2.0cm]
	
	% ====== SEMI TITLE ==========
	{\LARGE Una Pequeña Introducción}\\[7cm] 
	
	% ============  MY INFORMATION  =================
	\begin{center} \large
	\textbf{\textsc{Autor:}}\\
	Rosas Hernandez Oscar Andres
	\end{center}

	\vfill

\end{titlepage}

% =====================================================
% ========                INDICE              =========
% =====================================================
\tableofcontents{}
\label{sec:Index}

\clearpage










% ======================================================================================
% =============================       PRINCIPIOS BASICOS      ==========================
% ======================================================================================
\chapter{Relaciones}
    \clearpage

    % =====================================================
    % ============           DEFINICION            ========
    % =====================================================
    \section{Definición}

        Una relación $R$ entre dos conjuntos $A$ y $B$ es ante todo otro conjunto, una relación binaria es aquella
        que es en el fondo un conjunto de pares ordenados (x,y) donde $x$ es un elemento de $A$, y así mismo
        $y$ es un elemento de $B$.

        Este nuevo conjunto $R$ nos muestra como es que esta relacionados algunos (o todos) elementos de $A$ con
        otros elementos de $B$.



        % =====================================
        % =========   ¿QUE SON?     ===========
        % =====================================
        \subsection*{Definiciones Formales}

            Una Relación $R: A \to B$ es un subconjunto de $A \times B$.

            Solemos escribir la proposición $(x, y) \in R$ como $x R y$ para que se vea más bonito.

            Solemos escribir la proposición $(x, y) \notin R$ como $x\centernot{R}y$ para que se vea más bonito.





        % =====================================
        % ======   DOMINIO E IMAGEN    ========
        % =====================================
        \clearpage
        \subsection{Dominio e Imagen}
            
            \subsubsection*{Dominio}

                El dominio $D_R$ de una relación $R : A \to B$ es simplemente el subconjunto de $A$ que
                contiene a todos los elementos que están relacionados hacia algun elemento de $B$.

                \begin{equation}
                    D_R = \{ a \in A \Such \exists b, \MiniSpace aRb \}
                \end{equation}

            \subsubsection*{Imagen}

                También le llaman Contradominio o Rango, la Imagen $I_R$ de una relación $R : A \to B$
                es simplemente el subconjunto de $B$ que contiene a todos los elementos que están
                relacionados mediante $R$.

                \begin{equation}
                    I_R = \{ b \in B \Such \exists a, \MiniSpace aRb \}
                \end{equation}


        % ========================================
        % ==  REFLEXIVA, SIMETRICA Y TRANSITIVA ==
        % ========================================
        \clearpage
        \subsection{Reflexiva, Simétrica y Transitiva}

        Vamos a definir estas propiedades para una $R: A \to A$.
            
            \subsubsection*{Reflexiva}

                Una relación reflexiva es aquella en la que cualquier a tiene que estar relacionada consigo
                misma.

                \begin{equation}
                    \forall a \in A, \MiniSpace aRa
                \end{equation}

            \subsubsection*{Simétrica}

                Una relación simétrica es aquella en la que cualquier si existe $aRb$ existe $bRa$.

                \begin{equation}
                    \forall a,b \in A,\MiniSpace (aRb) \to (bRa)
                \end{equation}

            \subsubsection*{Transitiva}

                Una relación reflexiva es aquella en la que cualquier a tiene que estar relacionada consigo
                misma.

                \begin{equation}
                    \forall a,b,c \in A, \MiniSpace ((aRb) \land (bRa)) \linfire (aRc)
                \end{equation}


            \subsubsection*{Relación de Equivalencia}
                Decimos que $R$ es una relación de equivalencia si es que se cumplen las 3 propiedades antes vistas


  

% ======================================================================================
% =============================          FUNCIONES            ==========================
% ======================================================================================
\chapter{Funciones}
    \clearpage    



    % =====================================================
    % ============           DEFINICION            ========
    % =====================================================
    \section{Definición}

        Las funciones son más que meras descripciones de relaciones numéricas. En un sentido más general,
        las funciones pueden comparar y relacionar diferentes tipos de estructuras matemáticas.

        Es probable que vea una función como un tipo de fórmula que describe una relación entre dos
        (o más) cantidades.

        % =====================================
        % =========   ¿QUE SON?     ===========
        % =====================================
        \subsubsection*{Definición Formal}

        Digamos que tenemos una relación $f$ entre dos conjuntos $A$ y $B$.
        Decimos que esta relación es una función si y solo si:

        \begin{itemize}
            \item Todo los elementos del dominio tienen un valor asignado:
                \begin{equation}
                    \forall a \in A, \exists b \in B, \MiniSpace afb
                \end{equation}

            \item Si $aRb_1$ y $aRb_2$ entonces $b_1 = b_2$

        \end{itemize}

        Pero no es la única forma de definirlo, otra forma que una función es una relación que 
        cumple con la propiedad de que para cada $a \in A$, la relación $f$ contiene exactamente
        un par ordenado de la forma $(a,b) \in f$.

        Ya que solo existe un par ordenado para cada $a \in A$, entonces solemos escribir 
        $aRb$ como $f(a) = b$.







    % =====================================================
    % ============   DOMINIO E IMAGEN    ==================
    % =====================================================
    \clearpage
    \section{Dominio e Imagen}
        
        \subsubsection*{Dominio}

            El dominio $D_f$ de una función $f : A \to B$ es simplemente el conjunto de $A$.

            Digo, esto se deduce de la definición, ya que es el conjunto de todas las posibles
            entradas.

        \subsubsection*{Rango}

            El Rango de una función $f : A \to B$ es el conjunto de todas las posibles salidas
            de la función.

            \begin{equation}
                \text{Rango}_f = \{ b \in B \Such (a,b) \in f \} = = \{ f(a) \Such a \in A \}
            \end{equation}






    % =====================================================
    % ========   INYECTIVAS Y SUPRAYECTIVAS   =============
    % =====================================================
    \clearpage
    \section{Inyectivas, Suprayectivas y Biyectivas}
        








\end{document}
