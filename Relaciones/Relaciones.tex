% ****************************************************************************************
% ************************     	RELACIONES Y FUNCIONES   	  ****************************
% ****************************************************************************************


% =======================================================
% =======         HEADER FOR DOCUMENT        ============
% =======================================================
    % *********   DOCUMENT ITSELF   **************
    \documentclass[12pt]{report}                                    %Type of docuemtn and size of font
    \usepackage[margin=1.2in]{geometry}                             %Margins and Geometry pacakge
    \usepackage{ifthen}                                             %Allow simple programming
    \usepackage{hyperref}                                           %Create MetaData for a PDF and LINKS!
    \setlength{\parindent}{0pt}                                     %Eliminate ugly indentation
    \author{Oscar Andrés Rosas}                                     %Who I am

    % *********   LANGUAJE AND UFT-8   *********
    \usepackage[spanish]{babel}                                     %Please use spanish
    \usepackage[utf8]{inputenc}                                     %Please use spanish - UFT
    \usepackage[T1]{fontenc}                                        %Please use spanish
    \usepackage{textcmds}                                           %Allow us to use quoutes

    % *********   MATH AND HIS STYLE  *********
    \usepackage{amsthm, amssymb, amsfonts, mathrsfs}                %Make math beautiful
    \usepackage[fleqn]{amsmath}                                     %Please make equations left
    \usepackage{centernot}                                          %Allow me to negate a symbol
    \decimalpoint                                                   %Use decimal point

    % *********   GRAPHICS AND IMAGES *********
    \usepackage{graphicx}                                           %Allow to create graphics
    \usepackage{wrapfig}                                            %Allow to create images
    \graphicspath{ {Graphics/} }                                    %Where are the images :D

    % *********   LISTS AND TABLES ***********
    \usepackage{listings}                                           %We will be using code here
    \usepackage[inline]{enumitem}                                   %We will need to enumarate
    \usepackage{tasks}                                              %Horizontal lists
    \usepackage{longtable}                                          %Lets make tables awesome
    \usepackage{booktabs}                                           %Lets make tables awesome
    \usepackage{tabularx}                                           %Lets make tables awesome
    \usepackage{multirow}                                           %Lets make tables awesome
    \usepackage{multicol}                                           %Create multicolumns

    % *********   HEADERS AND FOOTERS ********
    \usepackage{fancyhdr}                                           %Lets make awesome headers/footers
    \pagestyle{fancy}                                               %Lets make awesome headers/footers
    \setlength{\headheight}{16pt}                                   %Top line
    \setlength{\parskip}{0.5em}                                     %Top line
    \renewcommand{\footrulewidth}{0.5pt}                            %Bottom line

    \lhead{                                                         %Left Header
        \hyperlink{chapter.\arabic{chapter}}                        %Make a link to the current chapter
        {\normalsize{\textsc{\nouppercase{\leftmark}}}}             %And fot it put the name
    }

    \rhead{                                                         %Right Header
        \hyperlink{section.\arabic{chapter}.\arabic{section}}       %Make a link to the current chapter
            {\footnotesize{\textsc{\nouppercase{\rightmark}}}}      %And fot it put the name
    }

    \rfoot{\textsc{\small{\hyperref[sec:Index]{Ve al Índice}}}}    %This will always be a footer  

    \fancyfoot[L]{                                                  %Algoritm for a changing footer
        \ifthenelse{\isodd{\value{page}}}                           %IF ODD PAGE:
            {\href{https://compilandoconocimiento.com/yo/}          %DO THIS:
                {\footnotesize                                      %Send the page
                    {\textsc{Oscar Andrés Rosas}}}}                 %Send the page
            {\href{https://compilandoconocimiento.com}              %ELSE DO THIS: 
                {\footnotesize                                      %Send the author
                    {\textsc{Compilando Conocimiento}}}}            %Send the author
    }
    
    
    
% ========================================
% ===========   COMMANDS    ==============
% ========================================

    % =====  GENERAL TEXT  ==========
    \newcommand \Quote {\qq}                                        %Use: \Quote to use quotes

    % =====  GENERAL MATH  ==========
    \DeclareMathOperator \Space {\quad}                             %Use: \Space for a cool mega space
    \DeclareMathOperator \MiniSpace {\;}                            %Use: \Space for a cool mini space
    \newcommand \Such {\MiniSpace|\MiniSpace}                       %Use: \Such like in sets

    % =====  LOGIC  ==================
    \DeclareMathOperator \doublearrow {\leftrightarrow}             %Use: \doublearrow for a double arrow
    \newcommand \lequal {\MiniSpace \Leftrightarrow \MiniSpace}     %Use: \lequal for a double arrow
    \newcommand \linfire {\MiniSpace \Rightarrow \MiniSpace}        %Use: \lequal for a double arrow

    % =====  NUMBER THEORY  ==========
    \DeclareMathOperator \Naturals {\mathbb{N}}                     %Use: \Naturals por Notation
    \DeclareMathOperator \Integers {\mathbb{Z}}                     %Use: \Integers por Notation
    \DeclareMathOperator \Racionals{\mathbb{Q}}                     %Use: \Racionals por Notation
    \DeclareMathOperator \Reals {\mathbb{R}}                        %Use: \Reals por Notation
    \DeclareMathOperator \Complexs {\mathbb{C}}                     %Use: \Complex por Notation

    % === LINEAL ALGEBRA & VECTORS ===
    \DeclareMathOperator \LinealTransformation {\mathcal{T}}        %Use: \LinealTransformation for a cool T

    \newcommand{\pVector}[1]{                                       %Use: \pVector {Matrix Notation} use parentesis
        \ensuremath{\begin{pmatrix}#1\end{pmatrix}}                 %Example: \pVector{a\\b\\c} or \pVector{a&b&c} 
    }
    \newcommand{\lVector}[1]{                                       %Use: \lVector {Matrix Notation} use a abs 
        \ensuremath{\begin{vmatrix}#1\end{vmatrix}}                 %Example: \lVector{a\\b\\c} or \lVector{a&b&c} 
    }
    \newcommand{\Vector}[1]{                                        %Use: \Vector {Matrix Notation} no parentesis
        \ensuremath{\begin{matrix}#1\end{matrix}}                   %Example: \Vector{a\\b\\c} or \Vector{a&b&c}
    }



% =====================================================
% ============     	  COVER PAGE	   ================
% =====================================================
\begin{document}
\begin{titlepage}

	\center
	% ============ UNIVERSITY NAME AND DATA =========
	\textbf{\textsc{\Large Proyecto Compilando Conocimiento}}\\[1.0cm] 
	\textsc{\Large Matemáticas Discretas}\\[1.0cm] 

	% ============ NAME OF THE DOCUMENT  ============
	\rule{\linewidth}{0.5mm} \\[1.0cm]
		{ \huge \bfseries Relaciones Binarias}\\[1.0cm] 
	\rule{\linewidth}{0.5mm} \\[2.0cm]
	
	% ====== SEMI TITLE ==========
	{\LARGE Una Pequeña Introducción}\\[7cm] 
	
	% ============  MY INFORMATION  =================
	\begin{center} \large
	\textbf{\textsc{Autor:}}\\
	Rosas Hernandez Oscar Andres
	\end{center}

	\vfill

\end{titlepage}

% =====================================================
% ========                INDICE              =========
% =====================================================
\tableofcontents{}
\label{sec:Index}

\clearpage










% ======================================================================================
% =============================       PRINCIPIOS BASICOS      ==========================
% ======================================================================================
\chapter{Relaciones}
    \clearpage

    % =====================================================
    % ============           DEFINICION            ========
    % =====================================================
    \section{Definición}

        Una relación $R$ entre dos conjuntos $A$ y $B$ es ante todo otro conjunto, una relación binaria es aquella
        que es en el fondo un conjunto de pares ordenados (x,y) donde $x$ es un elemento de $A$, y así mismo
        $y$ es un elemento de $B$.

        Este nuevo conjunto $R$ nos muestra como es que esta relacionados algunos (o todos) elementos de $A$ con
        otros elementos de $B$.



        % =====================================
        % =========   ¿QUE SON?     ===========
        % =====================================
        \subsection*{Definiciones Formales}

            Una Relación $R: A \to B$ es un subconjunto de $A \times B$.

            Solemos escribir la proposición $(x, y) \in R$ como $x R y$ para que se vea más bonito.

            Solemos escribir la proposición $(x, y) \notin R$ como $x\centernot{R}y$ para que se vea más bonito.





    % =====================================================
    % ==============   DOMINIO E IMAGEN    ================
    % =====================================================
    \clearpage
    \section{Dominio, Contradominio e Imagen}
            
        \subsection*{Dominio}

            El dominio $D_R$ de una relación $R : A \to B$ es simplemente el subconjunto de $A$ que
            contiene a todos los elementos que están relacionados hacia algun elemento de $B$.

            \begin{equation}
                D_R = \{ a \in A \Such \exists b, \MiniSpace aRb \}
            \end{equation}

        \subsection*{Imagen}

            También le llama Rango, la Imagen $I_R$ de una relación $R : A \to B$
            es simplemente el subconjunto de $B$ que contiene a todos los elementos que están
            relacionados mediante $R$.

            \begin{equation}
                I_R = \{ b \in B \Such \exists a, \MiniSpace aRb \}
            \end{equation}




    % =====================================================
    % ========         RELACION INVERSA        ============
    % =====================================================
    \clearpage
    \section{Relación Inversa}

        Una relación inversa es bastante fácil de definir:

        \begin{equation}
            R^{-1} = \{ (b, a) \in (B \times A) \Such (a, b) \in R \}
        \end{equation}





    % ====================================================
    % ========   REFLEXIVA, SIMETRICA Y TRANSITIVA   =====
    % ====================================================
    \clearpage
    \section{Reflexiva, Simétrica y Transitiva}

        Vamos a definir estas propiedades para una $R: A \to A$.
            
        % =============================
        % =====     REFLEXIVA     =====
        % =============================
        \subsection{Reflexiva}

            Una relación reflexiva es aquella en la que cualquier $a$ tiene que estar relacionada
            consigo misma.

            \begin{equation}
                \forall a \in A, \MiniSpace aRa
            \end{equation}


        \subsection*{Cerradura Reflexiva}

            Si te das cuenta la relación mas sencilla que es simétrica es bastante simple:
            \begin{equation}
                Id_A = \{ (a, a) \Such a \in A \}
            \end{equation}


            La cerradura reflexiva $Cl_R(R)$ de una relación $R$ es una relación que cumple con:

            \begin{itemize}
                \item Es reflexiva: $Cl_R(R)$ es reflexiva. 
                \item R esta contenida en ella: $R \subseteq Cl_R(R)$
                \item $Cl_R(R)$ es la relación mas pequeña posible: 
                Si es que $S$ es reflexiva y $R \subseteq S$ entonces $Cl_R(R) \subseteq S$
            \end{itemize}



            De hecho si te das cuenta es muy sencillo encontrarla, pues
            $Cl_R(R) = R \cup Id_A$.




        % =============================
        % =====     SIMETRICA     =====
        % =============================
        \clearpage
        \subsection{Simétrica}

            Una relación simétrica es aquella en la que cualquier si existe $aRb$ existe $bRa$.

            \begin{equation}
                \forall a,b \in A,\MiniSpace (aRb) \to (bRa)
            \end{equation}


            La cerradura simétrica $Cl_S(R)$ de una relación $R$ es una relación que cumple con:

            \begin{itemize}
                \item Es simétrica: $Cl_S(R)$ es simétrica. 
                \item R esta contenida en ella: $R \subseteq Cl_S(R)$
                \item $Cl_S(R)$ es la relación mas pequeña posible: 
                Si es que $S$ es simétrica y $R \subseteq S$ entonces $Cl_S(R) \subseteq S$
            \end{itemize}



            De hecho si te das cuenta es muy sencillo encontrarla, pues
            $Cl_S(R) = R \cup R^{-1}$.



        % =============================
        % =====     TRANSITIVA    =====
        % =============================
        \clearpage
        \subsection{Transitiva}

            Una relación transitiva es aquella en la que cualquier a tiene que estar relacionada consigo
            misma.

            \begin{equation}
                \forall a,b,c \in A, \MiniSpace ((aRb) \land (bRa)) \linfire (aRc)
            \end{equation}

            La cerradura transitiva $Cl_T(R)$ de una relación $R$ es una relación que cumple con:

            \begin{itemize}
                \item Es transitiva: $Cl_T(R)$ es transitiva. 
                \item R esta contenida en ella: $R \subseteq Cl_T(R)$
                \item $Cl_T(R)$ es la relación mas pequeña posible: 
                Si es que $S$ es transitiva y $R \subseteq S$ entonces $Cl_T(R) \subseteq S$
            \end{itemize}


        % =============================
        % =====     EQUIVALENCIA  =====
        % =============================
        \clearpage
        \subsection{Relación de Equivalencia}
            Decimos que $R$ es una relación de equivalencia si es que se cumplen las 3 propiedades antes vistas


  





% ======================================================================================
% =============================          FUNCIONES            ==========================
% ======================================================================================
\chapter{Funciones}
    \clearpage    



    % =====================================================
    % ============           DEFINICION            ========
    % =====================================================
    \section{Definición}

        Las funciones son más que meras descripciones de relaciones numéricas. En un sentido más general,
        las funciones pueden comparar y relacionar diferentes tipos de estructuras matemáticas.

        Es probable que vea una función como un tipo de fórmula que describe una relación entre dos
        (o más) cantidades.

        Podemos definirlas de una manera informal como aquellas relaciones donde cada elemento de $A$ 
        esta involucrado solo una vez en la relación.

        % =====================================
        % =========   ¿QUE SON?     ===========
        % =====================================
        \subsubsection*{Definición Formal}

        Digamos que tenemos una relación $f$ entre dos conjuntos $A$ y $B$.
        Decimos que esta relación es una función si y solo si:

        \begin{itemize}
            \item Todo los elementos del dominio tienen un valor asignado:
                \begin{equation}
                    \forall a \in A, \exists b \in B, \MiniSpace afb
                \end{equation}

            \item Si $aRb_1$ y $aRb_2$ entonces $b_1 = b_2$

        \end{itemize}

        Pero no es la única forma de definirlo, otra forma que una función es una relación que 
        cumple con la propiedad de que para cada $a \in A$, la relación $f$ contiene exactamente
        un par ordenado de la forma $(a,b) \in f$.

        Ya que solo existe un par ordenado para cada $a \in A$, entonces solemos escribir 
        $aRb$ como $f(a) = b$.







    % =====================================================
    % ============   DOMINIO E IMAGEN    ==================
    % =====================================================
    \clearpage
    \section{Dominio e Imagen}
        
        \subsubsection*{Dominio}

            El dominio $D_f$ de una función $f : A \to B$ es simplemente el conjunto de $A$.

            Digo, esto se deduce de la definición, ya que es el conjunto de todas las posibles
            entradas.

        \subsubsection*{Imagen}

            El Rango ó Imagen de una función $f : A \to B$ es el conjunto de todas las posibles salidas
            de la función.

            \begin{equation}
                \text{Rango}_f = \{ b \in B \Such (a,b) \in f \} = = \{ f(a) \Such a \in A \}
            \end{equation}






    % =====================================================
    % ========   INYECTIVAS Y SUPRAYECTIVAS   =============
    % =====================================================
    \clearpage
    \section{Inyectivas, Suprayectivas y Biyectivas}

        \subsection{Biyectiva}

            Son funciones muy especiales, son aquellas en las tanto son inyectivas como suprayectivas.
            Pero veamos mas a fondo que es eso de que una función sea alguna de estas cosas.


        \subsection{Funciones Inyectivas}

            \Quote{Una función es aquella en la que una línea horizontal NUNCA toca más de un
            punto de la función}.

            Una función es inyectiva si a cada valor del conjunto dominio le corresponde un valor
            distinto en el conjunto imágen, es decir en el que es imposible que la función mande
            dos valores al mismo valor en $B$.

            Es decir si es que es imposible que $f(a_1) = b$ y que también $f(a_2) = b$.

            \begin{equation}
               \forall x, y \in A, \MiniSpace (x \neq y) \linfire (f(x) \neq f(y))
            \end{equation}


            \subsection*{Demostraciones}
            
            Para lograr demostrar que una $f$ es inyectiva tenemos que demostrar que para
            cualquiera combinación de elementos $x, y \in A$ se cumple que el hecho de que
            sean diferentes infiere a que $f(x) \neq f(y)$.

            \begin{itemize}
                \item 
                    \textbf{Demostración Directa}:

                    Supón que $x, y \in A$ y que $x \neq y$
                    \\
                    $\cdots$
                    \\
                    Por lo tanto $f(x) \neq f(y)$

                \item 
                    \textbf{ContraPositiva}:

                    Supón que $x, y \in A$ y que $f(x) = f(y)$
                    \\
                    $\cdots$
                    \\
                    Por lo tanto $x = y$

            \end{itemize}


        \clearpage
        \subsection{Suprayectivas}

            \Quote{Una función es aquella en la que una línea horizontal SIEMPRE toca mínimo          
            punto de la función}.

            Son funciones en las que cada elemento del rango tiene un element o en A.
            Una función es suprayectiva si está aplicada sobre todo $B$.

            \begin{equation}
                \forall b \in B, \exists a \in A, \MiniSpace f(a) = b
            \end{equation}


        \clearpage
        \subsection{Principio del Palomar}

            Aquí hay una gran historia de porque se llama así este principio.

            Supón que $A$ y $B$ son conjuntos finitos y que existe una $f: A \to B$ que es una función.

            Entonces tenemos que:
            \begin{itemize}
                \item Si $|A| > |B|$, entonces $f$ no es inyectiva.
                \item Si $|A| < |B|$, entonces $f$ no es suprayectiva.
            \end{itemize}







    % =====================================================
    % ========         FUNCION INVERSA        =============
    % =====================================================
    \clearpage
    \section{Función Inversa}

        Aunque parecido, para que exista la relación inversa sea también una función, nuestra función original 
        tendrá que ser biyectia.

        Esta función inversa tiene una propiedad muy obvia:

        \begin{equation}
            \forall x \in A, \MiniSpace f^{-1}(f(x)) = x
        \end{equation}

        Y también otra obvia:

        \begin{equation}
            \forall y \in B, \MiniSpace f(f^{-1}(y)) = y
        \end{equation}




    % =====================================================
    % ========         IMAGEN Y PREIMAGEN       ===========
    % =====================================================
    \clearpage
    \section{Imagén y Preimagen}

        Sea una función $f : A \to B$ y además $X \subseteq A$ y $Y \subseteq B$.  

        \subsubsection{Imagén}
            La imágen es a fin de cuentas un conjunto, de manera fomral:
            \begin{equation} 
                f(X) = \{ f(x) \Such x \in X \} 
            \end{equation} 

            Y es bastante obvio que $f(X) \subseteq B$.

        \subsubsection{PreImagen}
            La preimagen es a fin de cuentas un conjunto, y esta definido,
            incluso aunque la función no sea invertible, esta esta definida
            de manera fomral:
            \begin{equation} 
                f^{-1}(Y) = \{ x \in A \Such f(x) \in Y \} 
            \end{equation} 

            Y es bastante obvio que $f^{-1}(Y) \subseteq A$.





    % =====================================================
    % ========         CARDINALIDAD Y FUNCIONES      ======
    % =====================================================
    \clearpage
    \section{Cardinalidad y Funciones}

        Podemos definir la cardinalidad de los conjuntos de una mejor manera
        usando la cardinalidad:

        Los conjuntos tienen la misma cardinalidad $|A| = |B|$ si y solo si es que existe
        una función biyectiva $f : A \to B$.


        Gracias a esto podemos decir que:

        \begin{itemize} 
            \item $|\Naturals| = |\Integers| = |\Racionals| = \aleph_0$

            \item $P(|\Naturals|) = |\Reals| = \aleph_1$

            \item $P(|\Reals|) = \aleph_2$

        \end{itemize} 



\end{document}
