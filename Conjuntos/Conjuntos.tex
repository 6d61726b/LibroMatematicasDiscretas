% ****************************************************************************************
% ************************     	TEORIA DE CONJUNTOS  	  ********************************
% ****************************************************************************************


% =======================================================
% =======         HEADER FOR DOCUMENT        ============
% =======================================================
    % *********   DOCUMENT ITSELF   **************
    \documentclass[12pt]{report}                                    %Type of docuemtn and size of font
    \usepackage[margin=1.2in]{geometry}                             %Margins and Geometry pacakge
    \usepackage{ifthen}                                             %Allow simple programming
    \usepackage{hyperref}                                           %Create MetaData for a PDF and LINKS!
    \setlength{\parindent}{0pt}                                     %Eliminate ugly indentation
    \author{Oscar Andrés Rosas}                                     %Who I am

    % *********   LANGUAJE AND UFT-8   *********
    \usepackage[spanish]{babel}                                     %Please use spanish
    \usepackage[utf8]{inputenc}                                     %Please use spanish - UFT
    \usepackage[T1]{fontenc}                                        %Please use spanish

    % *********   MATH AND HIS STYLE  *********
    \usepackage{amsthm, amssymb, amsfonts, mathrsfs}                %Make math beautiful
    \usepackage[fleqn]{amsmath}                                     %Please make equations left
    \decimalpoint                                                   %Use decimal point

    % *********   GRAPHICS AND IMAGES *********
    \usepackage{graphicx}                                           %Allow to create graphics
    \usepackage{wrapfig}                                            %Allow to create images
    \graphicspath{ {Graphics/} }                                    %Where are the images :D

    % *********   LISTS AND TABLES ***********
    \usepackage{listings}                                           %We will be using code here
    \usepackage[inline]{enumitem}                                   %We will need to enumarate
    \usepackage{tasks}                                              %Horizontal lists
    \usepackage{longtable}                                          %Lets make tables awesome
    \usepackage{booktabs}                                           %Lets make tables awesome


    % *********   HEADERS AND FOOTERS ********
    \usepackage{fancyhdr}                                           %Lets make awesome headers/footers
    \pagestyle{fancy}                                               %Lets make awesome headers/footers
    \setlength{\headheight}{16pt}                                   %Top line
    \setlength{\parskip}{0.5em}                                     %Top line
    \renewcommand{\footrulewidth}{0.5pt}                            %Bottom line

    \lhead{                                                         %Left Header
        \hyperlink{chapter.\arabic{chapter}}                        %Make a link to the current chapter
        {\normalsize{\textsc{\nouppercase{\leftmark}}}}             %And fot it put the name
    }

    \rhead{                                                         %Right Header
        \hyperlink{section.\arabic{chapter}.\arabic{section}}       %Make a link to the current chapter
            {\footnotesize{\textsc{\nouppercase{\rightmark}}}}      %And fot it put the name
    }

    \rfoot{\textsc{\small{\hyperref[sec:Index]{Ve al Índice}}}}    %This will always be a footer  

    \fancyfoot[L]{                                                  %Algoritm for a changing footer
        \ifthenelse{\isodd{\value{page}}}                           %IF ODD PAGE:
            {\href{https://compilandoconocimiento.com/yo/}          %DO THIS:
                {\footnotesize                                      %Send the page
                    {\textsc{Oscar Andrés Rosas}}}}                 %Send the page
            {\href{https://compilandoconocimiento.com}              %ELSE DO THIS: 
                {\footnotesize                                      %Send the author
                    {\textsc{Compilando Conocimiento}}}}            %Send the author
    }
    
    
    
% ========================================
% ===========   COMMANDS    ==============
% ========================================

    % =====  GENERAL MATH  ==========
    \DeclareMathOperator \Space {\quad}                             %Use: \Space for a cool mega space
    \DeclareMathOperator \MiniSpace {\;}                            %Use: \Space for a cool mini space
    \newcommand \Such {\MiniSpace|\MiniSpace}                       %Use: \Such like in sets

    % =====  NUMBER THEORY  ==========
    \DeclareMathOperator \Naturals {\mathbb{N}}                     %Use: \Naturals por Notation
    \DeclareMathOperator \Integers {\mathbb{Z}}                     %Use: \Integers por Notation
    \DeclareMathOperator \Racionals{\mathbb{Q}}                     %Use: \Racionals por Notation
    \DeclareMathOperator \Reals {\mathbb{R}}                        %Use: \Reals por Notation
    \DeclareMathOperator \Complexs {\mathbb{C}}                     %Use: \Complex por Notation

    % === LINEAL ALGEBRA & VECTORS ===
    \DeclareMathOperator \LinealTransformation {\mathcal{T}}        %Use: \LinealTransformation for a cool T

    \newcommand{\pVector}[1]{                                       %Use: \pVector {Matrix Notation} use parentesis
        \ensuremath{\begin{pmatrix}#1\end{pmatrix}}                 %Example: \pVector{a\\b\\c} or \pVector{a&b&c} 
    }
    \newcommand{\lVector}[1]{                                       %Use: \lVector {Matrix Notation} use a abs 
        \ensuremath{\begin{vmatrix}#1\end{vmatrix}}                 %Example: \lVector{a\\b\\c} or \lVector{a&b&c} 
    }
    \newcommand{\Vector}[1]{                                        %Use: \Vector {Matrix Notation} no parentesis
        \ensuremath{\begin{matrix}#1\end{matrix}}                   %Example: \Vector{a\\b\\c} or \Vector{a&b&c}
    }



% =====================================================
% ============     	  COVER PAGE	   ================
% =====================================================
\begin{document}
\begin{titlepage}

	\center
	% ============ UNIVERSITY NAME AND DATA =========
	\textbf{\textsc{\Large Proyecto Compilando Conocimiento}}\\[1.0cm] 
	\textsc{\Large Matemáticas Discretas}\\[1.0cm] 

	% ============ NAME OF THE DOCUMENT  ============
	\rule{\linewidth}{0.5mm} \\[1.0cm]
		{ \huge \bfseries Teoría de Conjuntos}\\[1.0cm] 
	\rule{\linewidth}{0.5mm} \\[2.0cm]
	
	% ====== SEMI TITLE ==========
	{\LARGE Una Pequeña Introducción}\\[7cm] 
	
	% ============  MY INFORMATION  =================
	\begin{center} \large
	\textbf{\textsc{Autor:}}\\
	Rosas Hernandez Oscar Andres
	\end{center}

	\vfill

\end{titlepage}

% =====================================================
% ========                INDICE              =========
% =====================================================
\tableofcontents{}
\label{sec:Index}

\clearpage

% ======================================================================================
% =============================       PRINCIPIOS BASICOS      ==========================
% ======================================================================================
\chapter{Principios Básicos}
    \clearpage

    % =====================================================
    % ============           DEFINICION            ========
    % =====================================================
    \section{Definición}

        % =====================================
        % =========   ¿QUE SON?     ===========
        % =====================================
        \subsection*{¿Qué son?}
            Olvida todo lo que sabes sobre números. Olvídate de que sabes lo que es un número.
            Aquí es donde empiezan las matemáticas. En vez de matemáticas con números,
            vamos a hacer matemáticas con 'cosas'.

            Se denomina conjunto a la agrupación de entes o elementos, que poseen una o
            varias características en común. 

            Un conjunto puede ser una agrupación de números, de vectores, de autos, de espacios vectoriales,
            de objectos, de funciones e incluso un conjunto puede ser una agrupación de otros conjuntos.


            % =====================================
            % =====   IDEAS IMPORTANTES    ========
            % =====================================
            \subsection*{Ideas Importantes}

            Lo conjuntos generalmente son denotados por letras mayúsculas, como $A, B, C, \cdots$, mientras
            que las letras minusculas como $a, b, c, \cdots x, y, z$ se usan para representar elementos
            de un conjunto.


            Quiza los conjuntos más importantes que verás a lo largo de estos apuntes son:

            \begin{itemize}
                \item $\Naturals$ : Representa al conjunto de todos los naturales, ya sabes
                        números como \\$1, 2, 3, \cdots$
                \item $\Integers$ : Representa al conjunto de todos los enteros, ya sabes
                        números como \\$\cdots, -1, 0, 1, \cdots$
                \item $\Racionals$ : Representa al conjunto de todos los racionales, ya sabes
                        números como \\$\frac{1}{1}, \frac{1}{2}, \frac{23}{72}, \cdots$
                \item $\Reals$ : Representa al conjunto de todos los reales, ya sabes
                        números como \\$\frac{3}{2}, -\pi, 3, \cdots$
                \item $\Complexs$ : Representa al conjunto de todos los complejos, ya sabes
                        números como \\$3 + 2i, \pi i, 3, \cdots$
            \end{itemize}

            

        % =====================================
        % =========  ¿COMO DEFINIRLO?  ========
        % =====================================
        \clearpage
        \section{¿Cómo Definirlo?}


            % =====================================
            % =========   PERTENCIA     ===========
            % =====================================
            \subsection{Pertencia}

            Creo que el símbolo más importante al hablar de conjuntos es este: $x \in A $. Esto quiere decir,
            el elemento x \textbf{pertenece} al Conjunto A.

            
            Y de la misma manera puedes usar el símbolo $\notin$ que significa \textbf{no pertenece}.


            % =====================================
            % ====   FORMAS DE DECLARAR   =========
            % =====================================
            \subsection{Notación Básica}

                La notación más común para declarar o hablar de un conjunto es colocar los elementos de esta
                manera : $ A = \{\ \text{Elementos de A} \}$.

                Puedes entonces hacer algo como:

                \begin{itemize}
                    \item Enumerar TODOS los elementos o entes que forman el conjunto :
                    \\$A = \{a, e, i, o, u\}$

                    \item Enumerar un patrón de los elementos o entes que forman el conjunto :
                    \\$A = \{+2, -4, +8, -16, \cdots \}$
                \end{itemize}


                Recuerda también:

                \begin{itemize}
                    \item Los elementos repetidos no cuentan, si ya esta un elemento dentro del
                        conjunto, da lo mismo que lo vuelvas a enumerar.\\
                        $A = \{a, e, i, o, u\} = \{a, a, e, i, o, u\}$

                    \item No importa el orden en el me muestres los elementos,
                        solo importa que esten dentro.\\
                        $A = \{a, e, i, o, u\} = \{u, a, i, e, o\}$
                \end{itemize}



            % =====================================
            % ====   FORMAS DE DECLARAR   =========
            % =====================================
            \clearpage
            \subsection{Notación Formal} 

                Esta notación tiene un nombre genial en inglés, se le conoce como \textbf{Set Builder Notation},
                esta notación es la que sueles encontrar en los libros. 

                Se ve fea al principio pero te da toda la información que necesitas.

                Veamos como formarla poco a poco:

                Lo primero que hacemos es elegir una letra minúscula (de forma normal, no es ninguna regla)
                que representará a cualquier elemento al azar del conjunto, por ejemplo usemos la $x$.

                También solemos usar la línea vertical, de esta forma: $P_1 | P_2$ que se lee como: 
                $P_1$ tal que $P_2$.  

                \subsubsection*{Formas Básica}

                    Ahora si, veamos como se ve esta notación:
                    \begin{equation}   
                        A = \{ x \Such x \dots \}
                    \end{equation}

                    Esto es la base y esto es lo que nos quiere decir:

                    Definimos cierto conjunto, al que llamaremos $A$ como la agrupación de todas las $x$
                    (es decir cada x es un elemento, un ente) tal que cumplen ciertas características
                    (eso es lo que significa esos puntitos, ahí deberías poner las reglas que tenga
                    tu conjunto). \\

                    Por ejemplo podemos poner algo como:

                    $B = \{ x \Such x \in \Naturals, \MiniSpace x < 4 \}$\\

                    B es el conjunto de todas las $x$ tal que $x$ pertenece a los Naturales
                    y $x$ es menor que 4.


                \subsubsection*{Formas Alternas}
                
                    También es común ver la característica que pertenece a cierto conjunto mayor
                    antes de la línea vertical.

                    Entonces el ejemplo anterior se vería como:

                    $B = \{ x \in \Naturals \Such x < 4 \} = \{ x \Such x \in \Naturals, \MiniSpace x < 4 \} $


                
            % =====================================
            % ====          EJEMPLOS      =========
            % =====================================
            \clearpage
            \subsection{Ejemplos} 

            % ==================
            % ===  EJEMPLO   ===
            % ==================
            \subsubsection*{Ejemplo 1:}

                Veamos por ejemplo como definir el Conjunto $C_2$ como aquel que contenga a
                TODOS las vocales:

                \begin{equation*}   
                \begin{split}   
                    C_2 &= \{ Vocales \}       \\
                    C_2 &= \{a, e, i, o, u \}
                \end{split}   
                \end{equation*}

                Si te das cuenta, podemos definirlos de muchas maneras.


            % ==================
            % ===  EJEMPLO   ===
            % ==================
            \subsubsection*{Ejemplo 2:}

                Veamos por ejemplo como definir el Conjunto $C$ \emph{(lo sé me muero con
                mi creatividad para los nombres)} como aquel que contenga a TODOS los números
                reales negativos:

                \begin{equation*}   
                    C = \{ x \in \Reals |\quad x < 0 \}
                \end{equation*}

                \textbf{En Lenguaje normal:\\}
                Esto lo podemos leer como $C$ es el conjunto \emph{(es decir todo lo que esta
                entre parentesís)} de todas las $x$ que pertenezcan al los números reales
                \emph{(eso quiere decir el $x \in \Reals$)} tal que (eso lo representamos
                como: $|$) $x$ es menor que 0 \emph{(esa es nuestra condición para
                encontrar si alguna $x$ pertenece a nuestro conjunto)}.



    % =====================================================
    % ============        CLASIFICACION            ========
    % =====================================================
    \clearpage
    \section{Clasificación}
            
        Podemos clasificar de muchas maneras a los conjuntos, veamos las más comunes:

        \textbf{\large \\Tamaño}

        \begin{itemize}
            \item \textbf{Finito}:
                Si tiene una colección que se pueda contar, aunque sea difícil.

                Por ejemplo, el conjunto de juguetes incluye todos los tipos de
                juguetes que hay en el mundo. Aunque sea difícil, se podrían contar
                todos los tipos de juguetes del mundo, por lo que es finito.

            \item \textbf{Infinito}:
                Si tiene una colección que no se pueda terminar de contar nunca.

                Por ejemplo, el conjunto de todos los números pares, que son
                infinitos, es un conjunto infinito.
        \end{itemize}


    % =====================================================
    % ============        CONJUNTO VACIO        ===========
    % =====================================================
    \clearpage
    \section{Conjunto Vacío}
            
        Ok, ya sabemos que un conjunto es un grupo de elementos, pero ...
        ¿Cómo represento a un conjunto en el que no hay nada?

        Como una caja vacía.

        De hecho, me gusta, hablemos de el Conjunto vacío como un caja vacía.

        \subsubsection*{Definición}
        Llamemos $\phi$ como aquel conjunto tal que $\phi = \{ \}$ es decir el conjunto que no
        tiene elementos.

        Solemos usar este simbolo por su parecido con un cero, pero recuerda no es un cero, simplemente
        es una forma de denotar al conjunto vacío.

        \subsubsection*{Ideas Importantes}
        Listo, eso es casí todo, además te gustará que te recuerde las siguientes
        proposiciones:

        \begin{itemize}
            \item $|\phi| = 0$ : 
                    Esto quiere decir que la cardinalidad \emph{(es decir
                    la cantidad de elementos)} del conjunto vacío es la misma que 
                    la cantidad de galletas en una caja vacía de galletas, osea 0.

            \item $\phi \neq \{\phi\}$: 
                    Esto quiere decir que no es lo mismo hablar del conjunto
                    vacío que de hablar de un conjunto cualquiera que contiene al
                    conjunto vacío.

                    Es decir simplemente no es lo mismo tener una caja vacía que una caja
                    con una caja vacía dentro \emph{(si lo piensas la segunda caja ya 
                    no esta completamente vacía)}
        \end{itemize}


    % =====================================================
    % ============        CONJUNTO UNIVERSO      ==========
    % =====================================================
    \clearpage
    \section{Conjunto Universo}
        
        Como podemos imaginarnos, tenía que existir un término inverso, digamos que estamos
        analizando y agrupando animales por su habitad, entonces tenemos muchos conjuntos cool
        como animales del bosque o marinos, pero también tenemos a un mega conjunto que llamamos
        universo donde tenemos a todos los animales.

        Muchas veces a la hora de hablar sobre conjuntos solemos definirlos sobre un universo.

        Podemos ver muchaS analogías, veamos a ver cual te gusta más:

        \begin{itemize}
            \item Es como si el universo fuera el padre, entonces las hijas son nuestros conjuntos
            \item Es como si el universo una caja de cereal, entonces nuestros conjuntos son grupos de cereales
                que estaban dentro de la caja.
        \end{itemize}


        \subsubsection*{Definición}
        Un conjunto universo es aquel conjunto que contiene a cualquier conjunto del que estemos hablando.


        \subsubsection*{Ideas Importantes}
        Creo que resulta bastante obvio pero aquí hay algunas cosas que quizá te interesen.
        \begin{itemize}
            \item $\phi^C = U$
            \item $U^C = \phi$
        \end{itemize}





% ======================================================================================
% ============================      ALGEBRA CON CONJUNTOS     ==========================
% ======================================================================================
\chapter{Álgebra y Operaciones}
    \clearpage


    % =====================================================
    % ============        RELACIONES               ========
    % =====================================================
    \clearpage
    \section{Relaciones}

        Es muy importante que ademas de hablar de los conjuntos hablemos de las relaciones
        que existen entre los dos conjuntos.


        % =================================
        % ======    EQUIVALENCIA    =======
        % =================================
        \subsection{Equivalencia}

            Creo que esta relación es más que obvia, tanto que ya la he usado sin darme cuenta
            porque considero que es de lo más obvio.

            Que dos conjuntos sea iguales quiere decir que ambos conjuntos contiene \textbf{exactamente}
            los mismos elementos.

            Ve entonces que ahora tiene mucho sentido las ideas que puse antes:

            \begin{itemize}
                \item Los elementos repetidos no cuentan, si ya esta un elemento dentro del
                    conjunto, da lo mismo que lo vuelvas a enumerar.\\
                    $A = \{a, e, i, o, u\} = \{a, a, e, i, o, u\}$

                \item No importa el orden en el me muestres los elementos,
                    solo importa que esten dentro.\\
                    $A = \{a, e, i, o, u\} = \{u, a, i, e, o\}$
            \end{itemize}


        % =================================
        % ======    SUBCONJUNTOS    =======
        % =================================
        \clearpage
        \subsection{Subconjuntos}

            Esta es la relación mas importante siento yo, porque será la que mas ocupemos a lo largo
            del tiempo.

            Que el A sea un subconjunto de B quiere decir que \textbf{todos} los elementos de A
            también son elementos de B.

            \subsubsection{Definición}

                Una forma más formal de definirlo es que $x \in A \to x \in B$

                Esta idea es muy inteligente, pues nos dice que un el hecho de que
                un elemento pertenezca a A infiere o nos obliga a que ese mismo
                elemento pertenezca a B.


            \subsubsection{Proposiciones}

                Algunas proposiciones muy obvias son que:
                \begin{itemize}
                    \item $A \subseteq A$
                    \item Si $A = B$, entonces $A \subseteq B$
                    \item Todos los conjuntos son subconjuntos de conjunto universo $U$
                    \item $\emptyset$ es un subconjunto de cualquier conjunto
                \end{itemize}

            \subsubsection{SubConjuntos Propios}

                Solemos usar la idea de un subconjunto propio $A \subset B$ si es que sabemos ya que:
                $A \subseteq B$ y $A \neq B$





    % =====================================================
    % ============           OPERACIONES           ========
    % =====================================================
    \clearpage
    \section{Operaciones}

        Podemos hacer operaciones con los conjuntos de una manera muy similiar a como hacemos
        operaciones con los números normales, tu defines una operación, y la haces entre dos conjuntos
        y esta te dará un nuevo conjunto,  pero aquí siento que son incluso más divertidas.

        Te presentaré estas operaciones una a una y te mostraré con mas detalle cada una, así 
        que vamos, empecemos:

        % =================================
        % ======    INTERSECCION    =======
        % =================================
        \subsection{Intersección}

            Lo mejor de dos mundos, veamos como seleccionar a los elementos en común con operaciones
            de conjuntos.

            \begin{equation}
                A \cap B = \{ x \Such x \in A \text{ y también } x \in B \}
            \end{equation}

            Esta operación basicamente nos da un conjunto en el que estan \textbf{solo} los elementos
            que bien pertenezcan a A y también que pertenezcan a B.


        % =================================
        % ======        UNION       =======
        % =================================
        \clearpage
        \subsection{Unión}

            La unión representa la fuerza, así que veamos que representa la unión cuando estemos
            hablando de conjuntos.

            \begin{equation}
                A \cup B = \{ x \Such x \in A \text{ ó bien } x \in B \}
            \end{equation}

            Esta operación basicamente nos da un conjunto en el que estan \textbf{todos} los elementos
            que bien pertenezcan a A o bien que pertenezcan a B.


        % =================================
        % ======   COMPLEMENTO      =======
        % =================================
        \clearpage
        \subsection{Complemento}

            Todo lo que no seas tu, así que veamos que representa el complemento cuando estemos
            hablando de conjuntos.

            \begin{equation}
                A^C = \overline A = \{ x \in U \Such x \notin A \}
            \end{equation}

            Esta operación basicamente nos da un conjunto en el que estan \textbf{todos} los elementos
            que no pertenezcen a A.

            También hay otra forma de definirlo como $A^C = U - A$

            \subsubsection{Conjunto Universo}
                Si lo piensas detenidamente, aquí hay un problema y es que no te dije que es $U$, este
                representa el conjunto universo, es decir, es aquel conjunto que del que todos los demás
                son subconjuntos.

                Es importantes especificar cual es tu conjunto universo.


        % =================================
        % ======      RESTA         =======
        % =================================
        \clearpage
        \subsection{Resta}

            No quiero nada que ver contigo, así que veamos que representa la resta cuando estemos
            hablando de conjuntos.

            \begin{equation}
                A - B = A \char`\\ B = \{ x \Such x \in A \text{ y también }  x \notin B \}
            \end{equation}

            Esta operación basicamente nos da un conjunto en el que estan \textbf{todos} los elementos
            de A que no pertenezcen a B.

            A esta operación también se la conoce como complemento relativo.

            También hay otra forma de definirlo como $A - B = A \cap B^C$



        % =================================
        % ==    DIFERENCIA SIMETRICA   ====
        % =================================
        \clearpage
        \subsection{Diferencia Simétrica}

            El XOR de los conjuntos, así que veamos que representa la diferencia simétrica cuando estemos
            hablando de conjuntos.

            \begin{equation}
                A \Delta B = \{ x \Such x \in A \cup B \text{ y también }  x \notin A \cap B \}
            \end{equation}

            Esta operación basicamente nos da un conjunto en el que estan \textbf{todos} los elementos
            que pertenezcen a $A$ y a $B$, pero no a ambos.

            Ve que de la definición se tiene que:
            \begin{equation*}
            \begin{split}
                A \Delta B &= (A \cup B) - ( A \cap B) \\
                A \Delta B &= (A - B) \cup ( B - A)
            \end{split}
            \end{equation*}

            De hecho de la útima proposición sale su nombre, es una diferencia simétrica.


        % =================================
        % ==    PRODUCTO CARTESIANO    ====
        % =================================
        \clearpage
        \subsection{Producto Cartesiano}

            Esta es la base de lo que se conoce como es plano cartesiano, y es quizá
            la operación mas útil que vas a conocer a lo largo de estos textos,
            veamos específicamente porque:

            \subsubsection*{N-Tuplas}

                El resultado de un producto cartesiano es un conjunto formado de n-tuplas,
                cada n-tuplas es una agrupación ordenada de elementos.
                Por ejemplo $(a,b)$ ó $(x,y,z)$.

                Al ser un ente ordenado $(a,b) \neq (b,a)$

                La n en su nombre solo nos dice la cantidad de elementos que tiene cada tupla.


            \subsubsection*{Definición}

                Esta operación basicamente nos da un conjunto en el que estan \textbf{todas} las n-tuplas
                donde su primer elemento pertenece a $A$ y su segundo elemento pertenece a $B$.

                \begin{equation}
                    A \times B = \{ (a, b) \Such a \in A \text{ y también } b \in B \}
                \end{equation}

            \subsubsection*{Ideas Importantes}

            \begin{itemize}
                \item Ve que de la definición se tiene que $ A \times B \neq B \times A$

                \item También de manera común solemos simplificar $A \times A$ como $A^2$

                \item Un elemento de $A \times B \times C$ es $((a,b),c) = (a,(b,c)) = (a,b,c)$
            \end{itemize}

            % ====================
            % ==    EJEMPLOS  ====
            % ====================
            \clearpage
            \subsection*{Ejemplo}

                Podemos hacer uso de una tabla para encontrar todos los elementos del producto
                cartesiano.

                Veamos por ejemplo $A = \{a, b, c\}$ y $B = \{ x, y, z\}$

                \begin{tabular}{r ||c |c | c  }
                   $A \times B$ & $x$ & $y$ & $z$ \\
                   \midrule
                  
                   $a$ &  $(a, x)$  &  $(a, y)$  &  $(a, z)$  \\
                   $b$ &  $(b, x)$  &  $(b, y)$  &  $(b, z)$  \\
                   $c$ &  $(c, x)$  &  $(c, y)$  &  $(c, z)$  \\
                 
                \end{tabular}

                Entonces : 
                \begin{equation*}
                    A \times B = 
                    \{
                        (a, x) , (a, y) , (a, z) ,
                        (b, x) , (b, y) , (b, z) ,
                        (c, x) , (c, y) , (c, z) 
                    \}
                \end{equation*}



        % =================================
        % ==    CONJUNTO POTENCIA      ====
        % =================================
        \clearpage
        \subsection{Conjunto Potencia}

            El conjunto que contiene a todos los subconjuntos posibles. 

            Esta operación es diferente en el sentido de que no toma sus elementos del conjunto
            que toma como entrada, sino que usa esos elementos para combinarlos y crear subconjuntos
            que son los elementos de esta nueva operación.

            Ok, ok, quizá me puse muy intenso con el párrafo de arriba, veamos un poco más calmado
            como es que funciona.

            \subsubsection*{Definición}

                Esta operación basicamente nos da un conjunto en el que estan \textbf{todos} los
                conjuntos que son subconjuntos de tu conjunto original.

                \begin{equation}
                    P(A) = \{ A' \Such A' \subseteq A \}
                \end{equation}

            \subsubsection*{Ideas Importantes}

            \begin{itemize}
                \item Si tu conjunto tiene $n$ elementos, tu conjunto potencia tendrá $2^n$ elementos.

                \item Si $n < |A|$ entonces existen $\pVector{n\\r} = \frac{n!}{r!(n-r)!}$ subconjuntos
                        de $A$ de cardinalidad $r$

                \item $P(\emptyset) = \{\emptyset\}$

                \item $P( \{\emptyset\}) = \{\emptyset, \{\emptyset\} \}$

                \item $A \in P(A)$
                \item Solo si $A = \{\emptyset\}$ entonces $A \subseteq P(A)$,
                en cualquier otro caso $A \nsubseteq P(A)$
            \end{itemize}

            % ====================
            % ==    EJEMPLOS  ====
            % ====================
            \clearpage
            \subsection*{Ejemplo}

                Podemos hacer uso de una tabla y el binario para encontrar todos los
                elementos del conjunto potencia.

                Si quieres crear un conjunto potencia, escribe la sucesión de números binarios de $n$ cifras,
                y con cada número haz un subconjunto: Cuando haya un $'1'$, añade el elemento que corresponde.\\

                Veamos por ejemplo $A = \{a, b, c\}$ \\


                \begin{tabular}{c | c  }
                   $a b c$ & SubConjuntos       \\
                   \midrule
                  
                   $0 0 0$ &  $\{ \}$           \\
                   $0 0 1$ &  $\{ c \}$         \\
                   $0 1 0$ &  $\{ b \}$         \\
                   $0 1 1$ &  $\{ b, c \}$      \\
                   $1 0 0$ &  $\{ a \}$         \\
                   $1 0 1$ &  $\{ a, c \}$      \\
                   $1 1 0$ &  $\{ a, b \}$      \\
                   $1 1 1$ &  $\{ a, b, c \}$   \\
                 
                \end{tabular}

                Entonces : 
                \begin{equation*}
                    P(A) = 
                    \{
                        \{ \}, \{ c \}, \{ b \}, \{ b, c \}, 
                        \{ a \}, \{ a, c \}, \{ a, b \}, \{ a, b, c \}
                    \}
                \end{equation*}



    % =====================================================
    % ========         LEYES DE CONJUNTOS          ========
    % =====================================================
    \clearpage
    \section{Leyes de los Conjuntos}
            
        Sean $A, B, C$ conjuntos de un universo $U$, entonces tenemos las siguientes propiedades:

        \begin{itemize}
            \item \textbf{Doble Complemento} \\
                    $(A^C)^C = A$

            \item \textbf{Propiedad Conmutativa}
                \begin{itemize}
                    \item $A \cap B = B \cap A$
                    \item $A \cup B = B \cup A$
                \end{itemize}

            \item \textbf{Propiedad Asociativa}
                \begin{itemize}
                    \item $A \cap (B \cap C) = (A \cap B) \cap C$
                    \item $A \cup (B \cup C) = (A \cup B) \cup C$
                \end{itemize}

            \item \textbf{Propiedad Distributiva}
                \begin{itemize}
                    \item $A \cap (B \cup C) = (A \cap B) \cup (A \cap C)$
                    \item $A \cup (B \cap C) = (A \cup B) \cap (A \cup C)$
                \end{itemize}

            \item \textbf{Leyes de Morgan}
                \begin{itemize}
                    \item $(A \cap B)^C = A^C \cup B^C$
                    \item $(A \cup B)^C = A^C \cap B^C$
                \end{itemize}

            \clearpage

            \item \textbf{Propiedad de los Neutros}
                \begin{itemize}
                    \item $A \cap U = A$
                    \item $A \cup \emptyset = A$
                \end{itemize}

            \item \textbf{Propiedad de los Inversos}
                \begin{itemize}
                    \item $A \cap A^C = \emptyset$
                    \item $A \cup A^C = U$
                \end{itemize}

            \item \textbf{Propiedad de Dominación}
                \begin{itemize}
                    \item $A \cap \emptyset = \emptyset$
                    \item $A \cup U = U$
                \end{itemize}

            \item \textbf{Propiedad de Inepotencia}
                \begin{itemize}
                    \item $A \cap A = A$
                    \item $A \cup A = A$
                \end{itemize}

            \item \textbf{Propiedad de Absorción}
                \begin{itemize}
                    \item $A \cap (A \cup B) = A$
                    \item $A \cup (A \cap B) = A$
                \end{itemize}

        \end{itemize}


    % =====================================================
    % ============           CARDINALIDAD          ========
    % =====================================================
    \clearpage
    \section{Identidades y Propiedades}
            
        Ya conocemos su definición formal, incluso me dedique a poner algunas cosas super obvias en
        la pagina de definción de cada uno de ellos, pero es que hay tantas propiedades super interesantes
        que tenia que ponerles una sección propia para las mejores:


        % =====================================
        % ==========   SUBCONJUNTOS    ========
        % =====================================
        \subsection{Propiedades usando SubConjuntos}
            
            \subsubsection{Proposición}
                Sea $A, B, C$ conjuntos cualesquiera tal que $A \subseteq B$. Entonces tenemos que:
                \begin{equation*}
                \begin{split}
                    (A \cap C) &\subseteq (B \cap C)        \\
                    (A \cup C) &\subseteq (B \cup C)
                \end{split}
                \end{equation*}

            \subsubsection{Proposición}
                Sea $A, B, C$ conjuntos cualesquiera. Si $A \subseteq B$ y $A \subseteq C$.
                Entonces tenemos que:
                \begin{equation*}
                    A \subseteq (B \cap C)
                \end{equation*}

            \subsubsection{Proposición}
                Sea $A, B, C$ conjuntos cualesquiera. Si $A \subseteq C$ y $B \subseteq C$.
                Entonces tenemos que:
                \begin{equation*}
                    (A \cup B) \subseteq C
                \end{equation*}


        % =====================================
        % =====   CONJUNTO POTENCIA    ========
        % =====================================
        \clearpage
        \subsection{Propiedades usando Conjunto Potencia}
            
            \subsubsection{Proposición}
                $P(A \cap B) = P(A) \cap P(B)$

            \subsubsection{Proposición}
                $P(A) \cup P(B) \subseteq P(A \cup B)$






    % =====================================================
    % ============           CARDINALIDAD          ========
    % =====================================================
    \clearpage
    \section{Cardinalidad y sus Propiedades}
            
        Ok, vamos avanzando, ahora es la hora de ver una característica de los conjuntos.
        La Cardinalidad, que no es mas que una forma \emph{fancy} de decir el número de 
        elementos ó entes que contiene cierto conjunto.

        Puedes verlo como una función que recibe un conjunto cualquiera y te regresa
        un número \emph{(Bueno, tecnicamente también sta el caso en el que la cardinalidad es infinita)}.

        Esta es la forma en que solemos expresar la cardinalidad de un conjunto cualquiera:
        \begin{equation}   
            |A| = \#A = Card(A)
        \end{equation}

        PD: Como te imaginas las propiedades no sirven de nada si es que alguno de tus conjuntos
        tiene una cardinalidad infinita, así que voy a suponer por obviedad que todos los siguientes
        conjuntos son finitos, que sino tendría que especificarlo a cada 4 palabras.



        % =====================================
        % =====        COMPLEMENTOS      ======
        % =====================================
        \clearpage
        \subsection{Propiedades usando Complemento}
            
            \subsubsection{Proposición}
                Sea $A$ un conjunto cualquiera de un universo U, entonces tenemos que:
                \begin{equation*}
                    |A^C| = |U| - |A|
                \end{equation*}  

        % =====================================
        % =====        SUBCONJUNTOS      ======
        % =====================================
        \subsection{Propiedades usando Subconjuntos}
            
            \subsubsection{Proposición}
                Sea $A, B$ conjuntos cualesquiera. Si  $A \subseteq B$, entonces tenemos que:
                \begin{equation*}
                    |A| \leq |B|
                \end{equation*}


        % =====================================
        % =====   UNION E INTERSECCION   ======
        % =====================================
        \clearpage
        \subsection{Propiedades usando Unión e Intersección}
            
            \subsubsection{Proposición}
                Sea $A, B$ conjuntos cualesquiera, entonces tenemos que:
                \begin{equation*}
                \begin{split}
                    |A \cup B| &= |A| + |B| - |A \cap B|    \\  
                    |A \cap B| &= |A| + |B| - |A \cup B|  
                \end{split}
                \end{equation*}


        % =====================================
        % =====           RESTA          ======
        % =====================================
        \clearpage
        \subsection{Propiedades usando Resta}

            \subsubsection{Proposición}
                Sea $A, B$ conjuntos cualesquiera, entonces tenemos que:
                \begin{equation*}
                    |A - B| = |A| - |A \cap B|
                \end{equation*}


        % =====================================
        % ====     DIFERENCIA SIMETRICA   =====
        % =====================================
        \subsection{Propiedades usando Diferencia Simetrica}

            \subsubsection{Proposición}
                Sea $A, B$ conjuntos cualesquiera, entonces tenemos que:
                \begin{equation*}
                    |A \Delta B| = |A| + |B| - 2|A \cap B|
                \end{equation*}

        % =====================================
        % =====   PRODUCTO CARTESIANO     =====
        % =====================================
        \clearpage
        \subsection{Propiedades usando Producto Cartesiano}

            \subsubsection{Proposición}
                Sea $A, B$ conjuntos cualesquiera, entonces tenemos que:
                \begin{equation*}
                    |A \times B| = |A| \cdot |B|
                \end{equation*}

        % =====================================
        % =====   CONJUNTO POTENCIA      ======
        % =====================================
        \subsection{Propiedades usando Conjunto Potencia}

            \subsubsection{Proposición}
                Sea $A$ conjunto cualquiera, tal que $|A| = n$ entonces tenemos que:
                \begin{equation*}
                    |P(A)| = 2^n
                \end{equation*}


\end{document}