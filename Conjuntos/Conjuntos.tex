% ****************************************************************************************
% ************************     	TEORIA DE CONJUNTOS  	  ********************************
% ****************************************************************************************


% =======================================================
% =======         HEADER FOR DOCUMENT        ============
% =======================================================
    % *********   DOCUMENT ITSELF   **************
    \documentclass[12pt]{report}                                    %Type of docuemtn and size of font
    \usepackage[margin=1.2in]{geometry}                             %Margins and Geometry pacakge
    \usepackage{ifthen}                                             %Allow simple programming
    \usepackage{hyperref}                                           %Create MetaData for a PDF and LINKS!
    \setlength{\parindent}{0pt}                                     %Eliminate ugly indentation
    \author{Oscar Andrés Rosas}                                     %Who I am

    % *********   LANGUAJE AND UFT-8   *********
    \usepackage[spanish]{babel}                                     %Please use spanish
    \usepackage[utf8]{inputenc}                                     %Please use spanish - UFT
    \usepackage[T1]{fontenc}                                        %Please use spanish

    % *********   MATH AND HIS STYLE  *********
    \usepackage{amsthm, amssymb, amsfonts, mathrsfs}                %Make math beautiful
    \usepackage[fleqn]{amsmath}                                     %Please make equations left
    \decimalpoint                                                   %Use decimal point

    % *********   GRAPHICS AND IMAGES *********
    \usepackage{graphicx}                                           %Allow to create graphics
    \usepackage{wrapfig}                                            %Allow to create images
    \graphicspath{ {Graphics/} }                                    %Where are the images :D

    % *********   LISTS AND TABLES ***********
    \usepackage{listings}                                           %We will be using code here
    \usepackage[inline]{enumitem}                                   %We will need to enumarate
    \usepackage{tasks}                                              %Horizontal lists
    \usepackage{longtable}                                          %Lets make tables awesome


    % *********   HEADERS AND FOOTERS ********
    \usepackage{fancyhdr}                                           %Lets make awesome headers/footers
    \pagestyle{fancy}                                               %Lets make awesome headers/footers
    \setlength{\headheight}{16pt}                                   %Top line
    \setlength{\parskip}{0.5em}                                     %Top line
    \renewcommand{\footrulewidth}{0.5pt}                            %Bottom line

    \lhead{                                                         %Left Header
        \hyperlink{chapter.\arabic{chapter}}                        %Make a link to the current chapter
        {\normalsize{\textsc{\nouppercase{\leftmark}}}}             %And fot it put the name
    }

    \rhead{                                                         %Right Header
        \hyperlink{section.\arabic{chapter}.\arabic{section}}       %Make a link to the current chapter
            {\footnotesize{\textsc{\nouppercase{\rightmark}}}}      %And fot it put the name
    }

    \rfoot{\textsc{\small{\hyperref[sec:Index]{Ve al Índice}}}}    %This will always be a footer  

    \fancyfoot[L]{                                                  %Algoritm for a changing footer
        \ifthenelse{\isodd{\value{page}}}                           %IF ODD PAGE:
            {\href{https://compilandoconocimiento.com/yo/}          %DO THIS:
                {\footnotesize                                      %Send the page
                    {\textsc{Oscar Andrés Rosas}}}}                 %Send the page
            {\href{https://compilandoconocimiento.com}              %ELSE DO THIS: 
                {\footnotesize                                      %Send the author
                    {\textsc{Compilando Conocimiento}}}}            %Send the author
    }
    
    
    
% ========================================
% ===========   COMMANDS    ==============
% ========================================

    % =====  NUMBER THEORY  ==========
    \DeclareMathOperator \Naturals {\mathbb{N}}                     %Use: \Naturals por Notation
    \DeclareMathOperator \Integers {\mathbb{Z}}                     %Use: \Integers por Notation
    \DeclareMathOperator \Racionals{\mathbb{Q}}                     %Use: \Racionals por Notation
    \DeclareMathOperator \Reals {\mathbb{R}}                        %Use: \Reals por Notation
    \DeclareMathOperator \Complexs {\mathbb{C}}                     %Use: \Complex por Notation
    \DeclareMathOperator \LinealTransformation {\mathcal{T}}        %Use: \LinealTransformation for a cool T
    \DeclareMathOperator \Space {\quad}                             %Use: \Space for a cool space

    % =====  VECTORS  ================
    \newcommand{\pVector}[1]{                                       %Use: \pVector {Matrix Notation} use parentesis
        \ensuremath{\begin{pmatrix}#1\end{pmatrix}}                 %Example: \pVector{a\\b\\c} or \pVector{a&b&c} 
    }
    \newcommand{\lVector}[1]{                                       %Use: \lVector {Matrix Notation} use a abs 
        \ensuremath{\begin{vmatrix}#1\end{vmatrix}}                 %Example: \lVector{a\\b\\c} or \lVector{a&b&c} 
    }
    \newcommand{\Vector}[1]{                                        %Use: \Vector {Matrix Notation} no parentesis
        \ensuremath{\begin{matrix}#1\end{matrix}}                   %Example: \Vector{a\\b\\c} or \Vector{a&b&c}
    }



% =====================================================
% ============     	  COVER PAGE	   ================
% =====================================================
\begin{document}
\begin{titlepage}

	\center
	% ============ UNIVERSITY NAME AND DATA =========
	\textbf{\textsc{\Large Proyecto Compilando Conocimiento}}\\[1.0cm] 
	\textsc{\Large Matemáticas Discretas}\\[1.0cm] 

	% ============ NAME OF THE DOCUMENT  ============
	\rule{\linewidth}{0.5mm} \\[1.0cm]
		{ \huge \bfseries Teoría de Conjuntos}\\[1.0cm] 
	\rule{\linewidth}{0.5mm} \\[2.0cm]
	
	% ====== SEMI TITLE ==========
	{\LARGE Una Pequeña Introducción}\\[7cm] 
	
	% ============  MY INFORMATION  =================
	\begin{center} \large
	\textbf{\textsc{Autor:}}\\
	Rosas Hernandez Oscar Andres
	\end{center}

	\vfill

\end{titlepage}

% =====================================================
% ========                INDICE              =========
% =====================================================
\tableofcontents{}
\label{sec:Index}

\clearpage

% ======================================================================================
% =============================       PRINCIPIOS BASICOS      ==========================
% ======================================================================================
\chapter{Principios Básicos}
    \clearpage

    % =====================================================
    % ============           DEFINICION            ========
    % =====================================================
    \section{Definición}

        % =====================================
        % =========   ¿QUE SON?     ===========
        % =====================================
        \subsection{¿Qué son?}
            Olvida todo lo que sabes sobre números. Olvídate de que sabes lo que es un número.
            Aquí es donde empiezan las matemáticas. En vez de matemáticas con números,
            vamos a hacer matemáticas con 'cosas'.

            Se denomina conjunto a la agrupación de entes o elementos, que poseen una o
            varias características en común. 


            \subsubsection{Ideas Importantes}

            Un conjunto puede ser una agrupación de números, de vectores, de autos, de espacios vectoriales,
            de objectos, de funciones e incluso un conjunto puede ser una agrupación de otros conjuntos.

            Lo conjuntos generalmente son denotados por letras mayúsculas, como $A, B, C, \cdots$


            Quiza los conjuntos más importantes que verás a lo largo de estos apuntes son:

            \begin{itemize}
                \item $\Naturals$ : Representa al conjunto de todos los naturales, ya sabes
                        números como \\$1, 2, 3, \cdots$
                \item $\Integers$ : Representa al conjunto de todos los enteros, ya sabes
                        números como \\$\cdots, -1, 0, 1, \cdots$
                \item $\Racionals$ : Representa al conjunto de todos los racionales, ya sabes
                        números como \\$\frac{1}{1}, \frac{1}{2}, \frac{23}{72}, \cdots$
                \item $\Reals$ : Representa al conjunto de todos los reales, ya sabes
                        números como \\$\frac{3}{2}, -\pi, 3, \cdots$
                \item $\Complexs$ : Representa al conjunto de todos los complejos, ya sabes
                        números como \\$3 + 2i, \pi i, 3 \cdots$
            \end{itemize}

        % =====================================
        % =========  ¿COMO DEFINIRLO?  ========
        % =====================================
        \clearpage
        \subsection{¿Cómo Definirlo?}

            Definimos cierto conjunto, al que llamaremos $A$ como la agrupación de todas las $x$
            (es decir cada x es un elemento, un ente) que cumplen ciertas características
            (eso es lo que significa esos puntitos, ahí deberías poner las reglas que tenga
            tu conjunto). 

            \begin{equation}   
                A = \{ x |\quad x \dots \}
            \end{equation}

            Recuerda que basicamente hay dos formas de 'declarar' un conjunto:
            \begin{itemize}
                \item \textbf{Explícitamente}:
                    Es decir, enumerando TODOS los elementos o entes que forman
                    el conjunto ($A = \{a, e, i, o, u\}$)

                \item \textbf{Implícitamente}:
                Es decir, enumerando las características de los elementos o entes que
                forman el conjunto ($B = \{ x \in \Naturals |\quad |\sqrt{x}| \in \Naturals \}$)
            \end{itemize}


            Recuerda también:
            \begin{itemize}
                \item Los elementos repetidos no cuentan,
                    si ya esta un elemento dentro del conjunto, da lo mismo que lo vuelvas
                    a enumerar.\\
                    ($A = \{a, e, i, o, u\} = \{a, a, e, i, o, u\}$)

                \item No importa el orden en el me muestres los elementos,
                    solo importa que esten dentro.
                    ($A = \{a, e, i, o, u\} = \{u, a, i, e, o\}$)
            \end{itemize}



            \clearpage
            % ==================
            % ===  EJEMPLO   ===
            % ==================
            \subsubsection{\large \\Ejemplo 1:}

                Veamos por ejemplo como definir el Conjunto $C_1$ \emph{(lo sé me muero con
                mi creatividad para los nombres)} como aquel que contenga a TODOS los números
                reales negativos:

                \begin{equation*}   
                    C_1 = \{ x \in \Reals |\quad x < 0 \}
                \end{equation*}

                \textbf{En Lenguaje normal:\\}
                Esto lo podemos leer como $C_1$ es el conjunto \emph{(es decir todo lo que esta
                entre parentesís)} de todas las $x$ que pertenezcan al los números reales
                \emph{(eso quiere decir el $x \in \Reals$)} tal que (eso lo representamos
                como: $|$ , no porque :/) $x$ es menor que 0 \emph{(esa es nuestra condición para
                encontrar si alguna $x$ pertenece a nuestro conjunto)}.

            % ==================
            % ===  EJEMPLO   ===
            % ==================
                \subsubsection{\large \\Ejemplo 2:}

                Veamos por ejemplo como definir el Conjunto $C_2$ como aquel que contenga a
                TODOS las vocales:

                \begin{equation*}   
                \begin{split}   
                    C_2 &= \{ Vocales \}            \\
                    C_2 &= \{a, e, i, o, u \}
                \end{split}   
                \end{equation*}

                Si te das cuenta, podemos definirlos de muchas maneras.




    % =====================================================
    % ============        CLASIFICACION            ========
    % =====================================================
    \clearpage
    \section{Clasificación}
            
        Podemos clasificar de muchas maneras a los conjuntos, veamos las más comunes:

        \textbf{\large \\Tamaño}

        \begin{itemize}
            \item \textbf{Finito}:
                Si tiene una colección que se pueda contar, aunque sea difícil.

                Por ejemplo, el conjunto de juguetes incluye todos los tipos de
                juguetes que hay en el mundo. Aunque sea difícil, se podrían contar
                todos los tipos de juguetes del mundo, por lo que es finito.

            \item \textbf{Infinito}:
                Si tiene una colección que no se pueda terminar de contar nunca.

                Por ejemplo, el conjunto de todos los números pares, que son
                infinitos, es un conjunto infinito.
        \end{itemize}


    % =====================================================
    % ============    CONJUNTO VACIO/UNIVERSO      ========
    % =====================================================
    \clearpage
    \section{Conjunto Vacío}
            
        Ok, ya sabemos que un conjunto es un grupo de elementos, pero ...
        ¿Cómo represento a un conjunto en el que no hay nada?

        Como una caja vacía.

        De hecho, me gusta, hablemos de el Conjunto vacío como un caja vacía.
        \begin{equation}   
            \phi = \{ \}
        \end{equation}

        Solemos usar este simbolo por su parecido con un cero, pero recuerda no es un cero, simplemente
        es una forma de denotar al conjunto vacío.

        Listo, eso es casí todo, además te gustará que te recuerde las siguientes
        proposiciones:

        \begin{itemize}
            \item $|\phi| = 0$ : 
                    Esto quiere decir que la cardinalidad \emph{(es decir
                    la cantidad de elementos)} del conjunto vacío es la misma que 
                    la cantidad de galletas en una caja vacía de galletas, osea 0.

            \item $\phi \neq \{\phi\}$: 
                    Esto quiere decir que no es lo mismo hablar del conjunto
                    vacío que de hablar de un conjunto cualquiera que contiene al
                    conjunto vacío.

                    Es decir simplemente no es lo mismo tener una caja vacía que una caja
                    con una caja vacía dentro \emph{(si lo piensas la segunda caja ya 
                    no esta completamente vacía)}
        \end{itemize}

    \section{Conjunto Universo}
        
        Como podemos imaginarnos, tenía que existir un término inverso, digamos que estamos
        analizando y agrupando animales por su habitad, entonces tenemos muchos conjuntos cool
        como animales del bosque o marinos, pero también tenemos a un mega conjunto que llamamos
        universo donde tenemos a todos los animales.

        Muchas veces a la hora de hablar sobre conjuntos solemos definirlos sobre un universo.

        Podemos ver muchasc analogías, veamos a ver cual te gusta más:

        \begin{itemize}
            \item Es como si el universo fuera el padre, entonces las hijas son nuestros conjuntos
            \item Es como si el universo una caja de cereal, entonces nuestros conjuntos son grupos de cereales
                que estaban dentro de la caja, ya que es imposible sacar mas cereales más que los que estaban
                en la caja.
        \end{itemize}

        Creo que resulta bastante obvio pero aquí hay algunas cosas que quizá te interesen.
        \begin{itemize}
            \item $\phi^c = u$
            \item $u^c = \phi$
        \end{itemize}





% ======================================================================================
% ============================      ALGEBRA CON CONJUNTOS     ==========================
% ======================================================================================
\chapter{Álgebra y Operaciones}
    \clearpage

    % =====================================================
    % ============           OPERACIONES           ========
    % =====================================================
    \clearpage
    \section{Operaciones}

        Podemos hacer operaciones con los conjuntos de una manera muy similiar a como hacemos
        operaciones con los números normales, tu defines una operación, y la haces entre dos conjuntos
        y esta te dará un nuevo conjunto,  pero aquí siento que son incluso más divertidas.

        Te presentaré estas operaciones una a una y te mostraré con mas detalle cada una, así 
        que vamos, empecemos:
            
        \subsection{Unión}

            La unión representa la fuerza, así que veamos que representa la unión cuando estemos
            hablando de conjuntos.

            \begin{equation}
                A \cup B = \{ x \Space | x \in A \text{ ó bien } x \in B \}
            \end{equation}

            Esta operación basicamente nos da un conjunto en el que estan \textbf{todos} los elementos
            que bien pertenezcan a A o bien que pertenezcan a B.




    % =====================================================
    % ============           CARDINALIDAD          ========
    % =====================================================
    \clearpage
    \section{Cardinalidad}
            
        Ok, vamos avanzando, ahora es la hora de ver una característica de los conjuntos.
        La Cardinalidad, que no es mas que una forma \emph{fancy} de decir el número de 
        elementos ó entes que contiene cierto conjunto.

        Puedes verlo como una función que
        recibe un conjunto cualquiera y te regresa un número \emph{(Bueno, tecnicamente también
        esta el caso en el que la cardinalidad es infinita)}.

        Esta es la forma en que solemos expresar la cardinalidad de un conjunto cualquiera:
        \begin{equation}   
            |A| = \#A = Card(A)
        \end{equation}


        Ahora si, veamos algunas proposiciones super geniales:






% =====================================================
% ============        BIBLIOGRAPHY   ==================
% =====================================================
\clearpage
\bibliographystyle{plain}
	\begin{thebibliography}{9}

	% ============ REFERENCE #1 ========
	\bibitem{Sitio1} 
		ProbRob
		\\\texttt{Youtube.com}


	 

\end{thebibliography}



\end{document}