% *************************************************************************************************************************
% *************************************     	MATEMATICAS DISCRETAS   	  *********************************************
% *************************************************************************************************************************


% =======================================================
% =======         HEADER FOR DOCUMENT        ============
% =======================================================
    % *********   DOCUMENT ITSELF   **************
    \documentclass[12pt]{report}                                    %Type of docuemtn and size of font
    \usepackage[margin=1.2in]{geometry}                             %Margins and Geometry pacakge
    \usepackage{ifthen}                                             %Allow simple programming
    \usepackage{hyperref}                                           %Create MetaData for a PDF and LINKS!
    \setlength{\parindent}{0pt}                                     %Eliminate ugly indentation
    \author{Oscar Andrés Rosas}                                     %Who I am

    % *********   LANGUAJE AND UFT-8   *********
    \usepackage[spanish]{babel}                                     %Please use spanish
    \usepackage[utf8]{inputenc}                                     %Please use spanish - UFT
    \usepackage[T1]{fontenc}                                        %Please use spanish
    \usepackage{textcmds}                                           %Allow us to use quoutes

    % *********   MATH AND HIS STYLE  *********
    \usepackage{amsthm, amssymb, amsfonts, mathrsfs}                %Make math beautiful
    \usepackage[fleqn]{amsmath}                                     %Please make equations left
    \usepackage{centernot}                                          %Allow me to negate a symbol
    \decimalpoint                                                   %Use decimal point

    % *********   GRAPHICS AND IMAGES *********
    \usepackage{graphicx}                                           %Allow to create graphics
    \usepackage{wrapfig}                                            %Allow to create images
    \graphicspath{ {Graphics/} }                                    %Where are the images :D

    % *********   LISTS AND TABLES ***********
    \usepackage{listings}                                           %We will be using code here
    \usepackage[inline]{enumitem}                                   %We will need to enumarate
    \usepackage{tasks}                                              %Horizontal lists
    \usepackage{longtable}                                          %Lets make tables awesome
    \usepackage{booktabs}                                           %Lets make tables awesome
    \usepackage{tabularx}                                           %Lets make tables awesome
    \usepackage{multirow}                                           %Lets make tables awesome
    \usepackage{multicol}                                           %Create multicolumns

    % *********   HEADERS AND FOOTERS ********
    \usepackage{fancyhdr}                                           %Lets make awesome headers/footers
    \pagestyle{fancy}                                               %Lets make awesome headers/footers
    \setlength{\headheight}{16pt}                                   %Top line
    \setlength{\parskip}{0.5em}                                     %Top line
    \renewcommand{\footrulewidth}{0.5pt}                            %Bottom line

    \lhead{                                                         %Left Header
        \hyperlink{chapter.\arabic{chapter}}                        %Make a link to the current chapter
        {\normalsize{\textsc{\nouppercase{\leftmark}}}}             %And fot it put the name
    }

    \rhead{                                                         %Right Header
        \hyperlink{section.\arabic{chapter}.\arabic{section}}       %Make a link to the current chapter
            {\footnotesize{\textsc{\nouppercase{\rightmark}}}}      %And fot it put the name
    }

    \rfoot{\textsc{\small{\hyperref[sec:Index]{Ve al Índice}}}}    %This will always be a footer  

    \fancyfoot[L]{                                                  %Algoritm for a changing footer
        \ifthenelse{\isodd{\value{page}}}                           %IF ODD PAGE:
            {\href{https://compilandoconocimiento.com/yo/}          %DO THIS:
                {\footnotesize                                      %Send the page
                    {\textsc{Oscar Andrés Rosas}}}}                 %Send the page
            {\href{https://compilandoconocimiento.com}              %ELSE DO THIS: 
                {\footnotesize                                      %Send the author
                    {\textsc{Compilando Conocimiento}}}}            %Send the author
    }
    
    
    
% ========================================
% ===========   COMMANDS    ==============
% ========================================

    % =====  GENERAL TEXT  ==========
    \newcommand \Quote {\qq}                                        %Use: \Quote to use quotes

    % =====  GENERAL MATH  ==========
    \DeclareMathOperator \Space {\quad}                             %Use: \Space for a cool mega space
    \DeclareMathOperator \MiniSpace {\;}                            %Use: \Space for a cool mini space
    \newcommand \Such {\MiniSpace|\MiniSpace}                       %Use: \Such like in sets

    % =====  LOGIC  ==================
    \DeclareMathOperator \doublearrow {\leftrightarrow}             %Use: \doublearrow for a double arrow
    \newcommand \lequal {\MiniSpace \Leftrightarrow \MiniSpace}     %Use: \lequal for a double arrow
    \newcommand \linfire {\MiniSpace \Rightarrow \MiniSpace}        %Use: \lequal for a double arrow

    % =====  NUMBER THEORY  ==========
    \DeclareMathOperator \Naturals {\mathbb{N}}                     %Use: \Naturals por Notation
    \DeclareMathOperator \Integers {\mathbb{Z}}                     %Use: \Integers por Notation
    \DeclareMathOperator \Racionals{\mathbb{Q}}                     %Use: \Racionals por Notation
    \DeclareMathOperator \Reals {\mathbb{R}}                        %Use: \Reals por Notation
    \DeclareMathOperator \Complexs {\mathbb{C}}                     %Use: \Complex por Notation

    % === LINEAL ALGEBRA & VECTORS ===
    \DeclareMathOperator \LinealTransformation {\mathcal{T}}        %Use: \LinealTransformation for a cool T

    \newcommand{\pVector}[1]{                                       %Use: \pVector {Matrix Notation} use parentesis
        \ensuremath{\begin{pmatrix}#1\end{pmatrix}}                 %Example: \pVector{a\\b\\c} or \pVector{a&b&c} 
    }
    \newcommand{\lVector}[1]{                                       %Use: \lVector {Matrix Notation} use a abs 
        \ensuremath{\begin{vmatrix}#1\end{vmatrix}}                 %Example: \lVector{a\\b\\c} or \lVector{a&b&c} 
    }
    \newcommand{\Vector}[1]{                                        %Use: \Vector {Matrix Notation} no parentesis
        \ensuremath{\begin{matrix}#1\end{matrix}}                   %Example: \Vector{a\\b\\c} or \Vector{a&b&c}
    }



% =====================================================
% ============     	  COVER PAGE	   ================
% =====================================================
\begin{document}
\begin{titlepage}

	\center
	% ============ UNIVERSITY NAME AND DATA =========
	\textbf{\textsc{\Large Proyecto Compilando Conocimiento}}\\[1.0cm] 
	\textsc{\Large Matemáticas Discretas}\\[1.0cm] 

	% ============ NAME OF THE DOCUMENT  ============
	\rule{\linewidth}{0.5mm} \\[1.0cm]
		{ \huge \bfseries Matemáticas Discretas}\\[1.0cm] 
	\rule{\linewidth}{0.5mm} \\[2.0cm]
	
	% ====== SEMI TITLE ==========
	{\LARGE Una Pequeña Introducción al corazón de las matemáticas}\\[7cm] 
	
	% ============  MY INFORMATION  =================
	\begin{center} \large
	\textbf{\textsc{Autor:}}\\
	Rosas Hernandez Oscar Andres
	\end{center}

	\vfill

\end{titlepage}

% =====================================================
% ========                INDICE              =========
% =====================================================
\tableofcontents{}
\label{sec:Index}

\clearpage






% ////////////////////////////////////////////////////////////////////////////////////////////////////////////////////
% ////////////////////////////////////////   FUNCIONES Y LÍMITES     /////////////////////////////////////////////////
% ////////////////////////////////////////////////////////////////////////////////////////////////////////////////////
\part{Lógica Matemática}



    % ======================================================================================
    % =============================       PRINCIPIOS BASICOS      ==========================
    % ======================================================================================
    \chapter{Proposiciones y Conectores}
        \clearpage

        % =====================================================
        % ============           DEFINICION            ========
        % =====================================================
        \section{Proposiciones}

            La lógica es una forma sistemática de pensar que nos permite deducir nueva información desde la
            información que ya conocemos.

            Recuerda que la lógica es un proceso de deducir la información correctamente,
            no sólo deducir la información correcta.

            La lógica trabajo con algo llamado proposiciones, son como las funciones para
            cálculo, o los lenguajes de programación para informática o los libros para la literatura.

            Así que empecemos por ahí ... ¿Qué son?


            % =====================================
            % =========   ¿QUE SON?     ===========
            % =====================================
            \subsection*{Definición}
                
                \textbf{Son proposiciones las frases que pueden adquirir un valor de verdadero o falso.}
                
                O dicho de manera formal:

                \textbf{Es una oración aseverativa de la que tiene sentido decir que es verdadera o falsa}.\\

                Y cuando digo frase, me refiero a:
                \begin{itemize}
                    \item Secuencia finita de signos con significado y sentido de ser calificado como verdadero o falso.
                            (es decir una simple expresión matemática).

                    \item Expresión lingüística susceptible de ser calificada de verdadera o falsa.
                            (es decir una frase aseverativa).
                \end{itemize}


            % =====================================
            % ====   SENTENCIAS ABIERTAS    =======
            % =====================================
            \subsection*{Sentencias Abiertas}
                Existen cosas que son parecidas a las proposiciones, pero no lo son exactamente, son cosas como:

                $p(x)$: $x$ es un número par.

                Puesto que la validez de $p(x)$ depende que número sea $x$, $p(x)$no es no totalmente cierta ni
                totalmente falsa, por lo tanto no es una proposición.

                Una oración como esta, cuya verdad depende del valor de una o más variables,
                se llama sentencias abierta.



                % ==========================
                % =====   EJEMPLOS   =======
                % ==========================
                \clearpage
                \subsection*{Ejemplo}


                    Por ejemplo son proposiciones frases como:
                    \begin{itemize}
                        \item $2 + 3 = 4$
                        \item Hay solamente 325 personas en Marte
                        \item $\forall x, y \in \Naturals$ se tiene que $\MiniSpace x+y \in \Reals$
                        \item Hoy es lunes
                        \item $f(x+y) = f(x) + f(y)$
                        \item Si x = 2 entonces 2x = 4
                    \end{itemize}

                    Pero no son cosas como:
                    \begin{itemize}
                        \item ¡Ojalá no llueva hoy!
                        \item Haz la tarea
                        \item Este enunciado es falso
                        \item Tomar una siesta
                    \end{itemize}



            % =====================================
            % ======   CLASIFICACION    ===========
            % =====================================
            \clearpage
            \subsection{Teoremas, Corolario y Tautológias}
                
                \subsubsection*{Clasificación de Propiedades}

                    \begin{itemize}
                        \item \textbf{Tautología}: Cuando para todos los valores posibles de un conjunto
                                de proposiciones siempre será verdadero el conjunto.

                        \item \textbf{Contradicción}: Cuando para todos los valores posibles de un conjunto
                        de proposiciones esta será siempre falso.

                        \item \textbf{Contingencia}: Una proposición “común” son básicamente todas las que
                        no son ni tautologías ni contradicciones.
                    \end{itemize}

                \subsubsection*{Notación}

                    Además a los matemáticas les encanta demostrar todo y cuando digo todo, es TODO, así que
                    aquí te dejo las diferencias entre varias palabras que se parecen:

                    \begin{itemize}
                        \item \textbf{Proposición}: Enunciado que encierra un valor de verdad.

                        \item \textbf{Axioma}: Principio tan claro y evidente que no necesita demostración.

                        \item \textbf{Corolario}: Proposición demostrado que provoca una afirmación.

                        \item \textbf{Demostración}: Razonamiento por el cuál se da prueba de la
                            exactitud de una proposición.

                        \item \textbf{Lema}: Proposición que es necesaria demostrar antes de
                        establecer un teorema.

                    \end{itemize}


        % =====================================================
        % ============           CONECTORES            ========
        % =====================================================
        \clearpage
        \section{Conectores Lógicos}

            Los conectores nos permiten 'concatenar' proposiciones o crear proposiciones mas avanzadas. 
            Veamos primero como solemos mostrarlos:

            \begin{longtable}{p{35mm} || p{30mm} || p{80mm}}

                % ==== HEADERS ============
                \textbf{\large Conector}
                &
                \textbf{\large Nombres}
                &
                \textbf{\large Símbolos}
                \\[1.5ex]
                \hline\hline
                & & \\                                                                    
                \endhead                                                     

                \large y  &  \large{p $\land$ q}                                    &

                \begin{minipage}[t]{\textwidth}\begin{itemize}
                    \item \textbf{Conjunción de} p \textbf{y de} q
                \end{itemize}\end{minipage}                                                 \\[1.5ex]
                
                \hline & & \\ \large o  &  \large{p $\lor$ q}                               &
                
                \begin{minipage}[t]{\textwidth}\begin{itemize}
                    \item \textbf{Disyunción de} p \textbf{y de} q
                \end{itemize}\end{minipage}                                                 \\[1.5ex]


                \hline & & \\ \large no  &  \large{$\lnot$ q}                               &

                \begin{minipage}[t]{\textwidth}\begin{itemize}
                    \item \textbf{Negación de} P
                \end{itemize}\end{minipage}                                                 \\[1.5ex]

                \hline & & \\ \large implica  &  \large{p $\to$ q}                          & 

                \begin{minipage}[t]{\textwidth}\begin{itemize}
                \small{
                    \item p \textbf{implica} q
                    \item \textbf{Si} p\textbf{, entonces} q
                    \item q \textbf{si} p
                    \item \textbf{Sólo si} q \textbf{entonces} p
                    \item p \textbf{sólo si} q
                    \item \textbf{Cuando} p\textbf{,} q
                    \item \textbf{Siempre que} q\textbf{,} p
                    \item q \textbf{siempre que} p
                    \item p \textbf{es una condición suficiente para} q
                    \item q \textbf{es una condición necesaria para} p
                    \item \textbf{Es necesario que} q \textbf{para} p
                    \item \textbf{Es suficiente que} p \textbf{para que} q
                }\\
                \end{itemize}\end{minipage}                                                 \\[1.5ex]

                \hline & & \\ \large si y solo si  &  \large{p $\doublearrow$ q}            &

                \begin{minipage}[t]{\textwidth}\begin{itemize}
                \small{
                    \item p \textbf{ssi} q
                    \item p \textbf{es equivalente a} q
                    \item p \textbf{es una condición necesaria y suficiente para} q
                    \item \textbf{Para que} p \textbf{es necesario y suficiente que} q
                }\\
                \end{itemize}\end{minipage}                                                 \\
     
            \end{longtable}

            % =====================================
            % ======     NEGACION     ===========
            % =====================================
            \clearpage

            Las que siguen a continuación son lo que yo denomino más operaciones mas básicas en lógica.

            \subsection{Negación}

                Devuelve el inverso del valor de verdad de la proposición que le pases.\\

                \begin{tabular}{ |c|c|c| } 
                    \hline &&\\
                    \large{Nombre} & $p$ & $\lnot p$ \\[0.5em]
                    \hline
                    \multirow{2}{5em}{Negación}
                    & $F$ & $V$  \\ \cline{2-3}
                    & $V$ & $F$  \\ \cline{2-3}
                    \hline
                \end{tabular}

            % =====================================
            % ======     CONJUNCION     ===========
            % =====================================
            \subsection{Conjunción}

                Devuelve verdadero \textbf{solo} cuando ambas son verdaderas, y falso en cualquier
                otra combinación.\\

                \begin{tabular}{ |c|c|c|c|c| } 
                    \hline &&&\\
                    \large{Nombre} & $p$ & $q$ & $p \land q$ \\[0.5em]
                    \hline
                    \multirow{4}{5em}{Conjunción}
                    & $F$ & $F$ & $F$ \\ \cline{2-4}
                    & $F$ & $V$ & $F$ \\ \cline{2-4}
                    & $V$ & $F$ & $F$ \\ \cline{2-4}
                    & $V$ & $V$ & $V$ \\ 
                    \hline
                \end{tabular}


            % =====================================
            % ======     DISYUNCION     ===========
            % =====================================
            \subsection{Disyunción}

                Devuelve falso \textbf{solo} cuando ambas son falsas, y verdadero en cualquier
                otra combinación.\\

                \begin{tabular}{ |c|c|c|c|c| } 
                    \hline &&&\\
                    \large{Nombre} & $p$ & $q$ & $p \lor q$ \\[0.5em]
                    \hline
                    \multirow{4}{5em}{Disyunción}
                    & $F$ & $F$ & $F$ \\ \cline{2-4}
                    & $F$ & $V$ & $V$ \\ \cline{2-4}
                    & $V$ & $F$ & $V$ \\ \cline{2-4}
                    & $V$ & $V$ & $V$ \\ 
                    \hline
                \end{tabular}





            % =====================================
            % ======     IMPLICACION    ===========
            % =====================================
            \clearpage
            \subsection{Implicación}

                Devuelve falso \textbf{solo} cuando la primera premisa es verdadera, pero la 
                segunda es falsa, y verdadero en cualquier otra combinación.

                Ve a $p \to q$ como una promesa de que siempre que $p$ es verdadera, $q$ será verdadera también.
                Sólo hay una manera de romper esta promesa y que es si $P$ sea verdad y $q$ es falso.\\


                \begin{tabular}{ |c|c|c|c|c| } 
                    \hline &&&\\
                    \large{Nombre} & $p$ & $q$ & $p \to q$ \\[0.5em]
                    \hline
                    \multirow{4}{5em}{Disyunción}
                    & $F$ & $F$ & $V$ \\ \cline{2-4}
                    & $F$ & $V$ & $V$ \\ \cline{2-4}
                    & $V$ & $F$ & $F$ \\ \cline{2-4}
                    & $V$ & $V$ & $V$ \\ 
                    \hline
                \end{tabular}\\[1.0em]

                \subsubsection*{Ideas Importantes}

                    La implicación es creo yo la más importante de todas, y no es porque sea básica, 
                    es más: $p \to q$ es totalmente equivalente a $\lnot p \lor q$.

                    Usando la implicación hay algunas cosas famosas que deberías saber:

                    \begin{itemize}
                        \item \textbf{Contrapositiva del Condicional}
                                Esta equivalencia es muy importante, pues es muy usada para las demostraciones
                                (no te preocupes Timmy, ya entenderas después).
                                \begin{equation*}
                                    p \to q \lequal \lnot q \to \lnot p
                                \end{equation*}

                        \item \textbf{Implicaciones Famosas}
                                No se a quién se le ocurrio ponerles nombres, pero creo que te combiene
                                que las conozcas.

                                \begin{tabular}{ |c|c|c| } 
                                    \hline &&\\
                                    \large{Nombre} & \large{Forma} & \large{Es equivalente con...}      \\[0.5em]
                                    \hline \hline
                                    
                                    \textbf{Condicional}    & $p \to q$             & Contrapositiva    \\ \hline
                                    \textbf{Contrapositiva} & $\lnot q \to \lnot p$ & Condicional       \\ \hline\hline 
                                    
                                    \textbf{Recíproca}      & $q \to p$             & Inversa           \\ \hline
                                    \textbf{Inversa}        & $\lnot p \to \lnot q$ & Recíproca         \\ \hline
                                \end{tabular}
                    \end{itemize}


            % =====================================
            % ======     BICONDICIONAL    =========
            % =====================================
            \clearpage
            \subsection{Bicondicional}

                En lógica la idea de $(p \to q ) \land (q \to p)$ aparece tan seguido que decidimos darle su
                propio símbolo $p \doublearrow q$.

                Esta operación nos regresa verdadero \textbf{solo} cuando ambas premisas tengan el mismo valor de
                verdad. Ojo no dije que ambas sean verdad, simplemente que si una es falsa, obliga a la otra a 
                ser falsa.

                Recuerda que sabemos que $p \to q$ se lee como 'p si q' y $q \to p$ se lee como 'p solo si q'.
                Entonces nuestro nuevo operador recibe el original nombre de 'p si y solo si q' o de forma normal
                'p ssi q'.



        % =====================================================
        % ============           DEFINICION            ========
        % =====================================================
        \section{Equivalente Lógico}

            Llega a pasar en lógica que tenemos dos expresiones lógicas que al momento de ver su tabla de 
            verdad vemos que son iguales en todos los valores de verdad de sus variables entonces podemos
            decir que son logicamente equivalentes. Y solemos denotar eso con este símbolo $p \lequal q$.

            Usamos este símbolo porque si $p$ y $q$ son logicamente equivalentes entonces $p \doublearrow q$
            será siempre verdad, una tautología.

            Esta idea es importante pues nos permite ver ideas que ya tenemos expresadas en la lógica de una
            manera completamente nueva si es que a nosotros nos convienen más.

            A continuación te muestro una tabla con las equivalencias lógicas mas comúnes.






        % =====================================================
        % ============    LEYES DE LOGICA            ==========
        % =====================================================
        \clearpage
        \section{Leyes de Lógica}
                    
                Sean $p, q, r$ sentencias lógicas y sea $T$ una tautológía y sea $F$ una contradicción.

                \begin{itemize}
                    \item \textbf{Doble Complemento} \\
                            $\lnot(\lnot p) \lequal p$

                    \item \textbf{Propiedad Conmutativa}
                        \begin{itemize}
                            \item $p \land q \lequal q \land p$
                            \item $p \lor  q \lequal q \lor  p$
                        \end{itemize}

                    \item \textbf{Propiedad Asociativa}
                        \begin{itemize}
                            \item $p \land (q \land r) \lequal (p \land q) \land C$
                            \item $p \lor (q \lor r) \lequal (p \lor q) \lor C$
                        \end{itemize}

                    \item \textbf{Propiedad Distributiva}
                        \begin{itemize}
                            \item $p \land (q \lor r) \lequal (p \land q) \lor (p \land r)$
                            \item $p \lor (q \land r) \lequal (p \lor q) \land (p \lor r)$
                        \end{itemize}

                    \item \textbf{Leyes de Morgan}
                        \begin{itemize}
                            \item $\lnot (p \land q) \lequal (\lnot p) \lor (\lnot q)$
                            \item $\lnot (p \lor q) \lequal (\lnot p) \land (\lnot q)$
                        \end{itemize}

                    \clearpage

                    \item \textbf{Propiedad de los Neutros}
                        \begin{itemize}
                            \item $p \land T \lequal p$
                            \item $p \lor F \lequal p$
                        \end{itemize}

                    \item \textbf{Propiedad de los Inversos}
                        \begin{itemize}
                            \item $p \land \lnot p \lequal F$
                            \item $p \lor \lnot p \lequal T$
                        \end{itemize}

                    \item \textbf{Propiedad de Dominación}
                        \begin{itemize}
                            \item $p \land F \lequal F$
                            \item $p \lor T \lequal T$
                        \end{itemize}

                    \item \textbf{Propiedad de Inepotencia}
                        \begin{itemize}
                            \item $p \land p \lequal p$
                            \item $p \lor p \lequal p$
                        \end{itemize}

                    \item \textbf{Propiedad de Absorción}
                        \begin{itemize}
                            \item $p \land (p \lor q) \lequal p$
                            \item $p \lor (p \land q) \lequal p$
                        \end{itemize}

                    \item \textbf{Propiedad de Contrapositiva}
                        \begin{itemize}
                            \item $p \to q \lequal \lnot q \to \lnot p$
                            \item $p \doublearrow q  \lequal \lnot p \doublearrow \lnot q$
                        \end{itemize}

                \end{itemize}




    % ======================================================================================
    % ========================              INFERENCIAS             ========================
    % ======================================================================================
    \clearpage
    \chapter{Inferencias Lógicas}
        \clearpage

        
        % =====================================================
        % ============           INFERENCIAS           ========
        % =====================================================
        \section{Inferencias Lógicas}

            La inferencia es la forma en la que obtenemos conclusiones en base a datos y declaraciones establecidas.
            Esto se va a poner intenso, pero creo que esta definición vale la pena:

            \subsubsection{Definición}

                Podemos entonces definir que una inferencia lógica es un proposición $q$ que si le aplicamos
                el condicional con la disyunción de todas las premisas sería una tautología.

                Es decir:
                \begin{equation}
                    [p_1 \land p_2 \land p_3 \dots \to q] \lequal T
                \end{equation}

                En español esto quiere decir que el hecho de que todas las premisas sean verdaderas obliga a
                que $q$ sea verdadera, o en otra manera podemos decir que la inferencia lógica como: 
                Dadas dos afirmaciones verdaderas podemos inferir que una tercera afirmación es verdadera.


            \subsubsection{Ejemplo}
                Supongamos que sabemos que una afirmación de la forma $p \to q$ es verdadera.
                Esto nos dice que siempre que $p$ es verdadera, $q$ también será verdadera. 

                Por sí mismo, $p \to q$ siendo verdadero no nos dice que $p$ o $q$ es verdadero
                (ambos podrían ser falsos, o $p$ podría ser falso y $q$ verdadero).

                Sin embargo, si además sabemos que $p$ es verdadera entonces debe ser que $q$ es verdadera.
                Esto se llama una inferencia lógica: dadas dos afirmaciones verdaderas podemos inferir que una
                tercera afirmación es verdadera.


            % =====================================
            % =====     INFERENCIAS BASICAS =======
            % =====================================
            \clearpage
            \subsection{Inferencias Básicas}

                Hay unas inferencias my importantes, sobretodo a la hora de demostrar algo, por eso les deje su propia
                sección:

                \begin{itemize}
                    \item
                        \textbf{Contrapositiva de la Inferencia}\\
                        $p \linfire q$ si y solo si $\lnot q \linfire \lnot p$

                    \item
                        \textbf{Por Contradicción}\\
                        $p \linfire q$ si y solo si $p \land \lnot q \linfire F$

                    \item
                        \textbf{Por Corriento del Condicional}\\
                        $p \linfire q \to r$ si y solo si $p \land q \linfire s$


                    \item
                        \textbf{Disyunción}\\
                        Si ya sabemos que $p \linfire q$ entonces sabremos que $p \land r \linfire q \land r$

                    \item
                        \textbf{Conjunción}\\
                        Si ya sabemos que $p \linfire q$ entonces sabremos que $p \lor r \linfire q \lor r$

                    \item
                        \textbf{Condicional}\\
                        Si ya sabemos que $p \linfire q$ entonces sabremos que $r \to p \linfire r \to q$

                    \item
                        \textbf{Transitiva}\\
                        Si ya sabemos que $p \linfire q$ y que $q \linfire r$ entonces sabremos que $p \linfire r$
                \end{itemize}        

               



        % ====================================================
        % ============    LEYES DE INFERENCIAS    ============
        % ====================================================
        \clearpage
        \section{Reglas de Inferencias}

            Hay unas inferencias my importantes, casi casi reglas, se las mostraré a continuación:\\

            \begin{multicols}{2}
                
                \large{\textbf{Modus Poness (PP)}}
                    \begin{equation*}
                    \begin{split}
                        &p \to q        \\
                        &p              \\
                        \midrule
                        &\therefore q
                    \end{split}
                    \end{equation*}

                \large{\textbf{Modus Tollens (TT)}}
                    \begin{equation*}
                    \begin{split}
                        &p \to q        \\
                        &\lnot q        \\
                        \midrule
                        &\therefore \lnot p
                    \end{split}
                    \end{equation*}

            \end{multicols}

            \bigskip

            \begin{multicols}{2}
                
                \large{\textbf{Silogismo Hipotético}}
                    \begin{equation*}
                    \begin{split}
                        &p \to q        \\
                        &q \to r        \\
                        \midrule
                        &\therefore p \to r
                    \end{split}
                    \end{equation*}

                \large{\textbf{Silogismo Disyuntivo}}
                    \begin{equation*}
                    \begin{split}
                        &p \lor q       \\
                        &\lnot p        \\
                        \midrule
                        &\therefore q
                    \end{split}
                    \end{equation*}

            \end{multicols}

            \bigskip

            \begin{multicols}{2}
                
                \large{\textbf{Amplificación Disyuntiva}}
                    \begin{equation*}
                    \begin{split}
                        &p              \\
                        \midrule
                        &\therefore p \lor q
                    \end{split}
                    \end{equation*}

                \large{\textbf{Simplificación Conjuntiva}}
                    \begin{equation*}
                    \begin{split}
                        &p \land q      \\
                        \midrule
                        &\therefore p
                    \end{split}
                    \end{equation*}

            \end{multicols}



            \clearpage

            \large{\textbf{Regla de Conjunción}}
            \begin{equation*}
            \begin{split}
                &p                  \\
                &q                  \\
                \midrule
                &\therefore p \land q
            \end{split}
            \end{equation*}


            \bigskip

            \begin{multicols}{2}
                
                \large{\textbf{Ley del Dilema Constructivo}}
                    \begin{equation*}
                    \begin{split}
                        &p \to q             \\
                        &r \to s             \\
                        &p \land r           \\
                        \midrule
                        &\therefore q \land r
                    \end{split}
                    \end{equation*}

                \large{\textbf{Ley del Dilema Constructivo}}
                    \begin{equation*}
                    \begin{split}
                        &p \to q             \\
                        &r \to s             \\
                        &p \lor r            \\
                        \midrule
                        &\therefore q \lor s
                    \end{split}
                    \end{equation*}

            \end{multicols}

            \bigskip

            \begin{multicols}{2}
                
                \large{\textbf{Ley del Dilema Destructivo}}
                    \begin{equation*}
                    \begin{split}
                        &p \to q                \\
                        &r \to s                \\
                        &\lnot q \land \lnot s  \\
                        \midrule
                        &\therefore \lnot p \land \lnot r
                    \end{split}
                    \end{equation*}

                \large{\textbf{Ley del Dilema Destructivo}}
                    \begin{equation*}
                    \begin{split}
                        &p \to q                \\
                        &r \to s                \\
                        &\lnot q \lor \lnot s   \\
                        \midrule
                        &\therefore \lnot p \lor \lnot r
                    \end{split}
                    \end{equation*}

            \end{multicols}




    % ======================================================================================
    % ========================            CUANTIFICADORES           ========================
    % ======================================================================================
    \clearpage
    \chapter{Cuantificadores Lógicos}
        \clearpage

        % =====================================================
        % ============      CUANTIFICADORES            ========
        % =====================================================
        \clearpage
        \section{Cuantificadores}

            Usar los conectores lógicos nos permiten traducir un teorema matemático en ideas lógicas, pero
            añadiré unos nuevos símbolos que nos permitirán traducir aún más ideas.


            \subsubsection{Sentencias Abiertas y Cuantificadores}
                Los cuantificadores trabajan con sentencias abiertas (o también llamadas funciones lógicas),
                son cosas que son parecidas a las proposiciones, pero no lo son exactamente, son cosas como:

                $p(x)$: $x$ es un número par.

                Puesto que la validez de $p(x)$ depende que número sea $x$, $p(x)$ no es no totalmente cierta ni
                totalmente falsa, por lo tanto no es una proposición.

                Los cuantificadores permiten la construcción de proposiciones a partir de oraciones abiertas,
                bien sea particularizando o generalizando. Así, un cuantificador transforma una oración abierta,
                en una proposición a la cual se le asigna un valor de verdad.

                Es decir, los cuantificadores trabajan con sentencias abiertas, ya que al aplicarles un cuantificador
                se vuelven una proposiciones normales.

            \subsubsection{Cuantificadores Ocultos}
                Ahora llegamos al punto muy importante. En matemáticas, la expresión $p(x) \linfire q(x)$
                se entiende que en realidad hablamos de la oración $\forall x \in A, \MiniSpace p(x) \linfire q(x)$.

                Si, lo se, matemáticos que les da flojera ser formales, aunque entiendelos, es tal común esta
                clase de enunciados que se notaría tan repetetitivo.



            % =====================================
            % ====  CUANTIFICADOR UNIVERSAL   =====
            % =====================================
            \clearpage
            \subsection{Cuantificador Universal}

                Se utiliza para afirmar que \textbf{todos} los elementos de un conjunto $A$ cumplen con una 
                propiedad determinada $p(x)$.

                \begin{equation}
                    \forall x \in A,\MiniSpace p(x)
                \end{equation}

                Es normal en matemáticas básicas escuchar frases como $p(a)$ para una $a$ cualquiera, esto es
                simplemente otra forma de decir $\forall x ,\MiniSpace p(x)$.\\

                Otra forma de escribir el cuantificador universal $\forall x \in A,\MiniSpace p(x)$ es
                escribir $(x \in A) \linfire p(x)$



            % =====================================
            % ====  CUANTIFICADOR EXISTENCIAL  ====
            % =====================================
            \subsection{Cuantificador Existencial}

                Se utiliza para afirmar que \textbf{existe al menos un} elemento de un conjunto $A$ que 
                cumple con una propiedad determinada $p(x)$.

                \begin{equation}
                    \exists x \in A,\MiniSpace p(x)
                \end{equation}

                Es normal en matemáticas básicas escuchar frases como $p(a)$ para una $a$ específica, esto es
                simplemente otra forma de decir $\exists x, \MiniSpace p(x)$



        % =====================================================
        % =======   LEYES DE CUANTIFICADORES      =============
        % =====================================================
        \clearpage
        \section{Leyes de Cuantificadores}
                    
                Sean $p_{(x)}$ sentencias abierta lógica, $A$ un conjunto que opera sobre $p(x)$ donde $x$ son
                elementos de $A$ y sea $T$ una tautología y sea $F$ una contradicción.

                \begin{itemize}
                    \item
                    \textbf{Negación del Universal} \\
                    $\lnot (\forall x \in A,\MiniSpace p(x)) \lequal \exists x \in A,\MiniSpace \lnot p(x)$

                    \item
                    \textbf{Negación del Existencial} \\
                    $\lnot (\exists x \in A,\MiniSpace p(x)) \lequal \forall x \in A,\MiniSpace \lnot p(x)$

                    \item \textbf{Cambio de Variables} \\
                            $p(a) \lequal (p(x) \land (x=a))$
                            
                    \item \textbf{Cuantificadores sobre Proposiciones} \\
                            $\exists x, p \lequal \forall x, p \lequal p$

                    \item \textbf{Leyes Conmutativas para Cuantificador Existencial}

                        \begin{itemize}

                            \item
                            $\exists x, [p(x) \lor q(x)] \lequal \exists x, p(x) \lor \exists x, q(x)$

                            \item
                            $\exists x, [p(x) \land q(x)] \linfire \exists x, p(x) \land \exists x, q(x)$

                        \end{itemize}

                    \item \textbf{Leyes Conmutativas para Cuantificador Universal}

                        \begin{itemize}
                            
                            \item
                            $\forall x, [p(x) \land q(x)] \lequal \forall x, p(x) \land \forall x, q(x)$

                            \item
                            $\forall x, p(x) \lor \forall x, q(x) \linfire \forall x, [p(x) \lor q(x)]$

                        \end{itemize}
                        
                \end{itemize}










% ////////////////////////////////////////////////////////////////////////////////////////////////////////////////////
% ////////////////////////////////////////       CONJUNTOS           /////////////////////////////////////////////////
% ////////////////////////////////////////////////////////////////////////////////////////////////////////////////////
\part{Conjuntos}

    % ======================================================================================
    % =============================       PRINCIPIOS BASICOS      ==========================
    % ======================================================================================
    \chapter{Principios Básicos}
        \clearpage

        % =====================================================
        % ============           DEFINICION            ========
        % =====================================================
        \section{Definición}

            % =====================================
            % =========   ¿QUE SON?     ===========
            % =====================================
            \subsection*{¿Qué son?}
                Olvida todo lo que sabes sobre números. Olvídate de que sabes lo que es un número.
                Aquí es donde empiezan las matemáticas. En vez de matemáticas con números,
                vamos a hacer matemáticas con 'cosas'.

                Se denomina conjunto a la agrupación de entes o elementos, que poseen una o
                varias características en común. 

                Un conjunto puede ser una agrupación de números, de vectores, de autos, de espacios vectoriales,
                de objectos, de funciones e incluso un conjunto puede ser una agrupación de otros conjuntos.


                % =====================================
                % =====   IDEAS IMPORTANTES    ========
                % =====================================
                \subsection*{Ideas Importantes}

                Lo conjuntos generalmente son denotados por letras mayúsculas, como $A, B, C, \cdots$, mientras
                que las letras minusculas como $a, b, c, \cdots x, y, z$ se usan para representar elementos
                de un conjunto.


                Quiza los conjuntos más importantes que verás a lo largo de estos apuntes son:

                \begin{itemize}
                    \item $\Naturals$ : Representa al conjunto de todos los naturales, ya sabes
                            números como \\$1, 2, 3, \cdots$
                    \item $\Integers$ : Representa al conjunto de todos los enteros, ya sabes
                            números como \\$\cdots, -1, 0, 1, \cdots$
                    \item $\Racionals$ : Representa al conjunto de todos los racionales, ya sabes
                            números como \\$\frac{1}{1}, \frac{1}{2}, \frac{23}{72}, \cdots$
                    \item $\Reals$ : Representa al conjunto de todos los reales, ya sabes
                            números como \\$\frac{3}{2}, -\pi, 3, \cdots$
                    \item $\Complexs$ : Representa al conjunto de todos los complejos, ya sabes
                            números como \\$3 + 2i, \pi i, 3, \cdots$
                \end{itemize}

                

            % =====================================
            % =========  ¿COMO DEFINIRLO?  ========
            % =====================================
            \clearpage
            \section{¿Cómo Definirlo?}


                % =====================================
                % =========   PERTENCIA     ===========
                % =====================================
                \subsection{Pertencia}

                Creo que el símbolo más importante al hablar de conjuntos es este: $x \in A $. Esto quiere decir,
                el elemento x \textbf{pertenece} al Conjunto A.

                
                Y de la misma manera puedes usar el símbolo $\notin$ que significa \textbf{no pertenece}.


                % =====================================
                % ====   FORMAS DE DECLARAR   =========
                % =====================================
                \subsection{Notación Básica}

                    La notación más común para declarar o hablar de un conjunto es colocar los elementos de esta
                    manera : $ A = \{\ \text{Elementos de A} \}$.

                    Puedes entonces hacer algo como:

                    \begin{itemize}
                        \item Enumerar TODOS los elementos o entes que forman el conjunto :
                        \\$A = \{a, e, i, o, u\}$

                        \item Enumerar un patrón de los elementos o entes que forman el conjunto :
                        \\$A = \{+2, -4, +8, -16, \cdots \}$
                    \end{itemize}


                    Recuerda también:

                    \begin{itemize}
                        \item Los elementos repetidos no cuentan, si ya esta un elemento dentro del
                            conjunto, da lo mismo que lo vuelvas a enumerar.\\
                            $A = \{a, e, i, o, u\} = \{a, a, e, i, o, u\}$

                        \item No importa el orden en el me muestres los elementos,
                            solo importa que esten dentro.\\
                            $A = \{a, e, i, o, u\} = \{u, a, i, e, o\}$
                    \end{itemize}



                % =====================================
                % ====   FORMAS DE DECLARAR   =========
                % =====================================
                \clearpage
                \subsection{Notación Formal} 

                    Esta notación tiene un nombre genial en inglés, se le conoce como \textbf{Set Builder Notation},
                    esta notación es la que sueles encontrar en los libros. 

                    Se ve fea al principio pero te da toda la información que necesitas.

                    Veamos como formarla poco a poco:

                    Lo primero que hacemos es elegir una letra minúscula (de forma normal, no es ninguna regla)
                    que representará a cualquier elemento al azar del conjunto, por ejemplo usemos la $x$.

                    También solemos usar la línea vertical, de esta forma: $P_1 | P_2$ que se lee como: 
                    $P_1$ tal que $P_2$.  

                    \subsubsection*{Formas Básica}

                        Ahora si, veamos como se ve esta notación:
                        \begin{equation}   
                            A = \{ x \Such x \dots \}
                        \end{equation}

                        Esto es la base y esto es lo que nos quiere decir:

                        Definimos cierto conjunto, al que llamaremos $A$ como la agrupación de todas las $x$
                        (es decir cada x es un elemento, un ente) tal que cumplen ciertas características
                        (eso es lo que significa esos puntitos, ahí deberías poner las reglas que tenga
                        tu conjunto). \\

                        Por ejemplo podemos poner algo como:

                        $B = \{ x \Such x \in \Naturals, \MiniSpace x < 4 \}$\\

                        B es el conjunto de todas las $x$ tal que $x$ pertenece a los Naturales
                        y $x$ es menor que 4.


                    \subsubsection*{Formas Alternas}
                    
                        También es común ver la característica que pertenece a cierto conjunto mayor
                        antes de la línea vertical.

                        Entonces el ejemplo anterior se vería como:

                        $B = \{ x \in \Naturals \Such x < 4 \} = \{ x \Such x \in \Naturals, \MiniSpace x < 4 \} $


                    
                % =====================================
                % ====          EJEMPLOS      =========
                % =====================================
                \clearpage
                \subsection{Ejemplos} 

                % ==================
                % ===  EJEMPLO   ===
                % ==================
                \subsubsection*{Ejemplo 1:}

                    Veamos por ejemplo como definir el Conjunto $C_2$ como aquel que contenga a
                    TODOS las vocales:

                    \begin{equation*}   
                    \begin{split}   
                        C_2 &= \{ Vocales \}       \\
                        C_2 &= \{a, e, i, o, u \}
                    \end{split}   
                    \end{equation*}

                    Si te das cuenta, podemos definirlos de muchas maneras.


                % ==================
                % ===  EJEMPLO   ===
                % ==================
                \subsubsection*{Ejemplo 2:}

                    Veamos por ejemplo como definir el Conjunto $C$ \emph{(lo sé me muero con
                    mi creatividad para los nombres)} como aquel que contenga a TODOS los números
                    reales negativos:

                    \begin{equation*}   
                        C = \{ x \in \Reals |\quad x < 0 \}
                    \end{equation*}

                    \textbf{En Lenguaje normal:\\}
                    Esto lo podemos leer como $C$ es el conjunto \emph{(es decir todo lo que esta
                    entre parentesís)} de todas las $x$ que pertenezcan al los números reales
                    \emph{(eso quiere decir el $x \in \Reals$)} tal que (eso lo representamos
                    como: $|$) $x$ es menor que 0 \emph{(esa es nuestra condición para
                    encontrar si alguna $x$ pertenece a nuestro conjunto)}.



        % =====================================================
        % ============        CLASIFICACION            ========
        % =====================================================
        \clearpage
        \section{Clasificación}
                
            Podemos clasificar de muchas maneras a los conjuntos, veamos las más comunes:

            \textbf{\large \\Tamaño}

            \begin{itemize}
                \item \textbf{Finito}:
                    Si tiene una colección que se pueda contar, aunque sea difícil.

                    Por ejemplo, el conjunto de juguetes incluye todos los tipos de
                    juguetes que hay en el mundo. Aunque sea difícil, se podrían contar
                    todos los tipos de juguetes del mundo, por lo que es finito.

                \item \textbf{Infinito}:
                    Si tiene una colección que no se pueda terminar de contar nunca.

                    Por ejemplo, el conjunto de todos los números pares, que son
                    infinitos, es un conjunto infinito.
            \end{itemize}


        % =====================================================
        % ============        CONJUNTO VACIO        ===========
        % =====================================================
        \clearpage
        \section{Conjunto Vacío}
                
            Ok, ya sabemos que un conjunto es un grupo de elementos, pero ...
            ¿Cómo represento a un conjunto en el que no hay nada?

            Como una caja vacía.

            De hecho, me gusta, hablemos de el Conjunto vacío como un caja vacía.

            \subsubsection*{Definición}
            Llamemos $\phi$ como aquel conjunto tal que $\phi = \{ \}$ es decir el conjunto que no
            tiene elementos.

            Solemos usar este simbolo por su parecido con un cero, pero recuerda no es un cero, simplemente
            es una forma de denotar al conjunto vacío.

            \subsubsection*{Ideas Importantes}
            Listo, eso es casí todo, además te gustará que te recuerde las siguientes
            proposiciones:

            \begin{itemize}
                \item $|\phi| = 0$ : 
                        Esto quiere decir que la cardinalidad \emph{(es decir
                        la cantidad de elementos)} del conjunto vacío es la misma que 
                        la cantidad de galletas en una caja vacía de galletas, osea 0.

                \item $\phi \neq \{\phi\}$: 
                        Esto quiere decir que no es lo mismo hablar del conjunto
                        vacío que de hablar de un conjunto cualquiera que contiene al
                        conjunto vacío.

                        Es decir simplemente no es lo mismo tener una caja vacía que una caja
                        con una caja vacía dentro \emph{(si lo piensas la segunda caja ya 
                        no esta completamente vacía)}
            \end{itemize}


        % =====================================================
        % ============        CONJUNTO UNIVERSO      ==========
        % =====================================================
        \clearpage
        \section{Conjunto Universo}
            
            Como podemos imaginarnos, tenía que existir un término inverso, digamos que estamos
            analizando y agrupando animales por su habitad, entonces tenemos muchos conjuntos cool
            como animales del bosque o marinos, pero también tenemos a un mega conjunto que llamamos
            universo donde tenemos a todos los animales.

            Muchas veces a la hora de hablar sobre conjuntos solemos definirlos sobre un universo.

            Podemos ver muchaS analogías, veamos a ver cual te gusta más:

            \begin{itemize}
                \item Es como si el universo fuera el padre, entonces las hijas son nuestros conjuntos
                \item Es como si el universo una caja de cereal, entonces nuestros conjuntos son grupos de cereales
                    que estaban dentro de la caja.
            \end{itemize}


            \subsubsection*{Definición}
            Un conjunto universo es aquel conjunto que contiene a cualquier conjunto del que estemos hablando.


            \subsubsection*{Ideas Importantes}
            Creo que resulta bastante obvio pero aquí hay algunas cosas que quizá te interesen.
            \begin{itemize}
                \item $\phi^C = U$
                \item $U^C = \phi$
            \end{itemize}





    % ======================================================================================
    % ============================      ALGEBRA CON CONJUNTOS     ==========================
    % ======================================================================================
    \chapter{Álgebra y Operaciones}
        \clearpage


        % =====================================================
        % ============        RELACIONES               ========
        % =====================================================
        \clearpage
        \section{Relaciones}

            Es muy importante que ademas de hablar de los conjuntos hablemos de las relaciones
            que existen entre los dos conjuntos.


            % =================================
            % ======    EQUIVALENCIA    =======
            % =================================
            \subsection{Equivalencia}

                Creo que esta relación es más que obvia, tanto que ya la he usado sin darme cuenta
                porque considero que es de lo más obvio.

                Que dos conjuntos sea iguales quiere decir que ambos conjuntos contiene \textbf{exactamente}
                los mismos elementos.

                Ve entonces que ahora tiene mucho sentido las ideas que puse antes:

                \begin{itemize}
                    \item Los elementos repetidos no cuentan, si ya esta un elemento dentro del
                        conjunto, da lo mismo que lo vuelvas a enumerar.\\
                        $A = \{a, e, i, o, u\} = \{a, a, e, i, o, u\}$

                    \item No importa el orden en el me muestres los elementos,
                        solo importa que esten dentro.\\
                        $A = \{a, e, i, o, u\} = \{u, a, i, e, o\}$
                \end{itemize}


            % =================================
            % ======    SUBCONJUNTOS    =======
            % =================================
            \clearpage
            \subsection{Subconjuntos}

                Esta es la relación mas importante siento yo, porque será la que mas ocupemos a lo largo
                del tiempo.

                Que el A sea un subconjunto de B quiere decir que \textbf{todos} los elementos de A
                también son elementos de B.

                \subsubsection{Definición}

                    Una forma más formal de definirlo es que $x \in A \to x \in B$

                    Esta idea es muy inteligente, pues nos dice que un el hecho de que
                    un elemento pertenezca a A infiere o nos obliga a que ese mismo
                    elemento pertenezca a B.


                \subsubsection{Proposiciones}

                    Algunas proposiciones muy obvias son que:
                    \begin{itemize}
                        \item $A \subseteq A$
                        \item Si $A = B$, entonces $A \subseteq B$
                        \item Todos los conjuntos son subconjuntos de conjunto universo $U$
                        \item $\emptyset$ es un subconjunto de cualquier conjunto
                    \end{itemize}

                \subsubsection{SubConjuntos Propios}

                    Solemos usar la idea de un subconjunto propio $A \subset B$ si es que sabemos ya que:
                    $A \subseteq B$ y $A \neq B$





        % =====================================================
        % ============           OPERACIONES           ========
        % =====================================================
        \clearpage
        \section{Operaciones}

            Podemos hacer operaciones con los conjuntos de una manera muy similiar a como hacemos
            operaciones con los números normales, tu defines una operación, y la haces entre dos conjuntos
            y esta te dará un nuevo conjunto,  pero aquí siento que son incluso más divertidas.

            Te presentaré estas operaciones una a una y te mostraré con mas detalle cada una, así 
            que vamos, empecemos:

            % =================================
            % ======    INTERSECCION    =======
            % =================================
            \subsection{Intersección}

                Lo mejor de dos mundos, veamos como seleccionar a los elementos en común con operaciones
                de conjuntos.

                \begin{equation}
                    A \cap B = \{ x \Such x \in A \text{ y también } x \in B \}
                \end{equation}

                Esta operación basicamente nos da un conjunto en el que estan \textbf{solo} los elementos
                que bien pertenezcan a A y también que pertenezcan a B.


            % =================================
            % ======        UNION       =======
            % =================================
            \clearpage
            \subsection{Unión}

                La unión representa la fuerza, así que veamos que representa la unión cuando estemos
                hablando de conjuntos.

                \begin{equation}
                    A \cup B = \{ x \Such x \in A \text{ ó bien } x \in B \}
                \end{equation}

                Esta operación basicamente nos da un conjunto en el que estan \textbf{todos} los elementos
                que bien pertenezcan a A o bien que pertenezcan a B.


            % =================================
            % ======   COMPLEMENTO      =======
            % =================================
            \clearpage
            \subsection{Complemento}

                Todo lo que no seas tu, así que veamos que representa el complemento cuando estemos
                hablando de conjuntos.

                \begin{equation}
                    A^C = \overline A = \{ x \in U \Such x \notin A \}
                \end{equation}

                Esta operación basicamente nos da un conjunto en el que estan \textbf{todos} los elementos
                que no pertenezcen a A.

                También hay otra forma de definirlo como $A^C = U - A$

                \subsubsection{Conjunto Universo}
                    Si lo piensas detenidamente, aquí hay un problema y es que no te dije que es $U$, este
                    representa el conjunto universo, es decir, es aquel conjunto que del que todos los demás
                    son subconjuntos.

                    Es importantes especificar cual es tu conjunto universo.


            % =================================
            % ======      RESTA         =======
            % =================================
            \clearpage
            \subsection{Resta}

                No quiero nada que ver contigo, así que veamos que representa la resta cuando estemos
                hablando de conjuntos.

                \begin{equation}
                    A - B = A \char`\\ B = \{ x \Such x \in A \text{ y también }  x \notin B \}
                \end{equation}

                Esta operación basicamente nos da un conjunto en el que estan \textbf{todos} los elementos
                de A que no pertenezcen a B.

                A esta operación también se la conoce como complemento relativo.

                También hay otra forma de definirlo como $A - B = A \cap B^C$



            % =================================
            % ==    DIFERENCIA SIMETRICA   ====
            % =================================
            \clearpage
            \subsection{Diferencia Simétrica}

                El XOR de los conjuntos, así que veamos que representa la diferencia simétrica cuando estemos
                hablando de conjuntos.

                \begin{equation}
                    A \Delta B = \{ x \Such x \in A \cup B \text{ y también }  x \notin A \cap B \}
                \end{equation}

                Esta operación basicamente nos da un conjunto en el que estan \textbf{todos} los elementos
                que pertenezcen a $A$ y a $B$, pero no a ambos.

                Ve que de la definición se tiene que:
                \begin{equation*}
                \begin{split}
                    A \Delta B &= (A \cup B) - ( A \cap B) \\
                    A \Delta B &= (A - B) \cup ( B - A)
                \end{split}
                \end{equation*}

                De hecho de la útima proposición sale su nombre, es una diferencia simétrica.


            % =================================
            % ==    PRODUCTO CARTESIANO    ====
            % =================================
            \clearpage
            \subsection{Producto Cartesiano}

                Esta es la base de lo que se conoce como es plano cartesiano, y es quizá
                la operación mas útil que vas a conocer a lo largo de estos textos,
                veamos específicamente porque:

                \subsubsection*{N-Tuplas}

                    El resultado de un producto cartesiano es un conjunto formado de n-tuplas,
                    cada n-tuplas es una agrupación ordenada de elementos.
                    Por ejemplo $(a,b)$ ó $(x,y,z)$.

                    Al ser un ente ordenado $(a,b) \neq (b,a)$

                    La n en su nombre solo nos dice la cantidad de elementos que tiene cada tupla.


                \subsubsection*{Definición}

                    Esta operación basicamente nos da un conjunto en el que estan \textbf{todas} las n-tuplas
                    donde su primer elemento pertenece a $A$ y su segundo elemento pertenece a $B$.

                    \begin{equation}
                        A \times B = \{ (a, b) \Such a \in A \text{ y también } b \in B \}
                    \end{equation}

                \subsubsection*{Ideas Importantes}

                \begin{itemize}
                    \item Ve que de la definición se tiene que $ A \times B \neq B \times A$

                    \item También de manera común solemos simplificar $A \times A$ como $A^2$

                    \item Un elemento de $A \times B \times C$ es $((a,b),c) = (a,(b,c)) = (a,b,c)$
                \end{itemize}

                % ====================
                % ==    EJEMPLOS  ====
                % ====================
                \clearpage
                \subsection*{Ejemplo}

                    Podemos hacer uso de una tabla para encontrar todos los elementos del producto
                    cartesiano.

                    Veamos por ejemplo $A = \{a, b, c\}$ y $B = \{ x, y, z\}$

                    \begin{tabular}{r ||c |c | c  }
                       $A \times B$ & $x$ & $y$ & $z$ \\
                       \midrule
                      
                       $a$ &  $(a, x)$  &  $(a, y)$  &  $(a, z)$  \\
                       $b$ &  $(b, x)$  &  $(b, y)$  &  $(b, z)$  \\
                       $c$ &  $(c, x)$  &  $(c, y)$  &  $(c, z)$  \\
                     
                    \end{tabular}

                    Entonces : 
                    \begin{equation*}
                        A \times B = 
                        \{
                            (a, x) , (a, y) , (a, z) ,
                            (b, x) , (b, y) , (b, z) ,
                            (c, x) , (c, y) , (c, z) 
                        \}
                    \end{equation*}



            % =================================
            % ==    CONJUNTO POTENCIA      ====
            % =================================
            \clearpage
            \subsection{Conjunto Potencia}

                El conjunto que contiene a todos los subconjuntos posibles. 

                Esta operación es diferente en el sentido de que no toma sus elementos del conjunto
                que toma como entrada, sino que usa esos elementos para combinarlos y crear subconjuntos
                que son los elementos de esta nueva operación.

                Ok, ok, quizá me puse muy intenso con el párrafo de arriba, veamos un poco más calmado
                como es que funciona.

                \subsubsection*{Definición}

                    Esta operación basicamente nos da un conjunto en el que estan \textbf{todos} los
                    conjuntos que son subconjuntos de tu conjunto original.

                    \begin{equation}
                        P(A) = \{ A' \Such A' \subseteq A \}
                    \end{equation}

                \subsubsection*{Ideas Importantes}

                \begin{itemize}
                    \item Si tu conjunto tiene $n$ elementos, tu conjunto potencia tendrá $2^n$ elementos.

                    \item Si $n < |A|$ entonces existen $\pVector{n\\r} = \frac{n!}{r!(n-r)!}$ subconjuntos
                            de $A$ de cardinalidad $r$

                    \item $P(\emptyset) = \{\emptyset\}$

                    \item $P( \{\emptyset\}) = \{\emptyset, \{\emptyset\} \}$

                    \item $A \in P(A)$
                    \item Solo si $A = \{\emptyset\}$ entonces $A \subseteq P(A)$,
                    en cualquier otro caso $A \nsubseteq P(A)$
                \end{itemize}

                % ====================
                % ==    EJEMPLOS  ====
                % ====================
                \clearpage
                \subsection*{Ejemplo}

                    Podemos hacer uso de una tabla y el binario para encontrar todos los
                    elementos del conjunto potencia.

                    Si quieres crear un conjunto potencia, escribe la sucesión de números binarios de $n$ cifras,
                    y con cada número haz un subconjunto: Cuando haya un $'1'$, añade el elemento que corresponde.\\

                    Veamos por ejemplo $A = \{a, b, c\}$ \\


                    \begin{tabular}{c | c  }
                       $a b c$ & SubConjuntos       \\
                       \midrule
                      
                       $0 0 0$ &  $\{ \}$           \\
                       $0 0 1$ &  $\{ c \}$         \\
                       $0 1 0$ &  $\{ b \}$         \\
                       $0 1 1$ &  $\{ b, c \}$      \\
                       $1 0 0$ &  $\{ a \}$         \\
                       $1 0 1$ &  $\{ a, c \}$      \\
                       $1 1 0$ &  $\{ a, b \}$      \\
                       $1 1 1$ &  $\{ a, b, c \}$   \\
                     
                    \end{tabular}

                    Entonces : 
                    \begin{equation*}
                        P(A) = 
                        \{
                            \{ \}, \{ c \}, \{ b \}, \{ b, c \}, 
                            \{ a \}, \{ a, c \}, \{ a, b \}, \{ a, b, c \}
                        \}
                    \end{equation*}



        % =====================================================
        % ========         LEYES DE CONJUNTOS          ========
        % =====================================================
        \clearpage
        \section{Leyes de los Conjuntos}
                
            Sean $A, B, C$ conjuntos de un universo $U$, entonces tenemos las siguientes propiedades:

            \begin{itemize}
                \item \textbf{Doble Complemento} \\
                        $(A^C)^C = A$

                \item \textbf{Propiedad Conmutativa}
                    \begin{itemize}
                        \item $A \cap B = B \cap A$
                        \item $A \cup B = B \cup A$
                    \end{itemize}

                \item \textbf{Propiedad Asociativa}
                    \begin{itemize}
                        \item $A \cap (B \cap C) = (A \cap B) \cap C$
                        \item $A \cup (B \cup C) = (A \cup B) \cup C$
                    \end{itemize}

                \item \textbf{Propiedad Distributiva}
                    \begin{itemize}
                        \item $A \cap (B \cup C) = (A \cap B) \cup (A \cap C)$
                        \item $A \cup (B \cap C) = (A \cup B) \cap (A \cup C)$
                    \end{itemize}

                \item \textbf{Leyes de Morgan}
                    \begin{itemize}
                        \item $(A \cap B)^C = A^C \cup B^C$
                        \item $(A \cup B)^C = A^C \cap B^C$
                    \end{itemize}

                \clearpage

                \item \textbf{Propiedad de los Neutros}
                    \begin{itemize}
                        \item $A \cap U = A$
                        \item $A \cup \emptyset = A$
                    \end{itemize}

                \item \textbf{Propiedad de los Inversos}
                    \begin{itemize}
                        \item $A \cap A^C = \emptyset$
                        \item $A \cup A^C = U$
                    \end{itemize}

                \item \textbf{Propiedad de Dominación}
                    \begin{itemize}
                        \item $A \cap \emptyset = \emptyset$
                        \item $A \cup U = U$
                    \end{itemize}

                \item \textbf{Propiedad de Inepotencia}
                    \begin{itemize}
                        \item $A \cap A = A$
                        \item $A \cup A = A$
                    \end{itemize}

                \item \textbf{Propiedad de Absorción}
                    \begin{itemize}
                        \item $A \cap (A \cup B) = A$
                        \item $A \cup (A \cap B) = A$
                    \end{itemize}

            \end{itemize}


        % =====================================================
        % ============           CARDINALIDAD          ========
        % =====================================================
        \clearpage
        \section{Identidades y Propiedades}
                
            Ya conocemos su definición formal, incluso me dedique a poner algunas cosas super obvias en
            la pagina de definción de cada uno de ellos, pero es que hay tantas propiedades super interesantes
            que tenia que ponerles una sección propia para las mejores:


            % =====================================
            % ==========   SUBCONJUNTOS    ========
            % =====================================
            \subsection{Propiedades usando SubConjuntos}
                
                \subsubsection{Proposición}
                    Sea $A, B, C$ conjuntos cualesquiera tal que $A \subseteq B$. Entonces tenemos que:
                    \begin{equation*}
                    \begin{split}
                        (A \cap C) &\subseteq (B \cap C)        \\
                        (A \cup C) &\subseteq (B \cup C)
                    \end{split}
                    \end{equation*}

                \subsubsection{Proposición}
                    Sea $A, B, C$ conjuntos cualesquiera. Si $A \subseteq B$ y $A \subseteq C$.
                    Entonces tenemos que:
                    \begin{equation*}
                        A \subseteq (B \cap C)
                    \end{equation*}

                \subsubsection{Proposición}
                    Sea $A, B, C$ conjuntos cualesquiera. Si $A \subseteq C$ y $B \subseteq C$.
                    Entonces tenemos que:
                    \begin{equation*}
                        (A \cup B) \subseteq C
                    \end{equation*}


            % =====================================
            % =====   CONJUNTO POTENCIA    ========
            % =====================================
            \clearpage
            \subsection{Propiedades usando Conjunto Potencia}
                
                \subsubsection{Proposición}
                    $P(A \cap B) = P(A) \cap P(B)$

                \subsubsection{Proposición}
                    $P(A) \cup P(B) \subseteq P(A \cup B)$






        % =====================================================
        % ============           CARDINALIDAD          ========
        % =====================================================
        \clearpage
        \section{Cardinalidad y sus Propiedades}
                
            Ok, vamos avanzando, ahora es la hora de ver una característica de los conjuntos.
            La Cardinalidad, que no es mas que una forma \emph{fancy} de decir el número de 
            elementos ó entes que contiene cierto conjunto.

            Puedes verlo como una función que recibe un conjunto cualquiera y te regresa
            un número \emph{(Bueno, tecnicamente también sta el caso en el que la cardinalidad es infinita)}.

            Esta es la forma en que solemos expresar la cardinalidad de un conjunto cualquiera:
            \begin{equation}   
                |A| = \#A = Card(A)
            \end{equation}

            PD: Como te imaginas las propiedades no sirven de nada si es que alguno de tus conjuntos
            tiene una cardinalidad infinita, así que voy a suponer por obviedad que todos los siguientes
            conjuntos son finitos, que sino tendría que especificarlo a cada 4 palabras.



            % =====================================
            % =====        COMPLEMENTOS      ======
            % =====================================
            \clearpage
            \subsection{Propiedades usando Complemento}
                
                \subsubsection{Proposición}
                    Sea $A$ un conjunto cualquiera de un universo U, entonces tenemos que:
                    \begin{equation*}
                        |A^C| = |U| - |A|
                    \end{equation*}  

            % =====================================
            % =====        SUBCONJUNTOS      ======
            % =====================================
            \subsection{Propiedades usando Subconjuntos}
                
                \subsubsection{Proposición}
                    Sea $A, B$ conjuntos cualesquiera. Si  $A \subseteq B$, entonces tenemos que:
                    \begin{equation*}
                        |A| \leq |B|
                    \end{equation*}


            % =====================================
            % =====   UNION E INTERSECCION   ======
            % =====================================
            \clearpage
            \subsection{Propiedades usando Unión e Intersección}
                
                \subsubsection{Proposición}
                    Sea $A, B$ conjuntos cualesquiera, entonces tenemos que:
                    \begin{equation*}
                    \begin{split}
                        |A \cup B| &= |A| + |B| - |A \cap B|    \\  
                        |A \cap B| &= |A| + |B| - |A \cup B|  
                    \end{split}
                    \end{equation*}


            % =====================================
            % =====           RESTA          ======
            % =====================================
            \clearpage
            \subsection{Propiedades usando Resta}

                \subsubsection{Proposición}
                    Sea $A, B$ conjuntos cualesquiera, entonces tenemos que:
                    \begin{equation*}
                        |A - B| = |A| - |A \cap B|
                    \end{equation*}


            % =====================================
            % ====     DIFERENCIA SIMETRICA   =====
            % =====================================
            \subsection{Propiedades usando Diferencia Simetrica}

                \subsubsection{Proposición}
                    Sea $A, B$ conjuntos cualesquiera, entonces tenemos que:
                    \begin{equation*}
                        |A \Delta B| = |A| + |B| - 2|A \cap B|
                    \end{equation*}

            % =====================================
            % =====   PRODUCTO CARTESIANO     =====
            % =====================================
            \clearpage
            \subsection{Propiedades usando Producto Cartesiano}

                \subsubsection{Proposición}
                    Sea $A, B$ conjuntos cualesquiera, entonces tenemos que:
                    \begin{equation*}
                        |A \times B| = |A| \cdot |B|
                    \end{equation*}

            % =====================================
            % =====   CONJUNTO POTENCIA      ======
            % =====================================
            \subsection{Propiedades usando Conjunto Potencia}

                \subsubsection{Proposición}
                    Sea $A$ conjunto cualquiera, tal que $|A| = n$ entonces tenemos que:
                    \begin{equation*}
                        |P(A)| = 2^n
                    \end{equation*}










% ////////////////////////////////////////////////////////////////////////////////////////////////////////////////////
% /////////////////////////////////       RELACIONES  Y FUNCIONES        /////////////////////////////////////////////
% ////////////////////////////////////////////////////////////////////////////////////////////////////////////////////
\part{Relaciones y Funciones}



    % ======================================================================================
    % =============================       PRINCIPIOS BASICOS      ==========================
    % ======================================================================================
    \chapter{Relaciones}
        \clearpage

        % =====================================================
        % ============           DEFINICION            ========
        % =====================================================
        \section{Definición}

            Una relación $R$ entre dos conjuntos $A$ y $B$ es ante todo otro conjunto, una relación binaria es aquella
            que es en el fondo un conjunto de pares ordenados (x,y) donde $x$ es un elemento de $A$, y así mismo
            $y$ es un elemento de $B$.

            Este nuevo conjunto $R$ nos muestra como es que esta relacionados algunos (o todos) elementos de $A$ con
            otros elementos de $B$.



            % =====================================
            % =========   ¿QUE SON?     ===========
            % =====================================
            \subsection*{Definiciones Formales}

                Una Relación $R: A \to B$ es un subconjunto de $A \times B$.

                Solemos escribir la proposición $(x, y) \in R$ como $x R y$ para que se vea más bonito.

                Solemos escribir la proposición $(x, y) \notin R$ como $x\centernot{R}y$ para que se vea más bonito.





        % =====================================================
        % ==============   DOMINIO E IMAGEN    ================
        % =====================================================
        \clearpage
        \section{Dominio, Contradominio e Imagen}
                
            \subsection*{Dominio}

                El dominio $D_R$ de una relación $R : A \to B$ es simplemente el subconjunto de $A$ que
                contiene a todos los elementos que están relacionados hacia algun elemento de $B$.

                \begin{equation}
                    D_R = \{ a \in A \Such \exists b, \MiniSpace aRb \}
                \end{equation}

            \subsection*{Imagen}

                También le llama Rango, la Imagen $I_R$ de una relación $R : A \to B$
                es simplemente el subconjunto de $B$ que contiene a todos los elementos que están
                relacionados mediante $R$.

                \begin{equation}
                    I_R = \{ b \in B \Such \exists a, \MiniSpace aRb \}
                \end{equation}




        % =====================================================
        % ========         RELACION INVERSA        ============
        % =====================================================
        \clearpage
        \section{Relación Inversa}

            Una relación inversa es bastante fácil de definir:

            \begin{equation}
                R^{-1} = \{ (b, a) \in (B \times A) \Such (a, b) \in R \}
            \end{equation}





        % ====================================================
        % ========   REFLEXIVA, SIMETRICA Y TRANSITIVA   =====
        % ====================================================
        \clearpage
        \section{Reflexiva, Simétrica y Transitiva}

            Vamos a definir estas propiedades para una $R: A \to A$.
                
            % =============================
            % =====     REFLEXIVA     =====
            % =============================
            \subsection{Reflexiva}

                Una relación reflexiva es aquella en la que cualquier $a$ tiene que estar relacionada
                consigo misma.

                \begin{equation}
                    \forall a \in A, \MiniSpace aRa
                \end{equation}


            \subsection*{Cerradura Reflexiva}

                Si te das cuenta la relación mas sencilla que es simétrica es bastante simple:
                \begin{equation}
                    Id_A = \{ (a, a) \Such a \in A \}
                \end{equation}


                La cerradura reflexiva $Cl_R(R)$ de una relación $R$ es una relación que cumple con:

                \begin{itemize}
                    \item Es reflexiva: $Cl_R(R)$ es reflexiva. 
                    \item R esta contenida en ella: $R \subseteq Cl_R(R)$
                    \item $Cl_R(R)$ es la relación mas pequeña posible: 
                    Si es que $S$ es reflexiva y $R \subseteq S$ entonces $Cl_R(R) \subseteq S$
                \end{itemize}



                De hecho si te das cuenta es muy sencillo encontrarla, pues
                $Cl_R(R) = R \cup Id_A$.




            % =============================
            % =====     SIMETRICA     =====
            % =============================
            \clearpage
            \subsection{Simétrica}

                Una relación simétrica es aquella en la que cualquier si existe $aRb$ existe $bRa$.

                \begin{equation}
                    \forall a,b \in A,\MiniSpace (aRb) \to (bRa)
                \end{equation}


                La cerradura simétrica $Cl_S(R)$ de una relación $R$ es una relación que cumple con:

                \begin{itemize}
                    \item Es simétrica: $Cl_S(R)$ es simétrica. 
                    \item R esta contenida en ella: $R \subseteq Cl_S(R)$
                    \item $Cl_S(R)$ es la relación mas pequeña posible: 
                    Si es que $S$ es simétrica y $R \subseteq S$ entonces $Cl_S(R) \subseteq S$
                \end{itemize}



                De hecho si te das cuenta es muy sencillo encontrarla, pues
                $Cl_S(R) = R \cup R^{-1}$.



            % =============================
            % =====     TRANSITIVA    =====
            % =============================
            \clearpage
            \subsection{Transitiva}

                Una relación transitiva es aquella en la que cualquier a tiene que estar relacionada consigo
                misma.

                \begin{equation}
                    \forall a,b,c \in A, \MiniSpace ((aRb) \land (bRa)) \linfire (aRc)
                \end{equation}

                La cerradura transitiva $Cl_T(R)$ de una relación $R$ es una relación que cumple con:

                \begin{itemize}
                    \item Es transitiva: $Cl_T(R)$ es transitiva. 
                    \item R esta contenida en ella: $R \subseteq Cl_T(R)$
                    \item $Cl_T(R)$ es la relación mas pequeña posible: 
                    Si es que $S$ es transitiva y $R \subseteq S$ entonces $Cl_T(R) \subseteq S$
                \end{itemize}


            % =============================
            % =====     EQUIVALENCIA  =====
            % =============================
            \clearpage
            \subsection{Relación de Equivalencia}
                Decimos que $R$ es una relación de equivalencia si es que se cumplen las 3 propiedades antes vistas.


            \subsubsection{Clases de Equivalencia}

                Una relación de equivalencia $R$ sobre un conjunto $A$ produce una partición
                del conjunto en subconjuntos disjuntos, llamados \textbf{Clases de
                Equivalencia}, cada uno de ellos formados por elementos que están
                relacionados entre sí.

                Esta partición se representa por $A \backslash R$ y se llama
                \textbf{Conjunto Cociente}.






    % ======================================================================================
    % =============================          FUNCIONES            ==========================
    % ======================================================================================
    \chapter{Funciones}
        \clearpage    



        % =====================================================
        % ============           DEFINICION            ========
        % =====================================================
        \section{Definición}

            Las funciones son más que meras descripciones de relaciones numéricas. En un sentido más general,
            las funciones pueden comparar y relacionar diferentes tipos de estructuras matemáticas.

            Es probable que vea una función como un tipo de fórmula que describe una relación entre dos
            (o más) cantidades.

            Podemos definirlas de una manera informal como aquellas relaciones donde cada elemento de $A$ 
            esta involucrado solo una vez en la relación.

            % =====================================
            % =========   ¿QUE SON?     ===========
            % =====================================
            \subsubsection*{Definición Formal}

            Digamos que tenemos una relación $f$ entre dos conjuntos $A$ y $B$.
            Decimos que esta relación es una función si y solo si:

            \begin{itemize}
                \item Todo los elementos del dominio tienen un valor asignado:
                    \begin{equation}
                        \forall a \in A, \exists b \in B, \MiniSpace afb
                    \end{equation}

                \item Si $aRb_1$ y $aRb_2$ entonces $b_1 = b_2$

            \end{itemize}

            Pero no es la única forma de definirlo, otra forma que una función es una relación que 
            cumple con la propiedad de que para cada $a \in A$, la relación $f$ contiene exactamente
            un par ordenado de la forma $(a,b) \in f$.

            Ya que solo existe un par ordenado para cada $a \in A$, entonces solemos escribir 
            $aRb$ como $f(a) = b$.







        % =====================================================
        % ============   DOMINIO E IMAGEN    ==================
        % =====================================================
        \clearpage
        \section{Dominio e Imagen}
            
            \subsubsection*{Dominio}

                El dominio $D_f$ de una función $f : A \to B$ es simplemente el conjunto de $A$.

                Digo, esto se deduce de la definición, ya que es el conjunto de todas las posibles
                entradas.

            \subsubsection*{Imagen}

                El Rango ó Imagen de una función $f : A \to B$ es el conjunto de todas las posibles salidas
                de la función.

                \begin{equation}
                    \text{Rango}_f = \{ b \in B \Such (a,b) \in f \} = = \{ f(a) \Such a \in A \}
                \end{equation}






        % =====================================================
        % ========   INYECTIVAS Y SUPRAYECTIVAS   =============
        % =====================================================
        \clearpage
        \section{Inyectivas, Suprayectivas y Biyectivas}

            \subsection{Biyectiva}

                Son funciones muy especiales, son aquellas en las tanto son inyectivas como suprayectivas.
                Pero veamos mas a fondo que es eso de que una función sea alguna de estas cosas.


            \subsection{Funciones Inyectivas}

                \Quote{Una función es aquella en la que una línea horizontal NUNCA toca más de un
                punto de la función}.

                Una función es inyectiva si a cada valor del conjunto dominio le corresponde un valor
                distinto en el conjunto imágen, es decir en el que es imposible que la función mande
                dos valores al mismo valor en $B$.

                Es decir si es que es imposible que $f(a_1) = b$ y que también $f(a_2) = b$.

                \begin{equation}
                   \forall x, y \in A, \MiniSpace (x \neq y) \linfire (f(x) \neq f(y))
                \end{equation}


                \subsection*{Demostraciones}
                
                Para lograr demostrar que una $f$ es inyectiva tenemos que demostrar que para
                cualquiera combinación de elementos $x, y \in A$ se cumple que el hecho de que
                sean diferentes infiere a que $f(x) \neq f(y)$.

                \begin{itemize}
                    \item 
                        \textbf{Demostración Directa}:

                        Supón que $x, y \in A$ y que $x \neq y$
                        \\
                        $\cdots$
                        \\
                        Por lo tanto $f(x) \neq f(y)$

                    \item 
                        \textbf{ContraPositiva}:

                        Supón que $x, y \in A$ y que $f(x) = f(y)$
                        \\
                        $\cdots$
                        \\
                        Por lo tanto $x = y$

                \end{itemize}


            \clearpage
            \subsection{Suprayectivas}

                \Quote{Una función es aquella en la que una línea horizontal SIEMPRE toca mínimo          
                punto de la función}.

                Son funciones en las que cada elemento del rango tiene un element o en A.
                Una función es suprayectiva si está aplicada sobre todo $B$.

                \begin{equation}
                    \forall b \in B, \exists a \in A, \MiniSpace f(a) = b
                \end{equation}


            \clearpage
            \subsection{Principio del Palomar}

                Aquí hay una gran historia de porque se llama así este principio.

                Supón que $A$ y $B$ son conjuntos finitos y que existe una $f: A \to B$ que es una función.

                Entonces tenemos que:
                \begin{itemize}
                    \item Si $|A| > |B|$, entonces $f$ no es inyectiva.
                    \item Si $|A| < |B|$, entonces $f$ no es suprayectiva.
                \end{itemize}







        % =====================================================
        % ========         FUNCION INVERSA        =============
        % =====================================================
        \clearpage
        \section{Función Inversa}

            Aunque parecido, para que exista la relación inversa sea también una función, nuestra función original 
            tendrá que ser biyectia.

            Esta función inversa tiene una propiedad muy obvia:

            \begin{equation}
                \forall x \in A, \MiniSpace f^{-1}(f(x)) = x
            \end{equation}

            Y también otra obvia:

            \begin{equation}
                \forall y \in B, \MiniSpace f(f^{-1}(y)) = y
            \end{equation}




        % =====================================================
        % ========         IMAGEN Y PREIMAGEN       ===========
        % =====================================================
        \clearpage
        \section{Imagén y Preimagen}

            Sea una función $f : A \to B$ y además $X \subseteq A$ y $Y \subseteq B$.  

            \subsubsection{Imagén}
                La imágen es a fin de cuentas un conjunto, de manera fomral:
                \begin{equation} 
                    f(X) = \{ f(x) \Such x \in X \} 
                \end{equation} 

                Y es bastante obvio que $f(X) \subseteq B$.

            \subsubsection{PreImagen}
                La preimagen es a fin de cuentas un conjunto, y esta definido,
                incluso aunque la función no sea invertible, esta esta definida
                de manera fomral:
                \begin{equation} 
                    f^{-1}(Y) = \{ x \in A \Such f(x) \in Y \} 
                \end{equation} 

                Y es bastante obvio que $f^{-1}(Y) \subseteq A$.





        % =====================================================
        % ========         CARDINALIDAD Y FUNCIONES      ======
        % =====================================================
        \clearpage
        \section{Cardinalidad y Funciones}

            Podemos definir la cardinalidad de los conjuntos de una mejor manera
            usando la cardinalidad:

            Los conjuntos tienen la misma cardinalidad $|A| = |B|$ si y solo si es que existe
            una función biyectiva $f : A \to B$.


            Gracias a esto podemos decir que:

            \begin{itemize} 
                \item $|\Naturals| = |\Integers| = |\Racionals| = \aleph_0$

                \item $P(|\Naturals|) = |\Reals| = \aleph_1$

                \item $P(|\Reals|) = \aleph_2$

            \end{itemize} 



\end{document}