% ****************************************************************************************
% ************************     	RELACIONES Y FUNCIONES   	  ****************************
% ****************************************************************************************


% =======================================================
% =======         HEADER FOR DOCUMENT        ============
% =======================================================
    % *********   DOCUMENT ITSELF   **************
    \documentclass[12pt]{report}                                    %Type of docuemtn and size of font
    \usepackage[margin=1.2in]{geometry}                             %Margins and Geometry pacakge
    \usepackage{ifthen}                                             %Allow simple programming
    \usepackage{hyperref}                                           %Create MetaData for a PDF and LINKS!
    \setlength{\parindent}{0pt}                                     %Eliminate ugly indentation
    \author{Oscar Andrés Rosas}                                     %Who I am

    % *********   LANGUAJE AND UFT-8   *********
    \usepackage[spanish]{babel}                                     %Please use spanish
    \usepackage[utf8]{inputenc}                                     %Please use spanish - UFT
    \usepackage[T1]{fontenc}                                        %Please use spanish
    \usepackage{textcmds}                                           %Allow us to use quoutes
    \usepackage{changepage}                                         %Allow us to use identate paragraphs

    % *********   MATH AND HIS STYLE  *********
    \usepackage{amsthm, amssymb, amsfonts, mathrsfs}                %Make math beautiful
    \usepackage[fleqn]{amsmath}                                     %Please make equations left
    \usepackage{centernot}                                          %Allow me to negate a symbol
    \decimalpoint                                                   %Use decimal point

    % *********   GRAPHICS AND IMAGES *********
    \usepackage{graphicx}                                           %Allow to create graphics
    \usepackage{wrapfig}                                            %Allow to create images
    \graphicspath{ {Graphics/} }                                    %Where are the images :D

    % *********   LISTS AND TABLES ***********
    \usepackage{listings}                                           %We will be using code here
    \usepackage[inline]{enumitem}                                   %We will need to enumarate
    \usepackage{tasks}                                              %Horizontal lists
    \usepackage{longtable}                                          %Lets make tables awesome
    \usepackage{booktabs}                                           %Lets make tables awesome
    \usepackage{tabularx}                                           %Lets make tables awesome
    \usepackage{multirow}                                           %Lets make tables awesome
    \usepackage{multicol}                                           %Create multicolumns

    % *********   HEADERS AND FOOTERS ********
    \usepackage{fancyhdr}                                           %Lets make awesome headers/footers
    \pagestyle{fancy}                                               %Lets make awesome headers/footers
    \setlength{\headheight}{16pt}                                   %Top line
    \setlength{\parskip}{0.5em}                                     %Top line
    \renewcommand{\footrulewidth}{0.5pt}                            %Bottom line

    \lhead{                                                         %Left Header
        \hyperlink{chapter.\arabic{chapter}}                        %Make a link to the current chapter
        {\normalsize{\textsc{\nouppercase{\leftmark}}}}             %And fot it put the name
    }

    \rhead{                                                         %Right Header
        \hyperlink{section.\arabic{chapter}.\arabic{section}}       %Make a link to the current chapter
            {\footnotesize{\textsc{\nouppercase{\rightmark}}}}      %And fot it put the name
    }

    \rfoot{\textsc{\small{\hyperref[sec:Index]{Ve al Índice}}}}    %This will always be a footer  

    \fancyfoot[L]{                                                  %Algoritm for a changing footer
        \ifthenelse{\isodd{\value{page}}}                           %IF ODD PAGE:
            {\href{https://compilandoconocimiento.com/yo/}          %DO THIS:
                {\footnotesize                                      %Send the page
                    {\textsc{Oscar Andrés Rosas}}}}                 %Send the page
            {\href{https://compilandoconocimiento.com}              %ELSE DO THIS: 
                {\footnotesize                                      %Send the author
                    {\textsc{Compilando Conocimiento}}}}            %Send the author
    }
    
    
    
% ========================================
% ===========   COMMANDS    ==============
% ========================================

    % =====  GENERAL TEXT  ==========
    \newcommand \Quote {\qq}                                        %Use: \Quote to use quotes
    \newenvironment{Indentation}[1][0.75em]                         %Use: \begin{Inde...}[Num]...\end{Inde...}
    {\begin{adjustwidth}{#1}{}}                                     %If you dont put nothing i will use 0.75 em
    {\end{adjustwidth}}                                             %This indentate a paragraph
    \newenvironment{SmallIndentation}[1][0.75em]                    %Use: The same that we upper one, just 
    {\begin{adjustwidth}{#1}{}\begin{footnotesize}}                 %footnotesize size of letter by default
    {\end{footnotesize}\end{adjustwidth}}                           %that's it
        
    % =====  GENERAL MATH  ==========
    \DeclareMathOperator \Space {\quad}                             %Use: \Space for a cool mega space
    \DeclareMathOperator \MiniSpace {\;}                            %Use: \Space for a cool mini space
    \newcommand \Such {\MiniSpace|\MiniSpace}                       %Use: \Such like in sets

    % =====  LOGIC  ==================
    \DeclareMathOperator \doublearrow {\leftrightarrow}             %Use: \doublearrow for a double arrow
    \newcommand \lequal {\MiniSpace \Leftrightarrow \MiniSpace}     %Use: \lequal for a double arrow
    \newcommand \linfire {\MiniSpace \Rightarrow \MiniSpace}        %Use: \lequal for a double arrow

    % =====  NUMBER THEORY  ==========
    \DeclareMathOperator \Naturals {\mathbb{N}}                     %Use: \Naturals por Notation
    \DeclareMathOperator \Integers {\mathbb{Z}}                     %Use: \Integers por Notation
    \DeclareMathOperator \Racionals{\mathbb{Q}}                     %Use: \Racionals por Notation
    \DeclareMathOperator \Reals {\mathbb{R}}                        %Use: \Reals por Notation
    \DeclareMathOperator \Complexs {\mathbb{C}}                     %Use: \Complex por Notation

    % === LINEAL ALGEBRA & VECTORS ===
    \DeclareMathOperator \LinealTransformation {\mathcal{T}}        %Use: \LinealTransformation for a cool T

    \newcommand{\pVector}[1]{                                       %Use: \pVector {Matrix Notation} use parentesis
        \ensuremath{\begin{pmatrix}#1\end{pmatrix}}                 %Example: \pVector{a\\b\\c} or \pVector{a&b&c} 
    }
    \newcommand{\lVector}[1]{                                       %Use: \lVector {Matrix Notation} use a abs 
        \ensuremath{\begin{vmatrix}#1\end{vmatrix}}                 %Example: \lVector{a\\b\\c} or \lVector{a&b&c} 
    }
    \newcommand{\Vector}[1]{                                        %Use: \Vector {Matrix Notation} no parentesis
        \ensuremath{\begin{matrix}#1\end{matrix}}                   %Example: \Vector{a\\b\\c} or \Vector{a&b&c}
    }



% =====================================================
% ============     	  COVER PAGE	   ================
% =====================================================
\begin{document}
\begin{titlepage}

	\center
	% ============ UNIVERSITY NAME AND DATA =========
	\textbf{\textsc{\Large Proyecto Compilando Conocimiento}}\\[1.0cm] 
	\textsc{\Large Matemáticas Discretas}\\[1.0cm] 

	% ============ NAME OF THE DOCUMENT  ============
	\rule{\linewidth}{0.5mm} \\[1.0cm]
		{ \huge \bfseries Teoría de Números}\\[1.0cm] 
	\rule{\linewidth}{0.5mm} \\[2.0cm]
	
	% ====== SEMI TITLE ==========
	{\LARGE Una Pequeña Introducción}\\[7cm] 
	
	% ============  MY INFORMATION  =================
	\begin{center} \large
	\textbf{\textsc{Autor:}}\\
	Rosas Hernandez Oscar Andres
	\end{center}

	\vfill

\end{titlepage}

% =====================================================
% ========                INDICE              =========
% =====================================================
\tableofcontents{}
\label{sec:Index}

\clearpage










% ======================================================================================
% ===========================    ENTEROS Y NATURALES          ==========================
% ======================================================================================
\chapter{Enteros y Naturales}
    \clearpage

    % =====================================================
    % ============        DIVISIBILIDAD            ========
    % =====================================================
    \section{Divisibilidad}

        % ==================================
        % =========   FORMAL     ===========
        % ==================================
        \subsection*{Definición Formal}

        Dados dos números cualquiera $a, b \in \Integers$. Decimos que la proposición
        \textbf{\Quote{b} divide a \Quote{a}} $b|a$ es verdad si y solo si 
        $\exists q \in \Integers, \MiniSpace a = bq$.

        % ==================================
        % =========   ALTERNA     ==========
        % ==================================
        \subsection*{Definición Alterna}
        Veamos que lo que de verdad nos estan preguntando si es que $\frac{a}{b} \in \Integers$.

        Ya que de ser así eso quiere decir que podemos escribir a $a$ como $a = bq$.
        Y con esto logramos ver que $\frac{bq}{b} = q$ y habiamos dicho que $q \in \Integers$.

        Por lo tanto podemos resumir esto en que:
        \Quote{a divide a b si y solo si es que $\frac{a}{b}$ continua estando en los enteros}
        \begin{equation*}
            b | a \lequal \frac{a}{b} \in \Integers
        \end{equation*}\\[1.0em]



        % ==================================
        % ======   IMPORTANTES     =========
        % ==================================
        \subsection*{Ideas Imporantes}

            \begin{itemize}
                \item Si $b|a$ y $b \neq 0$ entonces $q$ es único.

                \item Si $b|a$ y $a \neq 0$ entonces $|b| \leq |a|$.

                    % ======== DEMOSTRACION ========
                    \begin{SmallIndentation}[1em]
                        \textbf{Demostración}:

                        Supongamos entonces que $b$ divide a $a$ y que $a \neq 0$, por lo tanto
                        la frase $a = bq$ nos da mucha información, pues obliga a que $b$ y $q$
                        no sean ninguno $0$, entonces tenemos que $a = bq$ donde $b \neq 0$ y 
                        $q \neq 0$.

                        Luego ya que no son 0, tenemos que $|q| \geq 1$ y $|b| \geq 1$, ya que sabemos
                        como funcionan los números enteros tenemos que sin importar cuanto valgan $q$ y $b$
                        se cumple que $|b||q| \geq |b|$ esto es lo mismo que $|bq| \geq |b|$ y sabemos que
                        $a = bq$, por lo tanto tenemos que $|a| \geq |b|$.

                        Esto es lo mismo que $|b| \leq |a|$

                    \end{SmallIndentation}

            \end{itemize}





        % ==================================
        % ======   PROPIEDADES     =========
        % ==================================
        \clearpage
        \subsection*{Propiedades de Divisilibidad}

            \begin{itemize}

                \item $b|b$

                    % ======== DEMOSTRACION ========
                    \begin{SmallIndentation}[1em]
                        \textbf{Demostración}:

                        Basta con ver que si $a = b$ entonces $b = bq$, por lo tanto $q = 1$.
                        Y listo, $1 \in \Integers$.

                    \end{SmallIndentation}

                \item $b|0$

                    % ======== DEMOSTRACION ========
                    \begin{SmallIndentation}[1em]
                        \textbf{Demostración}:

                        Basta con ver que si $a = 0$ entonces $0 = bq$, por lo tanto $q = 0$.
                        Y listo, $0 \in \Integers$.

                    \end{SmallIndentation}

                \item $1|a$ y también $-1|a$

                    % ======== DEMOSTRACION ========
                    \begin{SmallIndentation}[1em]
                        \textbf{Demostración}:

                        Basta con ver que si $b = \pm 1$ entonces $a = \pm q$, por lo tanto $q = \pm a$.
                        Y listo, $\pm a \in \Integers$.

                    \end{SmallIndentation}
            
                \item $0|a$ si y solo $a=0$

                    % ======== DEMOSTRACION ========
                    \begin{SmallIndentation}[1em]
                        \textbf{Demostración}:

                        Basta con ver que tenemos $a = 0q$, esto es lo mismo que $a = 0$.

                    \end{SmallIndentation}

                \item $b|1$ si y solo si $b=1$ ó $b=-1$

                    % ======== DEMOSTRACION ========
                    \begin{SmallIndentation}[1em]
                        \textbf{Demostración}:

                        Sabemos que $a=1=bq$, esto nos obliga a que $b=\frac{1}{q}$, ahora tenemos que
                        recordar que $b, q \in \Integers$, por lo tanto $q=1$ o bien $q=-1$ que es lo
                        mismo que decir que $b=1$ ó $b=-1$. 

                    \end{SmallIndentation}


                \item $b|a$ y $a|b$ si y solo si $a=\pm b$

                    % ======== DEMOSTRACION ========
                    \begin{SmallIndentation}[1em]
                        \textbf{Demostración}:

                        Sabemos que $a=bq_1$, y $b=aq_2$ por lo tanto podemos sustituir, 
                        $a=(aq_2)q_1$ por lo tanto $1=(q_1)(q_2)$, que es lo mismo que
                        $\frac{1}{q_2}=q_1$ ahora que para $q_1$ siga en los $\Integers$,
                        $q_2 = \pm 1$ por lo tanto $q_1 = \pm \frac{1}{1} = \pm 1$ por
                        lo tanto tenemos que $a = bq_1$ que es lo mismo que decir que $a = \pm b$.

                    \end{SmallIndentation}


                \item Si $b|a$ y $a|c$ entonces $b|c$

                    % ======== DEMOSTRACION ========
                    \begin{SmallIndentation}[1em]
                        \textbf{Demostración}:

                        Sabemos que $a=bq_1$, y $c=aq_2$ por lo tanto podemos sustituir, 
                        $c=(bq_1)q_2$ que es lo mismo que $c=bq_3$, donde $q_3 = q_1q_2$
                        donde $q_3 \in \Integers$. 
                        Y ya que $c=bq_3$ podemos decir que $b|c$.

                    \end{SmallIndentation}

                \item Si $b|a$ y $b|c$ entonces $b|a+c$ y $b|a-c$

                    % ======== DEMOSTRACION ========
                    \begin{SmallIndentation}[1em]
                        \textbf{Demostración}:

                        Sabemos que $a=bq_1$, y $c=bq_2$ por lo tanto podemos decir que
                        sumar o restar ambas ecuaciones, lo que nos daría
                        $a \pm c = bq_1 \pm bq_2$ que es lo mismo que
                        $a \pm c = b (q_1 \pm q_2)$ por lo que podemos decir que $b|a \pm c$.

                    \end{SmallIndentation}

                \item Si $b|a$ entonces $b|ak$ $\forall k \in \Integers$.

                    % ======== DEMOSTRACION ========
                    \begin{SmallIndentation}[1em]
                        \textbf{Demostración}:

                        Sabemos que $a=bq$ por lo mismo podemos decir que
                        $ak=b(qk)$ por lo tanto $b|ak$.

                    \end{SmallIndentation}

                \item $b|a$ si y solo si $b|-a$ si y solo si $-b|a$ si y solo si $-b|-a$

                    % ======== DEMOSTRACION ========
                    \begin{SmallIndentation}[1em]
                        \textbf{Demostración}:

                        Sabemos que existe $q_1$ tal que $a=bq_1$ para nuestro primer ssi
                        basta con decir que $-a=b(-q_1) = bq_2$ y listo, encontre a $q_2$
                        con lo que puedo afirmar que $b|-a$.

                        Para el segundo basta con ver que $a = -bq_3$ donde $q_3 = q_2$, con
                        lo que puedo afirmar que $-b|a$.

                        Para el último ssi basta con con ver que $-a=-bq_4$ donde $q_4=q_1$
                        así que puedo afirmar que $-b|-a$.

                    \end{SmallIndentation}

            \end{itemize}

                











% ======================================================================================
% =============================         COMBINATORIA          ==========================
% ======================================================================================
\chapter{Combinatoria}
    \clearpage

    % =====================================================
    % ============           DEFINICION            ========
    % =====================================================
    \section{Definición}

        Una relación $R$ entre dos conjuntos $A$ y $B$ es ante todo otro conjunto, una relación binaria es aquella
        que es en el fondo un conjunto de pares ordenados (x,y) donde $x$ es un elemento de $A$, y así mismo
        $y$ es un elemento de $B$.

        Este nuevo conjunto $R$ nos muestra como es que esta relacionados algunos (o todos) elementos de $A$ con
        otros elementos de $B$.



        % =====================================
        % =========   ¿QUE SON?     ===========
        % =====================================
        \subsection*{Definiciones Formales}

            Una Relación $R: A \to B$ es un subconjunto de $A \times B$.

            Solemos escribir la proposición $(x, y) \in R$ como $x R y$ para que se vea más bonito.

            Solemos escribir la proposición $(x, y) \notin R$ como $x\centernot{R}y$ para que se vea más bonito.








\end{document}
