% ****************************************************************************************
% ************************     	RELACIONES Y FUNCIONES   	  ****************************
% ****************************************************************************************


% =======================================================
% =======         HEADER FOR DOCUMENT        ============
% =======================================================
    % *********   DOCUMENT ITSELF   **************
    \documentclass[12pt]{report}                                    %Type of docuemtn and size of font
    \usepackage[margin=1.2in]{geometry}                             %Margins and Geometry pacakge
    \usepackage{ifthen}                                             %Allow simple programming
    \usepackage{hyperref}                                           %Create MetaData for a PDF and LINKS!
    \setlength{\parindent}{0pt}                                     %Eliminate ugly indentation
    \author{Oscar Andrés Rosas}                                     %Who I am

    % *********   LANGUAJE AND UFT-8   *********
    \usepackage[spanish]{babel}                                     %Please use spanish
    \usepackage[utf8]{inputenc}                                     %Please use spanish - UFT
    \usepackage[T1]{fontenc}                                        %Please use spanish
    \usepackage{textcmds}                                           %Allow us to use quoutes
    \usepackage{changepage}                                         %Allow us to use identate paragraphs

    % *********   MATH AND HIS STYLE  *********
    \usepackage{amsthm, amssymb, amsfonts, mathrsfs}                %Make math beautiful
    \usepackage[fleqn]{amsmath}                                     %Please make equations left
    \usepackage{centernot}                                          %Allow me to negate a symbol
    \decimalpoint                                                   %Use decimal point

    % *********   GRAPHICS AND IMAGES *********
    \usepackage{graphicx}                                           %Allow to create graphics
    \usepackage{wrapfig}                                            %Allow to create images
    \graphicspath{ {Graphics/} }                                    %Where are the images :D

    % *********   LISTS AND TABLES ***********
    \usepackage{listings}                                           %We will be using code here
    \usepackage[inline]{enumitem}                                   %We will need to enumarate
    \usepackage{tasks}                                              %Horizontal lists
    \usepackage{longtable}                                          %Lets make tables awesome
    \usepackage{booktabs}                                           %Lets make tables awesome
    \usepackage{tabularx}                                           %Lets make tables awesome
    \usepackage{multirow}                                           %Lets make tables awesome
    \usepackage{multicol}                                           %Create multicolumns

    % *********   HEADERS AND FOOTERS ********
    \usepackage{fancyhdr}                                           %Lets make awesome headers/footers
    \pagestyle{fancy}                                               %Lets make awesome headers/footers
    \setlength{\headheight}{16pt}                                   %Top line
    \setlength{\parskip}{0.5em}                                     %Top line
    \renewcommand{\footrulewidth}{0.5pt}                            %Bottom line

    \lhead{                                                         %Left Header
        \hyperlink{chapter.\arabic{chapter}}                        %Make a link to the current chapter
        {\normalsize{\textsc{\nouppercase{\leftmark}}}}             %And fot it put the name
    }

    \rhead{                                                         %Right Header
        \hyperlink{section.\arabic{chapter}.\arabic{section}}       %Make a link to the current chapter
            {\footnotesize{\textsc{\nouppercase{\rightmark}}}}      %And fot it put the name
    }

    \rfoot{\textsc{\small{\hyperref[sec:Index]{Ve al Índice}}}}    %This will always be a footer  

    \fancyfoot[L]{                                                  %Algoritm for a changing footer
        \ifthenelse{\isodd{\value{page}}}                           %IF ODD PAGE:
            {\href{https://compilandoconocimiento.com/yo/}          %DO THIS:
                {\footnotesize                                      %Send the page
                    {\textsc{Oscar Andrés Rosas}}}}                 %Send the page
            {\href{https://compilandoconocimiento.com}              %ELSE DO THIS: 
                {\footnotesize                                      %Send the author
                    {\textsc{Compilando Conocimiento}}}}            %Send the author
    }
    
    
    
% ========================================
% ===========   COMMANDS    ==============
% ========================================

    % =====  GENERAL TEXT  ==========
    \newcommand \Quote {\qq}                                        %Use: \Quote to use quotes
    \newenvironment{Indentation}[1][0.75em]                         %Use: \begin{Inde...}[Num]...\end{Inde...}
    {\begin{adjustwidth}{#1}{}}                                     %If you dont put nothing i will use 0.75 em
    {\end{adjustwidth}}                                             %This indentate a paragraph
    \newenvironment{SmallIndentation}[1][0.75em]                    %Use: The same that we upper one, just 
    {\begin{adjustwidth}{#1}{}\begin{footnotesize}}                 %footnotesize size of letter by default
    {\end{footnotesize}\end{adjustwidth}}                           %that's it
        
    % =====  GENERAL MATH  ==========
    \DeclareMathOperator \Space {\quad}                             %Use: \Space for a cool mega space
    \DeclareMathOperator \MiniSpace {\;}                            %Use: \Space for a cool mini space
    \newcommand \Such {\MiniSpace|\MiniSpace}                       %Use: \Such like in sets

    % =====  LOGIC  ==================
    \DeclareMathOperator \doublearrow {\leftrightarrow}             %Use: \doublearrow for a double arrow
    \newcommand \lequal {\MiniSpace \Leftrightarrow \MiniSpace}     %Use: \lequal for a double arrow
    \newcommand \linfire {\MiniSpace \Rightarrow \MiniSpace}        %Use: \lequal for a double arrow

    % =====  NUMBER THEORY  ==========
    \DeclareMathOperator \Naturals {\mathbb{N}}                     %Use: \Naturals por Notation
    \DeclareMathOperator \Integers {\mathbb{Z}}                     %Use: \Integers por Notation
    \DeclareMathOperator \Racionals{\mathbb{Q}}                     %Use: \Racionals por Notation
    \DeclareMathOperator \Reals {\mathbb{R}}                        %Use: \Reals por Notation
    \DeclareMathOperator \Complexs {\mathbb{C}}                     %Use: \Complex por Notation

    % === LINEAL ALGEBRA & VECTORS ===
    \DeclareMathOperator \LinealTransformation {\mathcal{T}}        %Use: \LinealTransformation for a cool T

    \newcommand{\pVector}[1]{                                       %Use: \pVector {Matrix Notation} use parentesis
        \ensuremath{\begin{pmatrix}#1\end{pmatrix}}                 %Example: \pVector{a\\b\\c} or \pVector{a&b&c} 
    }
    \newcommand{\lVector}[1]{                                       %Use: \lVector {Matrix Notation} use a abs 
        \ensuremath{\begin{vmatrix}#1\end{vmatrix}}                 %Example: \lVector{a\\b\\c} or \lVector{a&b&c} 
    }
    \newcommand{\Vector}[1]{                                        %Use: \Vector {Matrix Notation} no parentesis
        \ensuremath{\begin{matrix}#1\end{matrix}}                   %Example: \Vector{a\\b\\c} or \Vector{a&b&c}
    }



% =====================================================
% ============     	  COVER PAGE	   ================
% =====================================================
\begin{document}
\begin{titlepage}

	\center
	% ============ UNIVERSITY NAME AND DATA =========
	\textbf{\textsc{\Large Proyecto Compilando Conocimiento}}\\[1.0cm] 
	\textsc{\Large Matemáticas Discretas}\\[1.0cm] 

	% ============ NAME OF THE DOCUMENT  ============
	\rule{\linewidth}{0.5mm} \\[1.0cm]
		{ \huge \bfseries Teoría de Números}\\[1.0cm] 
	\rule{\linewidth}{0.5mm} \\[2.0cm]
	
	% ====== SEMI TITLE ==========
	{\LARGE Una Pequeña Introducción}\\[7cm] 
	
	% ============  MY INFORMATION  =================
	\begin{center} \large
	\textbf{\textsc{Autor:}}\\
	Rosas Hernandez Oscar Andres
	\end{center}

	\vfill

\end{titlepage}

% =====================================================
% ========                INDICE              =========
% =====================================================
\tableofcontents{}
\label{sec:Index}

\clearpage










% ======================================================================================
% ===========================    ENTEROS Y NATURALES          ==========================
% ======================================================================================
\chapter{Enteros y Naturales}
    \clearpage

    % =====================================================
    % ==========     PRINCIPIO BUEN ORDEN        ==========
    % =====================================================
    \section{Principio de Buen Orden}

        % ==================================
        % =========   FORMAL     ===========
        % ==================================
        \subsection*{Definición Formal}





% ======================================================================================
% ===========================         DIVISIBILIDAD           ==========================
% ======================================================================================
\chapter{Divisibilidad}
    \clearpage

     

    % =====================================================
    % ==========     ALGORITMO DE LA DIVISIÓN     =========
    % =====================================================
    \clearpage
    \section{Algoritmo de División}

        % ==================================
        % =========   FORMAL     ===========
        % ==================================
        \subsection*{Definición Formal}

            Dados dos enteros $a, b$ donde $b \neq 0$, existen otros dos 
            enteros únicos $q, r$, donde $0 \leq r < |b|$ tal que se cumple:

            \begin{equation}
                a = bq+r
            \end{equation}

            Vemos que basicamente nos dice cuántas veces cabe $b$ en $a$ sin pasarse (esto 
            es $q$) y cuantos le faltan para alcanzar a $a$ (esto es $r$).

            % ======== DEMOSTRACION ========
            \begin{SmallIndentation}[1em]
                \textbf{Demostración}:

                El primer paso es crear el conjunto
                $Residuos = \{a-|b|q \Such q \in \Integers, (a-|b|q) \geq 0 \}$.

                Ahora lo primero que tenemos que ver que es $|Residuos| \neq 0$. Para hacerlo
                veamos por casos, si $a < |b|$, entonces intenta a $q=-1$ y vemos que $a+|b|$
                siempre sera mayor o igual que $0$.
                Si $a > |b|$, entonces intenta a $q=1$ y vemos que $a-|b|$ siempre sera mayor
                o igual que $0$.
                Finalmente si $a = |b|$ cualquiera de los 2 ejemplos anteriores te sirven.
                Por lo tanto mínimo $Residuos$ tiene mínimo un elemento.

                Esto es un conjunto que basicamente contiene a los residuos, o visto de otra
                manera a los números que salen como resultado de sumarle multiplos de $|b|$ a $a$
                y que son mayores que 0.

                Ahora gracias al principio de buen orden (y que $Residuos$ es el conjunto
                de los Naturales más el cero) podemos llamar a $r$ al elemento mas pequeño
                de este conjunto.

                Ahora, gracias a la definición del conjunto $Residuos$ podemos decir que
                $r=a-|b|q_1$ que es decir $a=|b|q_1+r$.

                Ahora podemos poner esto como $a=bq+r$ donde si $b<0 \linfire q = -q_1$ y si
                $b>0 \linfire q=q_1$.

                Para ver que $0 \leq r < |b|$, bueno, es mayor o igual que $0$ porque pertenece a los
                Naturales más el cero, ahora para ver que es menor que $|b|$, basta con ver que si no
                fuera así  pasaría que $r-|b| \geq 0$ (donde $r$ es el elemento más pequeño del conjunto
                $Residuos$) que es lo mismo que poner $(a-|b|q_1)-|b| \geq 0$ que es lo mismo que
                $a-|b|(q_1+1)\geq 0$, ahora basta con ver que esa no es la $r$ más pequeña,
                pues entonces si $a-|b|(q_1+1)\geq 0$, también $a-|b|q_1\geq 0$, por lo que la nueva $r_2$
                (donde $r_2 = a-|b|q_1$), es mas pequeña que $r$, pero elegimos a $r$ como la más pequeña,
                por lo tanto contradicción.

                Y ya por fin, para demostrar que $q, r$ son únicos dados $a, b$, tendría que pasar
                que $a=bq_1+r_1=bq_2+r_2$.

                Recordemos que $r$ debe de ser única, pues $r$ es el menor elemento del conjunto
                del que tendríamos que sacar a la otra, así que $r$ solo hay una.

                Dado eso, tenemos que $a=bq_1+r=bq_2+r$ que es lo mismo que $bq_1=bq_2$
                que es lo mismo que $q_1=q_2$ y bingo. Demostrado.

            \end{SmallIndentation}



        % ==================================
        % ======   PAR E INPAR     =========
        % ==================================
        \clearpage
        \subsection{Par e Inpar}

            Dado un 2 como divisor, osea $b=2$, nuestra $r$ siempre será 0 ó 1.
            Digo recuerda que $0\leq r < |b|$.

            \subsubsection{Pares}
                Por lo tanto puedo definir a un número entero par como aquellos números
                que podemos escribirlos gracias al algoritmo de la división como
                $2q+0$ o de manera más común como $2k$.

                \begin{equation}
                \begin{split}
                    Pares &= \{a \in \Integers \Such a = 2q + 0, \MiniSpace q \in \Integers \}\\
                    Pares &= \{2k \Such k \in \Integers\}
                \end{split}
                \end{equation}

            \subsubsection{Inpares}
                Por lo tanto puedo definir a un número entero inpar como aquellos números
                que podemos escribirlos gracias al algoritmo de la división como
                $2q+1$ o de manera más común como $2k+1$.

                \begin{equation}
                \begin{split}
                    Pares &= \{a \in \Integers \Such a = 2q + 1, \MiniSpace q \in \Integers \}\\
                    Pares &= \{2k+1 \Such k \in \Integers\}
                \end{split}
                \end{equation}


            Y de esto sacamos algunas ideas bastante obvias:

            % ==================================
            % ======   IMPORTANTES     =========
            % ==================================
            \subsubsection{Ideas Importantes}
            \begin{itemize}

                \item Un número $n$ es un cuadrado $n=m^2$ si y solo si al aplicarle
                el algoritmo de la división con $b=4$ implica que $r=1$ ó $r=0$.

                    % ======== DEMOSTRACION ========
                    \begin{SmallIndentation}[1em]
                        \textbf{Demostración}:

                        Si es un número par $m=2k$, entonces $(2k)^2$ que 
                        es igual a $4k^2$ donde podemos decir que
                        $n=4(k^2)+0$.

                        Si es inpar $m=2k+1$, entonces $(2k+1)^2$ que 
                        es igual a $4k^2+4k+1$ donde podemos decir que
                        $n=4(k^2+k)+1$. 

                    \end{SmallIndentation}

                \end{itemize}




    % =====================================================
    % ============        DIVISIBILIDAD            ========
    % =====================================================
    \clearpage
    \section{Divisibilidad}



        % ==================================
        % =========   FORMAL     ===========
        % ==================================
        \subsection*{Definición Formal}

            Dados dos números cualquiera $a, b \in \Integers$. Decimos que la proposición
            \textbf{\Quote{b} divide a \Quote{a}} $b|a$ es verdad si y solo si 
            $\exists q \in \Integers, \MiniSpace a = bq$.


            \begin{itemize}

                \item Los divisores de a son el conjunto:
                    \begin{equation*}
                        Divisores = \{ x \in \Integers \Such x|a \}
                    \end{equation*}


                \item Los múltiplos de b son:
                    \begin{equation*}
                        Multiplos = \{ x \in \Integers \Such b|x   \}
                    \end{equation*}

            \end{itemize}




        % ==================================
        % =========   ALTERNA     ==========
        % ==================================
        \subsection*{Definición Alterna}
            Veamos que lo que de verdad nos estan preguntando si es que $\frac{a}{b} \in \Integers$.

            \textbf{Podemos entonces enunciar que: \Quote{a divide a b si y solo si es que
            $\frac{a}{b}$ continua estando en los enteros}}.


            % ======== DEMOSTRACION ========
            \begin{SmallIndentation}[1em]
                \textbf{Demostración}:

                Podemos ver que nos estan preguntando lo mismo, ya que si mi definición
                alterna es verdad, eso quiere decir que podemos escribir a $a$ como $a = bq$.
                Y con esto logramos ver que $\frac{bq}{b} = q$ y habiamos dicho que $q \in \Integers$.

            \end{SmallIndentation}



            


        % ==================================
        % ======      EJEMPLOS     =========
        % ==================================
        \clearpage
        \subsection{Ejemplos}

            Supongamos que elegimos la proposición $5|35$.

            Entonces lo que nos estan preguntando en el fondo es si $\frac{35}{5} \in \Integers$
            podemos ver que si, pues $\frac{35}{5} = 7$.

            Podemos también decir que:

            \begin{itemize}

                \item Los divisores de 35 son:
                    \begin{equation*}
                    \begin{split}
                        Divisores &= \{ b \in \Integers \Such b|35 \}   \\
                        Divisores &= \{ \pm 1, \pm 3, \pm 7, \pm 35 \}
                    \end{split}
                    \end{equation*}


                \item Los múltiplos de 5 son:
                    \begin{equation*}
                    \begin{split}
                        Multiplos &= \{ a \in \Integers \Such 5|a   \}  \\
                        Multiplos &= \{ \dots,-10,-5,0,5,10, \dots  \}
                    \end{split}
                    \end{equation*}

            \end{itemize}







        % ==================================
        % ======   PROPIEDADES     =========
        % ==================================
        \clearpage
        \subsection{Propiedades de Divisilibidad}

            \begin{itemize}

                \item $b|b$

                    % ======== DEMOSTRACION ========
                    \begin{SmallIndentation}[1em]
                        \textbf{Demostración}:

                        Basta con ver que si $a = b$ entonces $b = bq$, por lo tanto $q = 1$.
                        Y listo, $1 \in \Integers$.

                    \end{SmallIndentation}

                \item $b|0$

                    % ======== DEMOSTRACION ========
                    \begin{SmallIndentation}[1em]
                        \textbf{Demostración}:

                        Basta con ver que si $a = 0$ entonces $0 = bq$, por lo tanto $q = 0$.
                        Y listo, $0 \in \Integers$.

                    \end{SmallIndentation}

                \item $1|a$ y también $-1|a$

                    % ======== DEMOSTRACION ========
                    \begin{SmallIndentation}[1em]
                        \textbf{Demostración}:

                        Basta con ver que si $b = \pm 1$ entonces $a = \pm q$, por lo tanto $q = \pm a$.
                        Y listo, $\pm a \in \Integers$.

                    \end{SmallIndentation}
            
                \item $0|a$ si y solo $a=0$

                    % ======== DEMOSTRACION ========
                    \begin{SmallIndentation}[1em]
                        \textbf{Demostración}:

                        Basta con ver que tenemos $a = 0q$, esto es lo mismo que $a = 0$.

                    \end{SmallIndentation}

                \item $b|1$ si y solo si $b=1$ ó $b=-1$

                    % ======== DEMOSTRACION ========
                    \begin{SmallIndentation}[1em]
                        \textbf{Demostración}:

                        Sabemos que $a=1=bq$, esto nos obliga a que $b=\frac{1}{q}$, ahora tenemos que
                        recordar que $b, q \in \Integers$, por lo tanto $q=1$ o bien $q=-1$ que es lo
                        mismo que decir que $b=1$ ó $b=-1$. 

                    \end{SmallIndentation}


                \item $b|a$ y $a|b$ si y solo si $a=\pm b$

                    % ======== DEMOSTRACION ========
                    \begin{SmallIndentation}[1em]
                        \textbf{Demostración}:

                        Sabemos que $a=bq_1$, y $b=aq_2$ por lo tanto podemos sustituir, 
                        $a=(aq_2)q_1$ por lo tanto $1=(q_1)(q_2)$, que es lo mismo que
                        $\frac{1}{q_2}=q_1$ ahora que para $q_1$ siga en los $\Integers$,
                        $q_2 = \pm 1$ por lo tanto $q_1 = \pm \frac{1}{1} = \pm 1$ por
                        lo tanto tenemos que $a = bq_1$ que es lo mismo que decir que $a = \pm b$.

                    \end{SmallIndentation}


                \item Si $b|a$ y $a|c$ entonces $b|c$

                    % ======== DEMOSTRACION ========
                    \begin{SmallIndentation}[1em]
                        \textbf{Demostración}:

                        Sabemos que $a=bq_1$, y $c=aq_2$ por lo tanto podemos sustituir, 
                        $c=(bq_1)q_2$ que es lo mismo que $c=bq_3$, donde $q_3 = q_1q_2$
                        donde $q_3 \in \Integers$. 
                        Y ya que $c=bq_3$ podemos decir que $b|c$.

                    \end{SmallIndentation}

                \clearpage

                \item Si $b|a$ y $b|c$ entonces $b|a \pm c$

                    % ======== DEMOSTRACION ========
                    \begin{SmallIndentation}[1em]
                        \textbf{Demostración}:

                        Sabemos que $a=bq_1$, y $c=bq_2$ por lo tanto podemos decir que
                        sumar o restar ambas ecuaciones, lo que nos daría
                        $a \pm c = bq_1 \pm bq_2$ que es lo mismo que
                        $a \pm c = b (q_1 \pm q_2)$ por lo que podemos decir que $b|a \pm c$.

                    \end{SmallIndentation}

                \item Si $b|a$ entonces $b|ak$ $\forall k \in \Integers$.

                    % ======== DEMOSTRACION ========
                    \begin{SmallIndentation}[1em]
                        \textbf{Demostración}:

                        Sabemos que $a=bq$ por lo mismo podemos decir que
                        $ak=b(qk)$ por lo tanto $b|ak$.

                    \end{SmallIndentation}

                \item $b|a$ si y solo si $b|-a$ si y solo si $-b|a$ si y solo si $-b|-a$

                    % ======== DEMOSTRACION ========
                    \begin{SmallIndentation}[1em]
                        \textbf{Demostración}:

                        Sabemos que existe $q_1$ tal que $a=bq_1$ para nuestro primer ssi
                        basta con decir que $-a=b(-q_1) = bq_2$ y listo, encontre a $q_2$
                        con lo que puedo afirmar que $b|-a$.

                        Para el segundo basta con ver que $a = -bq_3$ donde $q_3 = q_2$, con
                        lo que puedo afirmar que $-b|a$.

                        Para el último ssi basta con con ver que $-a=-bq_4$ donde $q_4=q_1$
                        así que puedo afirmar que $-b|-a$.

                    \end{SmallIndentation}


                \item Si $b|a$ y $a \neq 0$ entonces $|b| \leq |a|$.

                    % ======== DEMOSTRACION ========
                    \begin{SmallIndentation}[1em]
                        \textbf{Demostración}:

                        Supongamos entonces que $b$ divide a $a$ y que $a \neq 0$, por lo tanto
                        la frase $a = bq$ nos da mucha información, pues obliga a que $b$ y $q$
                        no sean ninguno $0$, entonces tenemos que $a = bq$ donde $b \neq 0$ y 
                        $q \neq 0$.

                        Luego ya que no son 0, tenemos que $|q| \geq 1$ y $|b| \geq 1$, ya que sabemos
                        como funcionan los números enteros tenemos que sin importar cuanto valgan $q$ y $b$
                        se cumple que $|b||q| \geq |b|$ esto es lo mismo que $|bq| \geq |b|$ y sabemos que
                        $a = bq$, por lo tanto tenemos que $|a| \geq |b|$.

                        Esto es lo mismo que $|b| \leq |a|$

                    \end{SmallIndentation}

            \end{itemize}

           


    % =====================================================
    % =====    MAXIMO COMUN DIVISOR : GCD /MCD     ========
    % =====================================================
    \clearpage
    \section{Máximo Común Divisor: GCD/MCD}

        % ==================================
        % =========   FORMAL     ===========
        % ==================================
        \subsection*{Definición Formal}

        Dados dos números cualquiera $a, b \in \Integers$ pero con mínimo alguno de ellos dos diferentes
        de 0. 

        Entonces decimos que el máximo común divisor de $a$ y $b$ denotado por $MCD(a,b) = GCD(a,b)$
        es el entero positivo $d$ que satisface:

        \begin{itemize}
            \item $d|a$ y $d|b$
            \item Si $c|a$ y $c|b$ entonces $c \leq d$.
        \end{itemize}


        % ======== IDEAS ========
        \begin{SmallIndentation}[1em]
            \textbf{Ideas}:

            Decimos que $d$ es un división común de $a$ y $b$ si $(d|a) \land (d|b)$.

            Ahora podemos construir el conjunto de los divisores comúnes.
            $Divisores = \{ d \in \Integers \Such (d|a) \land (d|b) \}$

            Ahora si, con todo esto listo, podemos ver que este conjunto nunca estará vació. 
            como 1 es un divisón común de todos los enteros.

            Ahora podemos ver que el conjunto no es infinito siempre que alguno de ellos no sea cero,
            hay sólo una cantidad finita de divisores comunes positivos. Dentro de ellos hay uno
            que es el mayor.

            La segunda condición se asegura de que $d$ sea el máximo elemento dentro del conjunto.

        \end{SmallIndentation}




        % ==================================
        % ======   PROPIEDADES     =========
        % ==================================
        \clearpage
        \subsection{Propiedades de MCD/GCD}

            Antes que nada, recuerda que para que tenga sentido hablar del máximo común divisor
            alguno de los dos $a,b$ debe de ser diferente de cero. Porfis.

            Recuerda también llamaré $c$ a lo que salga de $c=max(|a|,|b|)$.

            Ahora supongamos que es $a$ el que es diferente de 0, después de todo
            $MCD(a,b) = MCD(b,a)$

            \begin{itemize}

                \item Siempre se cumple que $0 < MCD(a,b) \leq max(|a|,|b|)$

                    % ======== DEMOSTRACION ========
                    \begin{SmallIndentation}[1em]
                        \textbf{Demostración}:

                        Para lo primero basta con recordar que 1 divide a todos los enteros,
                        así que 1 siempre será un divisor común, por lo tanto, cualquier otro
                        divisor que aspire a ser el MCD/GCD tendría que que ser mayor que 1,
                        o bien, si son primos relativos, ser el 1.

                        Basta con pensar que $c=max(|a|,|b|)$ es más grande o igual que 1,
                        y ahora veamos que es imposible que existe un número $n$ que sea el
                        máximo común divisor donde $c < n$. Ya que de ser así pasa que
                        $max(|a|,|b|) < n$. Digamos que puedo escribir a $n=c+k$.

                        Y eso nos díria que si $|(c+k)|a$ y $a \neq 0$ entonces $|c+k| \leq |a|$

                        Pero, c es positiva, y también k, por lo tanto la proposicíon $|c+k| \leq |a|$
                        es falsa. Espero que se vea claro porque, ya si $c$ es el mayor de sus
                        valores absolutos, si le añadimos otro natural a ese número solo se puede hacer más
                        grande, haciendo imposible la frase $|c+k| \leq |a|$.

                        Por lo tanto, es imposible que exista dicha $n$.

                        Y el máximo común divisor queda atrapado en esos límites.

                    \end{SmallIndentation}


                \item Siempre se cumple que $MCD(a,0) = GCD(a,0) = |a|$

                    % ======== DEMOSTRACION ========
                    \begin{SmallIndentation}[1em]
                        \textbf{Demostración}:
                        Basta con pensar que $|a|$ divide a ambos, y es más grande que 1, así
                        que vamos bien, y despúes pensar que si existiríera algún divisor más
                        grande que $|a|$ entonces se cumpliría que  $|(|a|+k)|a$ por lo tanto tambíen
                        se cumpliría lo que dijimos antes, (que si $|(|a|+k)|a$ y $a \neq 0$
                        entonces $|(|a|+k)| \leq |a|$) y eso claro es una contradicción
                        por lo tanto, $|a|$ es siempre el mayor divisor común.

                    \end{SmallIndentation} 


                \item Siempre se cumple que $GCD(a,b) = GCD(-a,b) = GCD(a,-b) = GCD(-a,-b)$

                    % ======== DEMOSTRACION ========
                    \begin{SmallIndentation}[1em]
                        \textbf{Demostración}:
                        Si $d= GCD(a,b)$ entonces también se que si $c$ es también un divisor común
                        $c\leq d$, pero vemos que $d|-a$ y $d|-b$.

                        Ahora, vemos que d es también un divisor común, y es que es el mayor, porque
                        si $c|-a$ y $c|-b$ ya habiamos dicho que $c\leq d$.

                        Literalmente no hay otra forma. Demostrado.

                    \end{SmallIndentation} 

            \end{itemize}




        % ==================================
        % ====    IDENTIDAD DE BEZOUT   ====
        % ==================================
        \clearpage
        \subsection{Identidad de Bezout}

            Existen unos $m, n \in \Integers$ llamados coeficientes de Bezout tal que se
            cumple siempre que:

            \begin{equation}
                MCD(a,b) = GCD(a,b) = am+bn
            \end{equation}

            % ======== DEMOSTRACION ========
            \begin{SmallIndentation}[1em]
                \textbf{Demostración}:

                Este \Quote{teorema} parece bastante importante, así que veamoslo con más detalle,
                nos dice que podemos escribir al MCD/GCD de $a,b$ como una combinación lineal de
                ellos.

                Ahora, concentremos en las combinaciones lineales que sean positivas, hagamos el conjunto
                $Combinaciones = \{ am+bn \Such m, n \in \Integers, \MiniSpace am+bn > 0 \}$.

                Con esto tenemos todas las combinaciones lineales positivas.
                También sabemos que no esta vacío ese conjunto, pues mínimo $max(|a|,|b|)$ esta ahí dentro.

                Por el principio del buen orden, este conjunto tiene un primero elemento.
                Llamemos $d$ a ese elemento, donde vemos que $0 < d \leq max(|a|,|b|)$,
                esto se parece a nuestro mínimo común múltiplo.

                Veamos si es un divisor común primero, por el algoritmo de la división podemos decir
                que podemos escribir $a = dq + r$ y también como $d \in Combinaciones$, osea
                $d = am+bn$ podemos decir que $a = (am+bn)q$.

                Por lo tanto veamos que pasa si despejo r:

                $r = a-dq = a + d(-q) = a + (am+bn)(-q) = a(1-qm) + b(-qn)$

                Si no te has dado cuenta, esta de la forma $ax+by$, osea que $r$ también debería estar
                en $Combinaciones$, pero creí que $d$ era la combinación más pequeña, la única forma
                de que esto no sea una contradicción es que $r=0$, pues $0 \leq r \leq (am+bn)$
                (Inteligente, ¿no?).
                
                Así podemos darnos cuenta de que si tomamos al menor elemento de la forma $am+bn$ este
                siempre tiene que dividir a $a$, y de hecho $a$ no tiene nada de especial.
                Lo mismo pasa con $b$.

                Ok, ahora sabemos que $d$ es un divisor común, para ver que es el más pequeño
                simplemente imaginate otro, como $x$ un divisor positivo común de $a$ y $b$,
                existen entonces enteros $s,t$ tales que $a=xs \land b=xt$ y como vimos podemos poner a 
                $d$ como $d=am+bn$.

                Tenemos que $d = am+bn = (xs)m+(xt)n = x(sm+tn)$, si te das cuenta la proposición
                $x|d$ es cierta, pues $d = x(sm+tn)$, por lo que podemos decir que $|x| \leq |d|$, 
                pero vamos, ambos son positivos, eso de antes es lo mismo que $x \leq d$, por lo
                tanto por definición $d$ es nuestro máximo común divisor.

            \end{SmallIndentation}

        % ==================================
        % ======   PROPIEDADES     =========
        % ==================================
        \clearpage
        \subsection{Propiedades de MCD/GCD: Bezout Edition}
            \begin{itemize}

                \item Si tengo 3 números $a, b, c \in \Integers$ donde $c$ y alguno de los
                dos restantes $a,b$ no son cero, entonces $c$ se puede escribir como
                una combinación lineal de $a$ y $b$ si y solo si $c$ es el $GCD \ MCD$ de
                $a, b$ o bien si es uno de sus múltiplos.

                    % ======== DEMOSTRACION ========
                    \begin{SmallIndentation}[1em]
                        \textbf{Demostración}:
                            Vamos, literalmente acabo de demostrar que el GCD es equivalente
                            a escribirlos como combinación lineal, ahora también funciona con
                            los múltiplos, pues si $d$ es el $GCD$ y $c$ un múltiplo, entonces
                            tenemos que $d=am+bn$ y también $c=kd$.

                            Por lo tanto nuestra ansiada combinación lineal es simplemente
                            $c=a(km)+b(kn)$. Y ¡Bingo!

                    \end{SmallIndentation} 


                \item El conjunto $Combinaciones = \{ am+bn \Such m, n \in \Integers, \MiniSpace am+bn > 0 \}$.
                es precisamente el conjunto de múltiplos de GCD(a,b).

                    % ======== DEMOSTRACION ========
                    \begin{SmallIndentation}[1em]
                        \textbf{Demostración}:
                            Sea $d=GCD(a,b)$, si $d|m$ entonces $m = dc$ para algún $c \in \Integers$
                            y entonces $m = dc = c(am+bn)=a(cm)+b(cn)$.

                            Así que cualquier multiplo de $d$ estará en este conjunto.

                            Además es claro que $d$ divide a cualquier combinación lineal
                            de $a$,$b$ por ser un divisor común.

                    \end{SmallIndentation}

                \item La pareja de $m, n \in \Integers$ llamados coeficientes de Bezout, ya sabes
                aquella que cumple que $GCD(a,b) = am+bn$, siempre serán coprimos.

                    % ======== DEMOSTRACION ========
                    \begin{SmallIndentation}[1em]
                        \textbf{Demostración}:

                        Sabemos que existen enteros $m,n$ tal que $d = am+bn$ por la
                        identidad de Bezout, además como $d$ es un divisor común
                        podemos escribir $a=dq_1$ $b=dq_2$ para algunos enteros $q_1,q_2$.

                        Por lo que $d=am+bn = dmq_1 + dnq_2 = d(mq_1 +nq_2)$, por
                        lo tanto tenemos que $1= mq_1 + nq_2$.

                        Esto es muy importante, porque nos dice que los enteros $m$ y $n$
                        son primos relativos (Dos enteros $a,b$ son primos relativos sí y sólo
                        si,existen enteros $x,y \in \Integers$ tales que $1=am+bn$).

                        Y bingo, ahí esta nuestra pareja de primos relativos.

                    \end{SmallIndentation} 

            \end{itemize}
                            


        % ==================================
        % ====    PRIMOS RELATIVOS     =====
        % ==================================
        \clearpage
        \subsection{Primos Relativos}

            Decimos que dos enteros $a$ y $b$ son primos relativos o coprimos si $mcd(a,b) = 1$.

        % ==================================
        % ====   IDEAS INTERESANTES    =====
        % ==================================
        \subsection*{Ideas Interesantes}
            \begin{itemize}

                \item Dos enteros $a,b$ son primos relativos sí y sólo si,existen
                enteros $x,y \in \Integers$ tales que $1=am+bn$.

                    % ======== DEMOSTRACION ========
                    \begin{SmallIndentation}[1em]
                        \textbf{Demostración}:
                            Esto es literalmente un corolario de la Identidad de Bezout,
                            porque si son primos relativos, entonces $GCD(a,b)=1$, y por 
                            la identidad existen $x,y$ tal que $1=am+bn$.

                    \end{SmallIndentation} 


                \item Sea $d=GCD(a,b)$. La pareja de $(\frac{a}{d}, \frac{b}{d})$ siempre son
                primos relativos.

                    % ======== DEMOSTRACION ========
                    \begin{SmallIndentation}[1em]
                        \textbf{Demostración}:

                        Sabemos que existen enteros $m,n$ tal que $d = am+bn$ por la
                        identidad de Bezout, además como $d$ es un divisor común
                        podemos escribir $a=dq_1$ $b=dq_2$ para algunos enteros $q_1,q_2$.

                        Por lo que $d=am+bn = dmq_1 + dnq_2 = d(mq_1 +nq_2)$, por
                        lo tanto tenemos que $1= mq_1 + nq_2$.

                        Esto es muy importante, porque nos dice que los enteros $q_1$ y $q_2$
                        son primos relativos (Dos enteros $a,b$ son primos relativos sí y sólo
                        si,existen enteros $x,y \in \Integers$ tales que $1=am+bn$).

                        Por lo tanto basta con ver que $q_1=\frac{a}{d}$ y que $q_1=\frac{b}{d}$.

                        Y bingo, ahí esta nuestra pareja de primos relativos.

                    \end{SmallIndentation} 

            \end{itemize}




    % =====================================================
    % =====          ALGORITMO DE EUCLIDES         ========
    % =====================================================
    \clearpage
    \section{Algorítmo de Euclides}

        % ==================================
        % =========   FORMAL     ===========
        % ==================================
        \subsection*{Definición Formal}

            Un algoritmo eficiente para calcular el máximo común divisor de dos enteros
            se puede conseguir aplicando repetidamente el algoritmo de Euclides.

            Si intentamos calcular $GCD(a,b)$ y sabemos del algoritmo de la división 
            que $a=bq+r$ entonces podemos simplificar el problema ya que:
            \begin{equation}
                GCD(a,b) = GCD(b,r)
            \end{equation}

            Podemos seguir aplicando esta identidad hasta que el $GCD(a, b)$ séa muy
            obvio.


        % ======== DEMOSTRACION ========
        \begin{SmallIndentation}[1em]
            \textbf{Demostración}:

            Esta afirmación es la importante: $GCD(a,b) = GCD(b,r)$ donde $r$ es el residuo
            del algoritmo de la división donde $a=bq+r$.

            Para probarla lo que haremos sera darnos cuenta que el conjunto de divisores
            comunes de $a$ y $b$ será el mismo que el de $b$ y $r$.
            Es decir para una $d$ cualquiera que sea un divisor común de $a$ y $b$ si y solo si
            $d$ es un divisor de $b$ y de $r$.

            Veamos que podemos probar esto gracias a que podemos verlo como una implicación de
            dos lados.

            Por un lado si $d$ es un divisor de $a$ y $b$, es decir $d|a$ y $d|b$
            sabemos que $d|a-k$, es un resultado que ya habiamos probado, pero que
            pasa si decimos que esa $k$ no es otra $bq$, ya que de $a=bq+r$ podemos 
            ver como $r=a-bq = a-k$, con lo que vemos que $d|r$, por lo tanto
            vimos que para cualquier $d$ que divida a $a,b$ también lo hará con $b,r$.

            Por otro lado supón que $d$ es un divisor común de $b$ y $r$, entonces
            $d|b$ y $d|r$, por lo tanto $d|bq$ y si ya sabemos que $d|bq$ entonces
            también lo hace con $d|bq+r$, por lo tanto $d|a$, por lo tanto
            vimos que para cualquier $d$ que divida a $b,r$ también lo hará con $a,b$.

            Y si tienen exactamente los mismos elementos en cada conjunto de divisores
            comunes entonces creo que es bastante obvio que el máximo elemento de cada
            conjunto será el mismo, es decir, tienen el mismo GCD.

        \end{SmallIndentation}


        % ==================================
        % =====   COMO APLICARLO    ========
        % ==================================
        \clearpage
        \subsection{Como Aplicarlo}

            Esto ya nos muestra una forma de calcular el máximo común divisor de dos
            números $a,b$ de una manera más sencilla pues en principio $b,r$ son
            números más pequeños.

            \begin{itemize}
                \item
                    El primer paso es aplicar el algoritmo para la división:
                    \begin{equation*}
                        a = bq_1 +r_1 \Space 0 \leq r_1 < b
                    \end{equation*}

                    \begin{SmallIndentation}[1em]
                        Si da la casualidad de que $r_1 = 0$ entonces $b|a$, por lo
                        que $GCD(a,b) = b$. Y listo, encontrado.

                        Si no tuvimos tanta suerte podemos al menos saber que
                        $GCD(a,b) = GCD(b,r_1)$, así que volvemos a aplicar el
                        algoritmo de la división.

                    \end{SmallIndentation}

                    
                \item
                    Ahora tenemos que 
                    \begin{equation*}
                        b = r_1q_2 +r_2 \Space 0 \leq r_2 < r_1
                    \end{equation*}

                    \begin{SmallIndentation}[1em]
                        Si da la casualidad de que $r_2 = 0$ entonces $r_1|b$, por lo
                        que $GCD(a,b) = GCD(b,r_1)=r_1$. Y listo, encontrado.

                        Si no tuvimos tanta suerte podemos al menos saber que
                        $GCD(b,r_1) = GCD(r_1,r_2)$, así que volvemos a aplicar el
                        algoritmo de la división.

                    \end{SmallIndentation}

                    


                \item
                    Como los números encontrados satisfacen $0 \leq r_n < \cdots < r_2 < r_1$,
                    vemos que este proceso terminar a lo mucho en $b$ pasos, es decir para algún
                    $n \leq b$ debemos tener que $r_n=0$ y entonces:

                    \begin{equation*}
                    \begin{split}
                        GCD(a,b)    &= GCD(b,r_1)           \\
                                    &= GCD(r_1,r_2)         \\
                                    &= \cdots               \\
                                    &= GCD(r_{n-2},r_{n-1}) \\
                                    &= GCD(r_{n-1}, 0)      \\
                                    &= r_{n-1} 
                    \end{split}
                    \end{equation*}

            \end{itemize}

            En otras palabras, el máximo común divisor de $a,b$ es el último residuo distinto
            de cero al aplicar repetidamente el algoritmo de la división como en proceso anterior.



        % ==================================
        % =====      EJEMPLO        ========
        % ==================================
        \clearpage
        \subsection{Ejemplo}
            Supón que tenemos que calcular el $GCD(2024, 748)$


            \begin{itemize}
                \item $GCD(2024, 748) = GCD(748, 528)$   donde $2024=748(2)+528$

                \item $GCD(748, 528)  = GCD(528, 220)$   donde $748=528(1)+220$

                \item $GCD(528, 220)  = GCD(220, 88)$    donde $528=220(2)+88$

                \item $GCD(220, 88)   = GCD(88, 44)$     donde $220=88(2)+44$

                \item $GCD(88, 44)    = GCD(44, 0) = 44$ donde $88=44(2)+0$
            \end{itemize}

            Y bingo, 44.





        % ==================================
        % ==  USARLO COEFICIENTES BEZUT  ===
        % ==================================
        \clearpage
        \subsection{Usarlo para encontrar los Coeficientes de Bezout}

            Nosotros sabemos que el máximo común divisor de dos números 
            $GCD(a,b)=ax+by$ para algunos enteros $x,y$. Estos son conocidos
            como los coeficientes de Bezout.

            El algoritmo de Euclides también nos brinda una forma de encontrarlos.

            Digamos que $r_n=0$, si $GCD(a,b)=r_{n-1}$ entonces


















% ======================================================================================
% =============================         COMBINATORIA          ==========================
% ======================================================================================
\chapter{Combinatoria}
    \clearpage

    % =====================================================
    % ============           DEFINICION            ========
    % =====================================================
    \section{Definición}

        Una relación $R$ entre dos conjuntos $A$ y $B$ es ante todo otro conjunto, una relación binaria es aquella
        que es en el fondo un conjunto de pares ordenados (x,y) donde $x$ es un elemento de $A$, y así mismo
        $y$ es un elemento de $B$.

        Este nuevo conjunto $R$ nos muestra como es que esta relacionados algunos (o todos) elementos de $A$ con
        otros elementos de $B$.



        % =====================================
        % =========   ¿QUE SON?     ===========
        % =====================================
        \subsection*{Definiciones Formales}

            Una Relación $R: A \to B$ es un subconjunto de $A \times B$.

            Solemos escribir la proposición $(x, y) \in R$ como $x R y$ para que se vea más bonito.

            Solemos escribir la proposición $(x, y) \notin R$ como $x\centernot{R}y$ para que se vea más bonito.








\end{document}
