% ****************************************************************************************
% ************************     	RELACIONES Y FUNCIONES   	  ****************************
% ****************************************************************************************


% =======================================================
% =======         HEADER FOR DOCUMENT        ============
% =======================================================
    % *********   DOCUMENT ITSELF   **************
    \documentclass[12pt]{report}                                    %Type of docuemtn and size of font
    \usepackage[margin=1.2in]{geometry}                             %Margins and Geometry pacakge
    \usepackage{ifthen}                                             %Allow simple programming
    \usepackage{hyperref}                                           %Create MetaData for a PDF and LINKS!
    \setlength{\parindent}{0pt}                                     %Eliminate ugly indentation
    \author{Oscar Andrés Rosas}                                     %Who I am

    % *********   LANGUAJE AND UFT-8   *********
    \usepackage[spanish]{babel}                                     %Please use spanish
    \usepackage[utf8]{inputenc}                                     %Please use spanish - UFT
    \usepackage[T1]{fontenc}                                        %Please use spanish
    \usepackage{textcmds}                                           %Allow us to use quoutes
    \usepackage{changepage}                                         %Allow us to use identate paragraphs

    % *********   MATH AND HIS STYLE  *********
    \usepackage{amsthm, amssymb, amsfonts, mathrsfs}                %Make math beautiful
    \usepackage[fleqn]{amsmath}                                     %Please make equations left
    \usepackage{centernot}                                          %Allow me to negate a symbol
    \decimalpoint                                                   %Use decimal point

    % *********   GRAPHICS AND IMAGES *********
    \usepackage{graphicx}                                           %Allow to create graphics
    \usepackage{wrapfig}                                            %Allow to create images
    \graphicspath{ {Graphics/} }                                    %Where are the images :D

    % *********   LISTS AND TABLES ***********
    \usepackage{listings}                                           %We will be using code here
    \usepackage[inline]{enumitem}                                   %We will need to enumarate
    \usepackage{tasks}                                              %Horizontal lists
    \usepackage{longtable}                                          %Lets make tables awesome
    \usepackage{booktabs}                                           %Lets make tables awesome
    \usepackage{tabularx}                                           %Lets make tables awesome
    \usepackage{multirow}                                           %Lets make tables awesome
    \usepackage{multicol}                                           %Create multicolumns

    % *********   HEADERS AND FOOTERS ********
    \usepackage{fancyhdr}                                           %Lets make awesome headers/footers
    \pagestyle{fancy}                                               %Lets make awesome headers/footers
    \setlength{\headheight}{16pt}                                   %Top line
    \setlength{\parskip}{0.5em}                                     %Top line
    \renewcommand{\footrulewidth}{0.5pt}                            %Bottom line

    \lhead{                                                         %Left Header
        \hyperlink{chapter.\arabic{chapter}}                        %Make a link to the current chapter
        {\normalsize{\textsc{\nouppercase{\leftmark}}}}             %And fot it put the name
    }

    \rhead{                                                         %Right Header
        \hyperlink{section.\arabic{chapter}.\arabic{section}}       %Make a link to the current chapter
            {\footnotesize{\textsc{\nouppercase{\rightmark}}}}      %And fot it put the name
    }

    \rfoot{\textsc{\small{\hyperref[sec:Index]{Ve al Índice}}}}    %This will always be a footer  

    \fancyfoot[L]{                                                  %Algoritm for a changing footer
        \ifthenelse{\isodd{\value{page}}}                           %IF ODD PAGE:
            {\href{https://compilandoconocimiento.com/yo/}          %DO THIS:
                {\footnotesize                                      %Send the page
                    {\textsc{Oscar Andrés Rosas}}}}                 %Send the page
            {\href{https://compilandoconocimiento.com}              %ELSE DO THIS: 
                {\footnotesize                                      %Send the author
                    {\textsc{Compilando Conocimiento}}}}            %Send the author
    }
    
    
    
% ========================================
% ===========   COMMANDS    ==============
% ========================================

    % =====  GENERAL TEXT  ==========
    \newcommand \Quote {\qq}                                        %Use: \Quote to use quotes
    \newenvironment{Indentation}[1][0.75em]                         %Use: \begin{Inde...}[Num]...\end{Inde...}
    {\begin{adjustwidth}{#1}{}}                                     %If you dont put nothing i will use 0.75 em
    {\end{adjustwidth}}                                             %This indentate a paragraph
    \newenvironment{SmallIndentation}[1][0.75em]                    %Use: The same that we upper one, just 
    {\begin{adjustwidth}{#1}{}\begin{footnotesize}}                 %footnotesize size of letter by default
    {\end{footnotesize}\end{adjustwidth}}                           %that's it
        
    % =====  GENERAL MATH  ==========
    \DeclareMathOperator \Space {\quad}                             %Use: \Space for a cool mega space
    \DeclareMathOperator \MiniSpace {\;}                            %Use: \Space for a cool mini space
    \newcommand \Such {\MiniSpace|\MiniSpace}                       %Use: \Such like in sets

    % =====  LOGIC  ==================
    \DeclareMathOperator \doublearrow {\leftrightarrow}             %Use: \doublearrow for a double arrow
    \newcommand \lequal {\MiniSpace \Leftrightarrow \MiniSpace}     %Use: \lequal for a double arrow
    \newcommand \linfire {\MiniSpace \Rightarrow \MiniSpace}        %Use: \lequal for a double arrow

    % =====  NUMBER THEORY  ==========
    \DeclareMathOperator \Naturals  {\mathbb{N}}                     %Use: \Naturals por Notation
    \DeclareMathOperator \Primes    {\mathbb{P}}                     %Use: \Naturals por Notation
    \DeclareMathOperator \Integers  {\mathbb{Z}}                     %Use: \Integers por Notation
    \DeclareMathOperator \Racionals {\mathbb{Q}}                     %Use: \Racionals por Notation
    \DeclareMathOperator \Reals     {\mathbb{R}}                     %Use: \Reals por Notation
    \DeclareMathOperator \Complexs  {\mathbb{C}}                     %Use: \Complex por Notation

    % === LINEAL ALGEBRA & VECTORS ===
    \DeclareMathOperator \LinealTransformation {\mathcal{T}}        %Use: \LinealTransformation for a cool T

    \newcommand{\pVector}[1]{                                       %Use: \pVector {Matrix Notation} use parentesis
        \ensuremath{\begin{pmatrix}#1\end{pmatrix}}                 %Example: \pVector{a\\b\\c} or \pVector{a&b&c} 
    }
    \newcommand{\lVector}[1]{                                       %Use: \lVector {Matrix Notation} use a abs 
        \ensuremath{\begin{vmatrix}#1\end{vmatrix}}                 %Example: \lVector{a\\b\\c} or \lVector{a&b&c} 
    }
    \newcommand{\Vector}[1]{                                        %Use: \Vector {Matrix Notation} no parentesis
        \ensuremath{\begin{matrix}#1\end{matrix}}                   %Example: \Vector{a\\b\\c} or \Vector{a&b&c}
    }



% =====================================================
% ============     	  COVER PAGE	   ================
% =====================================================
\begin{document}
\begin{titlepage}

	\center
	% ============ UNIVERSITY NAME AND DATA =========
	\textbf{\textsc{\Large Proyecto Compilando Conocimiento}}\\[1.0cm] 
	\textsc{\Large Matemáticas Discretas}\\[1.0cm] 

	% ============ NAME OF THE DOCUMENT  ============
	\rule{\linewidth}{0.5mm} \\[1.0cm]
		{ \huge \bfseries Teoría de Números}\\[1.0cm] 
	\rule{\linewidth}{0.5mm} \\[2.0cm]
	
	% ====== SEMI TITLE ==========
	{\LARGE Una Pequeña Introducción}\\[7cm] 
	
	% ============  MY INFORMATION  =================
	\begin{center} \large
	\textbf{\textsc{Autor:}}\\
	Rosas Hernandez Oscar Andres
	\end{center}

	\vfill

\end{titlepage}

% =====================================================
% ========                INDICE              =========
% =====================================================
\tableofcontents{}
\label{sec:Index}

\clearpage










% ======================================================================================
% ===========================    ENTEROS Y NATURALES          ==========================
% ======================================================================================
\chapter{Naturales}
    \clearpage

    % =====================================================
    % ==========     PRINCIPIO BUEN ORDEN        ==========
    % =====================================================
    \section{Principio de Buen Orden}

        % ==================================
        % =========   FORMAL     ===========
        % ==================================
        \subsection*{Definición Formal}





% ======================================================================================
% ===========================         DIVISIBILIDAD           ==========================
% ======================================================================================
\chapter{Divisibilidad}
    \clearpage

     

    % =====================================================
    % ==========     ALGORITMO DE LA DIVISIÓN     =========
    % =====================================================
    \clearpage
    \section{Algoritmo de División}

        % ==================================
        % =========   FORMAL     ===========
        % ==================================
        \subsection*{Definición Formal}

            Dados dos enteros $a, b$ donde $b \neq 0$, existen otros dos 
            enteros únicos $q, r$, donde $0 \leq r < |b|$ tal que se cumple:

            \begin{equation}
                a = bq+r
            \end{equation}

            Vemos que basicamente nos dice cuántas veces cabe $b$ en $a$ sin pasarse (esto 
            es $q$) y cuantos le faltan para alcanzar a $a$ (esto es $r$).

            % ======== DEMOSTRACION ========
            \begin{SmallIndentation}[1em]
                \textbf{Demostración}:

                El primer paso es crear el conjunto
                $Residuos = \{a-|b|q \Such q \in \Integers, (a-|b|q) \geq 0 \}$.

                Ahora lo primero que tenemos que ver que es $|Residuos| \neq 0$. Para hacerlo
                veamos por casos, si $a < |b|$, entonces intenta a $q=-1$ y vemos que $a+|b|$
                siempre sera mayor o igual que $0$.
                Si $a > |b|$, entonces intenta a $q=1$ y vemos que $a-|b|$ siempre sera mayor
                o igual que $0$.
                Finalmente si $a = |b|$ cualquiera de los 2 ejemplos anteriores te sirven.
                Por lo tanto mínimo $Residuos$ tiene mínimo un elemento.

                Esto es un conjunto que basicamente contiene a los residuos, o visto de otra
                manera a los números que salen como resultado de sumarle multiplos de $|b|$ a $a$
                y que son mayores que 0.

                Ahora gracias al principio de buen orden (y que $Residuos$ es el conjunto
                de los Naturales más el cero) podemos llamar a $r$ al elemento mas pequeño
                de este conjunto.

                Ahora, gracias a la definición del conjunto $Residuos$ podemos decir que
                $r=a-|b|q_1$ que es decir $a=|b|q_1+r$.

                Ahora podemos poner esto como $a=bq+r$ donde si $b<0 \linfire q = -q_1$ y si
                $b>0 \linfire q=q_1$.

                Para ver que $0 \leq r < |b|$, bueno, es mayor o igual que $0$ porque pertenece a los
                Naturales más el cero, ahora para ver que es menor que $|b|$, basta con ver que si no
                fuera así  pasaría que $r-|b| \geq 0$ (donde $r$ es el elemento más pequeño del conjunto
                $Residuos$) que es lo mismo que poner $(a-|b|q_1)-|b| \geq 0$ que es lo mismo que
                $a-|b|(q_1+1)\geq 0$, ahora basta con ver que esa no es la $r$ más pequeña,
                pues entonces si $a-|b|(q_1+1)\geq 0$, también $a-|b|q_1\geq 0$, por lo que la nueva $r_2$
                (donde $r_2 = a-|b|q_1$), es mas pequeña que $r$, pero elegimos a $r$ como la más pequeña,
                por lo tanto contradicción.

                Y ya por fin, para demostrar que $q, r$ son únicos dados $a, b$, tendría que pasar
                que $a=bq_1+r_1=bq_2+r_2$.

                Recordemos que $r$ debe de ser única, pues $r$ es el menor elemento del conjunto
                del que tendríamos que sacar a la otra, así que $r$ solo hay una.

                Dado eso, tenemos que $a=bq_1+r=bq_2+r$ que es lo mismo que $bq_1=bq_2$
                que es lo mismo que $q_1=q_2$ y bingo. Demostrado.

            \end{SmallIndentation}



        % ==================================
        % ======   PAR E INPAR     =========
        % ==================================
        \clearpage
        \subsection{Par e Inpar}

            Dado un 2 como divisor, osea $b=2$, nuestra $r$ siempre será 0 ó 1.
            Digo recuerda que $0\leq r < |b|$.

            \subsubsection{Pares}
                Por lo tanto puedo definir a un número entero par como aquellos números
                que podemos escribirlos gracias al algoritmo de la división como
                $2q+0$ o de manera más común como $2k$.

                \begin{equation}
                \begin{split}
                    Pares &= \{a \in \Integers \Such a = 2q + 0, \MiniSpace q \in \Integers \}\\
                    Pares &= \{2k \Such k \in \Integers\}
                \end{split}
                \end{equation}

            \subsubsection{Inpares}
                Por lo tanto puedo definir a un número entero inpar como aquellos números
                que podemos escribirlos gracias al algoritmo de la división como
                $2q+1$ o de manera más común como $2k+1$.

                \begin{equation}
                \begin{split}
                    Pares &= \{a \in \Integers \Such a = 2q + 1, \MiniSpace q \in \Integers \}\\
                    Pares &= \{2k+1 \Such k \in \Integers\}
                \end{split}
                \end{equation}


            Y de esto sacamos algunas ideas bastante obvias:

            % ==================================
            % ======   IMPORTANTES     =========
            % ==================================
            \subsubsection{Ideas Importantes}
            \begin{itemize}

                \item Un número $n$ es un cuadrado $n=m^2$ si y solo si al aplicarle
                el algoritmo de la división con $b=4$ implica que $r=1$ ó $r=0$.

                    % ======== DEMOSTRACION ========
                    \begin{SmallIndentation}[1em]
                        \textbf{Demostración}:

                        Si es un número par $m=2k$, entonces $(2k)^2$ que 
                        es igual a $4k^2$ donde podemos decir que
                        $n=4(k^2)+0$.

                        Si es inpar $m=2k+1$, entonces $(2k+1)^2$ que 
                        es igual a $4k^2+4k+1$ donde podemos decir que
                        $n=4(k^2+k)+1$. 

                    \end{SmallIndentation}

                \end{itemize}




    % =====================================================
    % ============        DIVISIBILIDAD            ========
    % =====================================================
    \clearpage
    \section{Divisibilidad}



        % ==================================
        % =========   FORMAL     ===========
        % ==================================
        \subsection*{Definición Formal}

            Dados dos números cualquiera $a, b \in \Integers$. Decimos que la proposición
            \textbf{\Quote{b} divide a \Quote{a}} $b|a$ es verdad si y solo si 
            $\exists q \in \Integers, \MiniSpace a = bq$.


            \begin{itemize}

                \item Los divisores de a son el conjunto:
                    \begin{equation*}
                        Divisores = \{ x \in \Integers \Such x|a \}
                    \end{equation*}


                \item Los múltiplos de b son:
                    \begin{equation*}
                        Multiplos = \{ x \in \Integers \Such b|x   \}
                    \end{equation*}

            \end{itemize}




        % ==================================
        % =========   ALTERNA     ==========
        % ==================================
        \subsection*{Definición Alterna}
            Veamos que lo que de verdad nos estan preguntando si es que $\frac{a}{b} \in \Integers$.

            \textbf{Podemos entonces enunciar que: \Quote{b divide a a si y solo si es que
            $\frac{a}{b}$ continua estando en los enteros}}.


            % ======== DEMOSTRACION ========
            \begin{SmallIndentation}[1em]
                \textbf{Demostración}:

                Podemos ver que nos estan preguntando lo mismo, ya que si mi definición
                alterna es verdad, eso quiere decir que podemos escribir a $a$ como $a = bq$.
                Y con esto logramos ver que $\frac{bq}{b} = q$ y habiamos dicho que $q \in \Integers$.

            \end{SmallIndentation}



            


        % ==================================
        % ======      EJEMPLOS     =========
        % ==================================
        \clearpage
        \subsection{Ejemplos}

            Supongamos que elegimos la proposición $5|35$.

            Entonces lo que nos estan preguntando en el fondo es si $\frac{35}{5} \in \Integers$
            podemos ver que si, pues $\frac{35}{5} = 7$.

            Podemos también decir que:

            \begin{itemize}

                \item Los divisores de 35 son:
                    \begin{equation*}
                    \begin{split}
                        Divisores &= \{ b \in \Integers \Such b|35 \}   \\
                        Divisores &= \{ \pm 1, \pm 3, \pm 7, \pm 35 \}
                    \end{split}
                    \end{equation*}


                \item Los múltiplos de 5 son:
                    \begin{equation*}
                    \begin{split}
                        Multiplos &= \{ a \in \Integers \Such 5|a   \}  \\
                        Multiplos &= \{ \dots,-10,-5,0,5,10, \dots  \}
                    \end{split}
                    \end{equation*}

            \end{itemize}







        % ==================================
        % ======   PROPIEDADES     =========
        % ==================================
        \clearpage
        \subsection{Propiedades de Divisilibidad}

            \begin{itemize}

                \item $b|b$

                    % ======== DEMOSTRACION ========
                    \begin{SmallIndentation}[1em]
                        \textbf{Demostración}:

                        Basta con ver que si $a = b$ entonces $b = bq$, por lo tanto $q = 1$.
                        Y listo, $1 \in \Integers$.

                    \end{SmallIndentation}

                \item $b|0$

                    % ======== DEMOSTRACION ========
                    \begin{SmallIndentation}[1em]
                        \textbf{Demostración}:

                        Basta con ver que si $a = 0$ entonces $0 = bq$, por lo tanto $q = 0$.
                        Y listo, $0 \in \Integers$.

                    \end{SmallIndentation}

                \item $1|a$ y también $-1|a$

                    % ======== DEMOSTRACION ========
                    \begin{SmallIndentation}[1em]
                        \textbf{Demostración}:

                        Basta con ver que si $b = \pm 1$ entonces $a = \pm q$, por lo tanto $q = \pm a$.
                        Y listo, $\pm a \in \Integers$.

                    \end{SmallIndentation}
            
                \item $0|a$ si y solo $a=0$

                    % ======== DEMOSTRACION ========
                    \begin{SmallIndentation}[1em]
                        \textbf{Demostración}:

                        Basta con ver que tenemos $a = 0q$, esto es lo mismo que $a = 0$.

                    \end{SmallIndentation}

                \item $b|1$ si y solo si $b=1$ ó $b=-1$

                    % ======== DEMOSTRACION ========
                    \begin{SmallIndentation}[1em]
                        \textbf{Demostración}:

                        Sabemos que $a=1=bq$, esto nos obliga a que $b=\frac{1}{q}$, ahora tenemos que
                        recordar que $b, q \in \Integers$, por lo tanto $q=1$ o bien $q=-1$ que es lo
                        mismo que decir que $b=1$ ó $b=-1$. 

                    \end{SmallIndentation}


                \item $b|a$ y $a|b$ si y solo si $a=\pm b$

                    % ======== DEMOSTRACION ========
                    \begin{SmallIndentation}[1em]
                        \textbf{Demostración}:

                        Sabemos que $a=bq_1$, y $b=aq_2$ por lo tanto podemos sustituir, 
                        $a=(aq_2)q_1$ por lo tanto $1=(q_1)(q_2)$, que es lo mismo que
                        $\frac{1}{q_2}=q_1$ ahora que para $q_1$ siga en los $\Integers$,
                        $q_2 = \pm 1$ por lo tanto $q_1 = \pm \frac{1}{1} = \pm 1$ por
                        lo tanto tenemos que $a = bq_1$ que es lo mismo que decir que $a = \pm b$.

                    \end{SmallIndentation}


                \item Si $b|a$ y $a|c$ entonces $b|c$

                    % ======== DEMOSTRACION ========
                    \begin{SmallIndentation}[1em]
                        \textbf{Demostración}:

                        Sabemos que $a=bq_1$, y $c=aq_2$ por lo tanto podemos sustituir, 
                        $c=(bq_1)q_2$ que es lo mismo que $c=bq_3$, donde $q_3 = q_1q_2$
                        donde $q_3 \in \Integers$. 
                        Y ya que $c=bq_3$ podemos decir que $b|c$.

                    \end{SmallIndentation}

                \clearpage

                \item Si $b|a$ y $b|c$ entonces $b|a \pm c$

                    % ======== DEMOSTRACION ========
                    \begin{SmallIndentation}[1em]
                        \textbf{Demostración}:

                        Sabemos que $a=bq_1$, y $c=bq_2$ por lo tanto podemos decir que
                        sumar o restar ambas ecuaciones, lo que nos daría
                        $a \pm c = bq_1 \pm bq_2$ que es lo mismo que
                        $a \pm c = b (q_1 \pm q_2)$ por lo que podemos decir que $b|a \pm c$.

                    \end{SmallIndentation}

                \item Si $a|b$ y $a|b \pm c$ entonces $a|c$

                    % ======== DEMOSTRACION ========
                    \begin{SmallIndentation}[1em]
                        \textbf{Demostración}:

                        Sabemos que $b=aq_1$, y $b \pm c=aq_2$, si restamos tenemos que
                        $b \pm c -b = aq_2 - aq_1$, que es lo mismo que $\pm c = (q_2-q_1)a$,
                        que es lo mismo que $c = \pm(q_2-q_1)a$ que es lo mismo que $c = q_3a$.

                    \end{SmallIndentation}

                \item Si $b|a$ entonces $b|ak$ $\forall k \in \Integers$.

                    % ======== DEMOSTRACION ========
                    \begin{SmallIndentation}[1em]
                        \textbf{Demostración}:

                        Sabemos que $a=bq$ por lo mismo podemos decir que
                        $ak=b(qk)$ por lo tanto $b|ak$.

                    \end{SmallIndentation}

                \item $b|a$ si y solo si $b|-a$ si y solo si $-b|a$ si y solo si $-b|-a$

                    % ======== DEMOSTRACION ========
                    \begin{SmallIndentation}[1em]
                        \textbf{Demostración}:

                        Sabemos que existe $q_1$ tal que $a=bq_1$ para nuestro primer ssi
                        basta con decir que $-a=b(-q_1) = bq_2$ y listo, encontre a $q_2$
                        con lo que puedo afirmar que $b|-a$.

                        Para el segundo basta con ver que $a = -bq_3$ donde $q_3 = q_2$, con
                        lo que puedo afirmar que $-b|a$.

                        Para el último ssi basta con con ver que $-a=-bq_4$ donde $q_4=q_1$
                        así que puedo afirmar que $-b|-a$.

                    \end{SmallIndentation}


                \item Si $b|a$ y $a \neq 0$ entonces $|b| \leq |a|$.

                    % ======== DEMOSTRACION ========
                    \begin{SmallIndentation}[1em]
                        \textbf{Demostración}:

                        Supongamos entonces que $b$ divide a $a$ y que $a \neq 0$, por lo tanto
                        la frase $a = bq$ nos da mucha información, pues obliga a que $b$ y $q$
                        no sean ninguno $0$, entonces tenemos que $a = bq$ donde $b \neq 0$ y 
                        $q \neq 0$.

                        Luego ya que no son 0, tenemos que $|q| \geq 1$ y $|b| \geq 1$, ya que sabemos
                        como funcionan los números enteros tenemos que sin importar cuanto valgan $q$ y $b$
                        se cumple que $|b||q| \geq |b|$ esto es lo mismo que $|bq| \geq |b|$ y sabemos que
                        $a = bq$, por lo tanto tenemos que $|a| \geq |b|$.

                        Esto es lo mismo que $|b| \leq |a|$

                    \end{SmallIndentation}

            \end{itemize}

           


    % =====================================================
    % =====    MAXIMO COMUN DIVISOR : GCD /MCD     ========
    % =====================================================
    \clearpage
    \section{Máximo Común Divisor: GCD/MCD}

        % ==================================
        % =========   FORMAL     ===========
        % ==================================
        \subsection*{Definición Formal}

            Dados dos números cualquiera $a, b \in \Integers$ pero con mínimo alguno de ellos dos diferentes
            de 0. 

            Entonces decimos que el máximo común divisor de $a$ y $b$ denotado por $MCD(a,b) = GCD(a,b)$
            es el entero positivo $d$ que satisface:

            \begin{itemize}
                \item $d|a$ y $d|b$
                \item Si $c|a$ y $c|b$ entonces $c \leq d$.
            \end{itemize}


            % ======== IDEAS ========
            \begin{SmallIndentation}[1em]
                \textbf{Ideas}:

                Decimos que $d$ es un división común de $a$ y $b$ si $(d|a) \land (d|b)$.

                Ahora podemos construir el conjunto de los divisores comúnes.
                $Divisores = \{ d \in \Integers \Such (d|a) \land (d|b) \}$

                Ahora si, con todo esto listo, podemos ver que este conjunto nunca estará vació. 
                como 1 es un divisón común de todos los enteros.

                Ahora podemos ver que el conjunto no es infinito siempre que alguno de ellos no sea cero,
                hay sólo una cantidad finita de divisores comunes positivos. Dentro de ellos hay uno
                que es el mayor.

                La segunda condición se asegura de que $d$ sea el máximo elemento dentro del conjunto.

            \end{SmallIndentation}




        % ==================================
        % ======   PROPIEDADES     =========
        % ==================================
        \clearpage
        \subsection{Propiedades de MCD/GCD}

            Antes que nada, recuerda que para que tenga sentido hablar del máximo común divisor
            alguno de los dos $a,b$ debe de ser diferente de cero. Porfis.

            Recuerda también llamaré $c$ a lo que salga de $c=max(|a|,|b|)$.

            Ahora supongamos que es $a$ el que es diferente de 0, después de todo
            $MCD(a,b) = MCD(b,a)$

            \begin{itemize}

                \item Siempre se cumple que $0 < MCD(a,b) \leq max(|a|,|b|)$

                    % ======== DEMOSTRACION ========
                    \begin{SmallIndentation}[1em]
                        \textbf{Demostración}:

                        Para lo primero basta con recordar que 1 divide a todos los enteros,
                        así que 1 siempre será un divisor común, por lo tanto, cualquier otro
                        divisor que aspire a ser el MCD/GCD tendría que que ser mayor que 1,
                        o bien, si son primos relativos, ser el 1.

                        Basta con pensar que $c=max(|a|,|b|)$ es más grande o igual que 1,
                        y ahora veamos que es imposible que existe un número $n$ que sea el
                        máximo común divisor donde $c < n$. Ya que de ser así pasa que
                        $max(|a|,|b|) < n$. Digamos que puedo escribir a $n=c+k$.

                        Y eso nos díria que si $|(c+k)|a$ y $a \neq 0$ entonces $|c+k| \leq |a|$

                        Pero, c es positiva, y también k, por lo tanto la proposicíon $|c+k| \leq |a|$
                        es falsa. Espero que se vea claro porque, ya si $c$ es el mayor de sus
                        valores absolutos, si le añadimos otro natural a ese número solo se puede hacer más
                        grande, haciendo imposible la frase $|c+k| \leq |a|$.

                        Por lo tanto, es imposible que exista dicha $n$.

                        Y el máximo común divisor queda atrapado en esos límites.

                    \end{SmallIndentation}


                \item Siempre se cumple que $MCD(a,0) = GCD(a,0) = |a|$

                    % ======== DEMOSTRACION ========
                    \begin{SmallIndentation}[1em]
                        \textbf{Demostración}:
                        Basta con pensar que $|a|$ divide a ambos, y es más grande que 1, así
                        que vamos bien, y despúes pensar que si existiríera algún divisor más
                        grande que $|a|$ entonces se cumpliría que  $|(|a|+k)|a$ por lo tanto tambíen
                        se cumpliría lo que dijimos antes, (que si $|(|a|+k)|a$ y $a \neq 0$
                        entonces $|(|a|+k)| \leq |a|$) y eso claro es una contradicción
                        por lo tanto, $|a|$ es siempre el mayor divisor común.

                    \end{SmallIndentation} 


                \clearpage

                \item Siempre se cumple que $GCD(a,b) = GCD(-a,b) = GCD(a,-b) = GCD(-a,-b)$

                    % ======== DEMOSTRACION ========
                    \begin{SmallIndentation}[1em]
                        \textbf{Demostración}:
                        Si $d= GCD(a,b)$ entonces también se que si $c$ es también un divisor común
                        $c\leq d$, pero vemos que $d|-a$ y $d|-b$.

                        Ahora, vemos que d es también un divisor común, y es que es el mayor, porque
                        si $c|-a$ y $c|-b$ ya habiamos dicho que $c\leq d$.

                        Literalmente no hay otra forma. Demostrado.

                    \end{SmallIndentation} 


                \item El $GCD(a,b) = GCD(a, b \pm ka)$ donde $k \in \Naturals$

                    % ======== DEMOSTRACION ========
                    \begin{SmallIndentation}[1em]
                        \textbf{Demostración}:
                        
                        Vamos a hacer una primera aproximación diciendo que $GCD(a,b)=GCD(a,\pm b)$, ya 
                        si se demostrará que eso fuera cierto, creo que es obvio que puedes aplicar el 
                        proceso varias veces para llegar a $GCD(a,b) = GCD(a, \pm kb)$ donde
                        $k \in \Naturals$.

                        Para hacerlo lo que vamos a demostrar es que ambos conjuntos de divisores, el
                        primero el de $a,b$ (llamemoslo $Divisores1$) y el de $a, b \pm a$ ($Divisores2$)
                        es el mismo conjunto.

                        Si $x \in Divisores1$ entonces sabemos que $x|a$ y $x|b$, entonces gracias a una
                        propiedad de divisibilidad ya demostre antes (Si $b|a$ y $b|c$ entonces $b|a\pm c$)
                        sabemos que $x|a \pm b$, es decir $x \in Divisores2$. Además, si $y \in Divisores2$
                        entonces $y|a$ y $y|a \pm b$, por lo tanto (sabiendo que Si $a|b$ y $a|b \pm c$
                        entonces $a|c$) $y|b$, por lo tanto $y \in Divisores1$. Por lo que vemos que son el
                        mismo conjunto.

                        Si son el mismo conjunto de divisores naturales, tendrán el mismo máximo elemento.
                        ¡Bingo!

                        

                    \end{SmallIndentation} 

            \end{itemize}




        % ==================================
        % ====    IDENTIDAD DE BEZOUT   ====
        % ==================================
        \clearpage
        \subsection{Identidad de Bezout}

            Existen unos $m, n \in \Integers$ llamados coeficientes de Bezout tal que se
            cumple siempre que:

            \begin{equation}
                MCD(a,b) = GCD(a,b) = am+bn
            \end{equation}

            % ======== DEMOSTRACION ========
            \begin{SmallIndentation}[1em]
                \textbf{Demostración}:

                Este \Quote{teorema} parece bastante importante, así que veamoslo con más detalle,
                nos dice que podemos escribir al MCD/GCD de $a,b$ como una combinación lineal de
                ellos.

                Ahora, concentremos en las combinaciones lineales que sean positivas, hagamos el conjunto
                $Combinaciones = \{ am+bn \Such m, n \in \Integers, \MiniSpace am+bn > 0 \}$.

                Con esto tenemos todas las combinaciones lineales positivas.
                También sabemos que no esta vacío ese conjunto, pues mínimo $max(|a|,|b|)$ esta ahí dentro.

                Por el principio del buen orden, este conjunto tiene un primero elemento.
                Llamemos $d$ a ese elemento, donde vemos que $0 < d \leq max(|a|,|b|)$,
                esto se parece a nuestro mínimo común múltiplo.

                Veamos si es un divisor común primero, por el algoritmo de la división podemos decir
                que podemos escribir $a = dq + r$ y también como $d \in Combinaciones$, osea
                $d = am+bn$ podemos decir que $a = (am+bn)q$.

                Por lo tanto veamos que pasa si despejo r:

                $r = a-dq = a + d(-q) = a + (am+bn)(-q) = a(1-qm) + b(-qn)$

                Si no te has dado cuenta, esta de la forma $ax+by$, osea que $r$ también debería estar
                en $Combinaciones$, pero creí que $d$ era la combinación más pequeña, la única forma
                de que esto no sea una contradicción es que $r=0$, pues $0 \leq r \leq (am+bn)$
                (Inteligente, ¿no?).
                
                Así podemos darnos cuenta de que si tomamos al menor elemento de la forma $am+bn$ este
                siempre tiene que dividir a $a$, y de hecho $a$ no tiene nada de especial.
                Lo mismo pasa con $b$.

                Ok, ahora sabemos que $d$ es un divisor común, para ver que es el más pequeño
                simplemente imaginate otro, como $x$ un divisor positivo común de $a$ y $b$,
                existen entonces enteros $s,t$ tales que $a=xs \land b=xt$ y como vimos podemos poner a 
                $d$ como $d=am+bn$.

                Tenemos que $d = am+bn = (xs)m+(xt)n = x(sm+tn)$, si te das cuenta la proposición
                $x|d$ es cierta, pues $d = x(sm+tn)$, por lo que podemos decir que $|x| \leq |d|$, 
                pero vamos, ambos son positivos, eso de antes es lo mismo que $x \leq d$, por lo
                tanto por definición $d$ es nuestro máximo común divisor.

            \end{SmallIndentation}

        % ==================================
        % ======   PROPIEDADES     =========
        % ==================================
        \clearpage
        \subsection{Propiedades de MCD/GCD: Bezout Edition}
            \begin{itemize}

                \item Si tengo 3 números $a, b, c \in \Integers$ donde $c$ y alguno de los
                dos restantes $a,b$ no son cero, entonces $c$ se puede escribir como
                una combinación lineal de $a$ y $b$ si y solo si $c$ es el $GCD \ MCD$ de
                $a, b$ o bien si es uno de sus múltiplos.

                    % ======== DEMOSTRACION ========
                    \begin{SmallIndentation}[1em]
                        \textbf{Demostración}:
                            Vamos, literalmente acabo de demostrar que el GCD es equivalente
                            a escribirlos como combinación lineal, ahora también funciona con
                            los múltiplos, pues si $d$ es el $GCD$ y $c$ un múltiplo, entonces
                            tenemos que $d=am+bn$ y también $c=kd$.

                            Por lo tanto nuestra ansiada combinación lineal es simplemente
                            $c=a(km)+b(kn)$. Y ¡Bingo!

                    \end{SmallIndentation} 


                \item El conjunto $Combinaciones = \{ am+bn \Such m, n \in \Integers, \MiniSpace am+bn > 0 \}$.
                es precisamente el conjunto de múltiplos de $GCD(a,b)$.

                    % ======== DEMOSTRACION ========
                    \begin{SmallIndentation}[1em]
                        \textbf{Demostración}:
                            Sea $d=GCD(a,b)$, si $d|m$ entonces $m = dc$ para algún $c \in \Integers$
                            y entonces $m = dc = c(am+bn)=a(cm)+b(cn)$.

                            Así que cualquier multiplo de $d$ estará en este conjunto.

                            Además es claro que $d$ divide a cualquier combinación lineal
                            de $a$,$b$ por ser un divisor común.

                    \end{SmallIndentation}

                \item La pareja de $m, n \in \Integers$ llamados coeficientes de Bezout, ya sabes
                aquella que cumple que $GCD(a,b) = am+bn$, siempre serán coprimos.

                    % ======== DEMOSTRACION ========
                    \begin{SmallIndentation}[1em]
                        \textbf{Demostración}:

                        Sabemos que existen enteros $m,n$ tal que $d = am+bn$ por la
                        identidad de Bezout, además como $d$ es un divisor común
                        podemos escribir $a=dq_1$ $b=dq_2$ para algunos enteros $q_1,q_2$.

                        Por lo que $d=am+bn = dmq_1 + dnq_2 = d(mq_1 +nq_2)$, por
                        lo tanto tenemos que $1= mq_1 + nq_2$.

                        Esto es muy importante, porque nos dice que los enteros $m$ y $n$
                        son primos relativos (Dos enteros $a,b$ son primos relativos sí y sólo
                        si,existen enteros $x,y \in \Integers$ tales que $1=am+bn$).

                        Y bingo, ahí esta nuestra pareja de primos relativos.

                    \end{SmallIndentation} 


                \item Supón $GCD(a,b) = 1$ y que $a|bc$, entonces $a|c$.

                    % ======== DEMOSTRACION ========
                    \begin{SmallIndentation}[1em]
                        \textbf{Demostración}:

                        Esta idea suena muy específica, pero creeme que es muy útil.

                        Además demostrarlo es más sencillo de lo que te imaginas,
                        sabemos que por la identidad de Bezout $am+bn = 1$,
                        ahora multiplica todo por $c$ y tendremos que:
                        $amc + bnc = c$.

                        Además recuerda que $a|bc$, es decir $bc = aq$, por lo tanto
                        podemos decir que $amc + aqc = c$ esto es lo mismo que $a(mc+qc) = c$
                        por lo tanto podemos decir que $q=mc+qc$ y tener que $c = aq$, 
                        es decir $a|c$. 

                    \end{SmallIndentation} 

            \end{itemize}
                            


        % ==================================
        % ====    PRIMOS RELATIVOS     =====
        % ==================================
        \clearpage
        \subsection{Primos Relativos}

            Decimos que dos enteros $a$ y $b$ son primos relativos o coprimos si $mcd(a,b) = 1$.

        % ==================================
        % ====   IDEAS INTERESANTES    =====
        % ==================================
        \subsection*{Ideas Interesantes}
            \begin{itemize}

                \item Dos enteros $a,b$ son primos relativos sí y sólo si,existen
                enteros $x,y \in \Integers$ tales que $1=am+bn$.

                    % ======== DEMOSTRACION ========
                    \begin{SmallIndentation}[1em]
                        \textbf{Demostración}:
                            Esto es literalmente un corolario de la Identidad de Bezout,
                            porque si son primos relativos, entonces $GCD(a,b)=1$, y por 
                            la identidad existen $x,y$ tal que $1=am+bn$.

                    \end{SmallIndentation} 


                \item Sea $d=GCD(a,b)$. La pareja de $(\frac{a}{d}, \frac{b}{d})$ siempre son
                primos relativos.

                    % ======== DEMOSTRACION ========
                    \begin{SmallIndentation}[1em]
                        \textbf{Demostración}:

                        Sabemos que existen enteros $m,n$ tal que $d = am+bn$ por la
                        identidad de Bezout, además como $d$ es un divisor común
                        podemos escribir $a=dq_1$ $b=dq_2$ para algunos enteros $q_1,q_2$.

                        Por lo que $d=am+bn = dmq_1 + dnq_2 = d(mq_1 +nq_2)$, por
                        lo tanto tenemos que $1= mq_1 + nq_2$.

                        Esto es muy importante, porque nos dice que los enteros $q_1$ y $q_2$
                        son primos relativos (Dos enteros $a,b$ son primos relativos sí y sólo
                        si,existen enteros $x,y \in \Integers$ tales que $1=am+bn$).

                        Por lo tanto basta con ver que $q_1=\frac{a}{d}$ y que $q_1=\frac{b}{d}$.

                        Y bingo, ahí esta nuestra pareja de primos relativos.

                    \end{SmallIndentation} 

            \end{itemize}




    % =====================================================
    % =====          ALGORITMO DE EUCLIDES         ========
    % =====================================================
    \clearpage
    \section{Algorítmo de Euclides}

        % ==================================
        % =========   FORMAL     ===========
        % ==================================
        \subsection*{Definición Formal}

            Un algoritmo eficiente para calcular el máximo común divisor de dos enteros
            se puede conseguir aplicando repetidamente el algoritmo de Euclides.

            Si intentamos calcular $GCD(a,b)$ y sabemos del algoritmo de la división 
            que $a=bq+r$ entonces podemos simplificar el problema ya que:
            \begin{equation}
                GCD(a,b) = GCD(b,r) = GCD(b, b\%a)
            \end{equation}

            Podemos seguir aplicando esta identidad hasta que el $GCD(a, b)$ séa muy
            obvio.


        % ======== DEMOSTRACION ========
        \begin{SmallIndentation}[1em]
            \textbf{Demostración}:

            Esta afirmación es la importante: $GCD(a,b) = GCD(b,r)$ donde $r$ es el residuo
            del algoritmo de la división donde $a=bq+r$.

            Para probarla lo que haremos sera darnos cuenta que el conjunto de divisores
            comunes de $a$ y $b$ será el mismo que el de $b$ y $r$.
            Es decir para una $d$ cualquiera que sea un divisor común de $a$ y $b$ si y solo si
            $d$ es un divisor de $b$ y de $r$.

            Veamos que podemos probar esto gracias a que podemos verlo como una implicación de
            dos lados.

            Por un lado si $d$ es un divisor de $a$ y $b$, es decir $d|a$ y $d|b$
            sabemos que $d|a-k$, es un resultado que ya habiamos probado, pero que
            pasa si decimos que esa $k$ no es otra $bq$, ya que de $a=bq+r$ podemos 
            ver como $r=a-bq = a-k$, con lo que vemos que $d|r$, por lo tanto
            vimos que para cualquier $d$ que divida a $a,b$ también lo hará con $b,r$.

            Por otro lado supón que $d$ es un divisor común de $b$ y $r$, entonces
            $d|b$ y $d|r$, por lo tanto $d|bq$ y si ya sabemos que $d|bq$ entonces
            también lo hace con $d|bq+r$, por lo tanto $d|a$, por lo tanto
            vimos que para cualquier $d$ que divida a $b,r$ también lo hará con $a,b$.

            Y si tienen exactamente los mismos elementos en cada conjunto de divisores
            comunes entonces creo que es bastante obvio que el máximo elemento de cada
            conjunto será el mismo, es decir, tienen el mismo GCD.

        \end{SmallIndentation}


        % ==================================
        % =====   COMO APLICARLO    ========
        % ==================================
        \clearpage
        \subsection{Como Aplicarlo}

            Esto ya nos muestra una forma de calcular el máximo común divisor de dos
            números $a,b$ de una manera más sencilla pues en principio $b,r$ son
            números más pequeños.

            \begin{itemize}
                \item
                    El primer paso es aplicar el algoritmo para la división:
                    \begin{equation*}
                        a = bq_1 +r_1 \Space 0 \leq r_1 < b
                    \end{equation*}

                    \begin{SmallIndentation}[1em]
                        Si da la casualidad de que $r_1 = 0$ entonces $b|a$, por lo
                        que $GCD(a,b) = b$. Y listo, encontrado.

                        Si no tuvimos tanta suerte podemos al menos saber que
                        $GCD(a,b) = GCD(b,r_1)$, así que volvemos a aplicar el
                        algoritmo de la división.

                    \end{SmallIndentation}

                    
                \item
                    Ahora tenemos que 
                    \begin{equation*}
                        b = r_1q_2 +r_2 \Space 0 \leq r_2 < r_1
                    \end{equation*}

                    \begin{SmallIndentation}[1em]
                        Si da la casualidad de que $r_2 = 0$ entonces $r_1|b$, por lo
                        que $GCD(a,b) = GCD(b,r_1)=r_1$. Y listo, encontrado.

                        Si no tuvimos tanta suerte podemos al menos saber que
                        $GCD(b,r_1) = GCD(r_1,r_2)$, así que volvemos a aplicar el
                        algoritmo de la división.

                    \end{SmallIndentation}

                    


                \item
                    Como los números encontrados satisfacen $0 \leq r_n < \cdots < r_2 < r_1$,
                    vemos que este proceso terminar a lo mucho en $b$ pasos, es decir para algún
                    $n \leq b$ debemos tener que $r_n=0$ y entonces:

                    \begin{equation*}
                    \begin{split}
                        GCD(a,b)    &= GCD(b,r_1)           \\
                                    &= GCD(r_1,r_2)         \\
                                    &= \cdots               \\
                                    &= GCD(r_{n-2},r_{n-1}) \\
                                    &= GCD(r_{n-1}, 0)      \\
                                    &= r_{n-1} 
                    \end{split}
                    \end{equation*}

            \end{itemize}

            En otras palabras, el máximo común divisor de $a,b$ es el último residuo distinto
            de cero al aplicar repetidamente el algoritmo de la división como en proceso anterior.



        % ==================================
        % =====      EJEMPLO        ========
        % ==================================
        \clearpage
        \subsection{Ejemplo}
            Supón que tenemos que calcular el $GCD(2024, 748)$


            \begin{itemize}
                \item $GCD(2024, 748) = GCD(748, 528)$   donde $2024=748(2)+528$

                \item $GCD(748, 528)  = GCD(528, 220)$   donde $748=528(1)+220$

                \item $GCD(528, 220)  = GCD(220, 88)$    donde $528=220(2)+88$

                \item $GCD(220, 88)   = GCD(88, 44)$     donde $220=88(2)+44$

                \item $GCD(88, 44)    = GCD(44, 0) = 44$ donde $88=44(2)+0$
            \end{itemize}

            Y bingo, 44.





        % ==================================
        % ==  USARLO COEFICIENTES BEZUT  ===
        % ==================================
        \clearpage
        \subsection{Algorítmo Extendido de Euclides}

            Podemos añadir mas pasos al algoritmo de Euclides para darles
            más utilidad. Esta utilidad es casi exclusiva para encontrar los
            coeficientes de Bezout.

            Ya idea básica esta en que podemos despejar los residuos de cada paso
            del algoritmo de Euclides original e ir sustituyendo cada uno de los residuos
            ya que podremos ir describiendo cada uno como una combinación lineal de $a,b$,
            cuando llegemos al último residuo, donde será cero, bastare con buscar la
            combinación anterior para encontrar las $m,n$.

            Recuerda que la Identidad de Bezut nos dice que:
            \begin{equation}
                GCD(a,b) = am +bn \Space \text{donde} m,n \in \Integers
            \end{equation}

            Conocemos a $m,n$ como los coeficientes de Bezut.


            % ======== DEMOSTRACION ========
            \begin{SmallIndentation}[1em]
                \textbf{Demostración}:

                Supongamos que después de $n+1$ pasos del algoritmo de Euclides llegamos que
                $r_{n+1}=0$.

                Eso nos dice que $r_n=GCD(a,b)$. Después de todo, eso es todo lo que se
                trata el algoritmo de Euclides.

                Recuerda que $GCD(a,b) = GCD(b,r_1) = GCD(r_1,r_2)$ y así seguimos hasta que
                $GCD(r_{n-2},r_{n-1}) = GCD(r_{n-1},r_n) = GCD(r_n,0) = r_n$

                Recuerda que $a=bq+r$, es decir $r=a-bq$, pero recuerda que vimos que
                $b_n = r_{n-1}$, que $a_n=r_{n-2}$ por lo tanto vemos que
                $r_n=a-r_{n-1}q_n$ que es lo mismo que  $r_n = r_{n-2} - r_{n-1}q_n$.

                Ests fórmula es muy importante, así que la voy a repetir $r_n = r_{n-2} - r_{n-1}q_n$.

                Ok, ahora con la fórmula lista podemos en vez de hacerlo para $r_n$ hacerlo para
                $r_{n-1}$, donde vemos que $r_{n-1} = r_{n-3} - r_{n-2}q_{n-1}$.

                Ahora sustituyamos en la original:

                $r_n = r_{n-2} - (r_{n-3} - r_{n-2}q_{n-1})q_n$\\
                $r_n = (1+q_n q_{n-1})r_{n-2} + (-q_n)r_{n-3}$

                Si te das cuenta lo que hemos hecho es poner a $r_n$, es decir el GCD como
                una combinación lineal de los dos anteriores, y tu sabemos que si sigo
                aplicando este proceso hasta que $r_n$ que descrito como combinación 
                lineal de las r originales, es decir $a,b$.

            \end{SmallIndentation}






        % ==================================
        % =====      EJEMPLO        ========
        % ==================================
        \clearpage
        \subsection{Ejemplo}

        % ======== DEMOSTRACION ========
            \begin{SmallIndentation}[1em]
                Supón que tenemos que calcular el $GCD(2024, 748)$
                y también los coeficientes de Bezout.

                Primero con el GCD, es decir con el algoritmo tradicional
                tenemos que:

                \begin{itemize}
                    \item $GCD(2024, 748) = GCD(748, 528)$   donde $2024=748(2)+528$

                    \item $GCD(748, 528)  = GCD(528, 220)$   donde $748=528(1)+220$

                    \item $GCD(528, 220)  = GCD(220, 88)$    donde $528=220(2)+88$

                    \item $GCD(220, 88)   = GCD(88, 44)$     donde $220=88(2)+44$

                    \item $GCD(88, 44)    = GCD(44, 0) = 44$ donde $88=44(2)+0$
                \end{itemize}

                Y bingo, 44.


                Ahora vayamos haciendo las combinaciones lineales, 
                ten en cuenta que muchas veces hacen el algoritmo extendido empezando
                por el último paso, hasta llegar a $a,b$, pero yo lo haré \Quote{al réves}
                empezando por $a,b$ para llegar a $GCD(a,b)$, verás que lo entiendes mejor:


                \begin{itemize}
                    \item
                        Empecemos por mostrar a los originales $a,b$ como combinación lineal
                        de ellos:

                        $2024 = 2024(1) + 748(0)$\\
                        $748  = 2024(0) + 748(1)$

                    \item
                        Ahora podemos describir el primer paso del algoritmo como:\\
                        $r = a - bq = 528 = 2024 - 748(2)$

                        Y ahora sustituimos:\\
                        $528 = 2024(1) + 748(-2)$


                    \item
                        Y lo hacemos para el segundo paso:\\
                        $r = a - bq = 220 = 748 - 528(1)$

                        Y ahora sustituimos:\\
                        $220 = (2024(0) + 748(1)) - (2024(-1) + 748(-2)) = 2024(-1) + 748(3)$                    

                    \item
                        Y lo hacemos para el tercer paso:\\
                        $r = a - bq = 88 = 528 - 220(2)$

                        Y ahora sustituimos:\\
                        $88 = (2024(1) + 748(-2)) - 2(2024(-1) + 748(3)) = 2024(3) + 748(-8)$    

                    \item
                        Y lo hacemos para el cuarto paso:\\
                        $r = a - bq = 44 = 220 - 88(2)$

                        Y ahora sustituimos:\\
                        $44 = (2024(-1) + 748(3)) - 2(2024(3) + 748(-8)) = 2024(-7) + 748(19)$   

                        Ahora llegamos a lo que queriamos 

                \end{itemize}

                Bingo $44 = 2024(-7) + 748(19)$ 

            \end{SmallIndentation}


            

    % =====================================================
    % =====    MINIMO COMUN MULTIPLO : MCM / LCM     ======
    % =====================================================
    \clearpage
    \section{Mínimo Común Múltiplo: MCM/LCM}

        % ==================================
        % =========   FORMAL     ===========
        % ==================================
        \subsection*{Definición Formal}

            Dados dos números cualquiera $a, b \in \Integers - \{0\}$.

            Decimos que $c$ es un múltiplo común de $a,b$ si y solo si $a|c$ y $b|c$, es decir,
            si y solo si $\frac{c}{a}$ y $\frac{c}{b}$ $\in \Integers$.

            Consideramos al mínimo común múltiplo como la mínima $c \in \Naturals$ que cumple
            con ser un múltiplo común.

            % ======== IDEAS ========
            \begin{SmallIndentation}[1em]
                \textbf{Ideas}:

                Si $ab \neq 0$, el conjunto múltiplos comunes positivos es distinto del vacío
                y por lo tanto tiene un elemento mínimo. Este elemento es llamado el mínimo
                común múltiplo de $a$ y $b$. Este es denotado por $MCM(a,b) = LCM(a,b)$.

            \end{SmallIndentation}



        % ==================================
        % ======   PROPIEDADES     =========
        % ==================================
        \clearpage
        \subsection{Propiedades de MCM/LCM}


            \begin{itemize}

                \item Siempre se cumple que $GCD(a,b) \cdot LCM(a,b) = |ab|$

                    % ======== DEMOSTRACION ========
                    \begin{SmallIndentation}[1em]
                        \textbf{Demostración}:

                        Sea $d = GCD(a,b)$ y $m = LCM(a,b)$.

                        Entonces $d|a$ (digo es un divisor), entonces $d|ak$, digamos que
                        $k=b$, entonces $d|ab$, es decir, $\frac{ab}{d} \in \Integers$.

                        Llamemos $m'$ a $\frac{ab}{d}$ ya que se parece al máximo común multiplo.

                        Veamos que $a|\frac{ab}{d}$, que es lo mismo que decir
                        $\frac{\frac{ab}{d}}{a} = \frac{\frac{ab}{d}}{\frac{a}{1}}$
                        simplificando tenemos que $\frac{ab}{ad} = \frac{b}{d} \in \Integers$.
                        Esta oración debe ser verdadera pues, sabemos que $d|b$,
                        por lo tanto $\frac{b}{d} \in \Integers$. Es decir $m'$ es un multiplo
                        de $a$.

                        Podemos ver que algo parecido pasa con $b$, preguntar si $b|\frac{ab}{d}$
                        es lo mismo que preguntar si  $\frac{\frac{ab}{d}}{b} = \frac{\frac{ab}{d}}{\frac{b}{1}}$
                        simplicando tenemos que  $\frac{ab}{db} = \frac{a}{d} \in \Integers$.
                        Esta oración debe ser verdadera pues, sabemos que $d|a$,
                        por lo tanto $\frac{a}{d} \in \Integers$. Es decir $m'$ es un multiplo
                        de $b$.

                        Por lo tanto $m' \leq m$, es decir o $m'$ es el mínimo común multiplo
                        o es mayor que el. Podemos expresar lo anterior también como 
                        $\frac{ab}{d} \geq m = ab \geq md$.

                        Por otro lado tenemos que por la identidad de Bezout $d=ax+by$,
                        además sabemos que $m=as$ y $m=bt$ 

                        Por lo que tenemos que 
                        $dm = (ax+by)m = axm + bym = ax(bt) + by(as) = ab(xt+ys)$ llamemos
                        $k=(xt+ys)$, por lo que tenemos que $dm = ab(k)$, es decir $ab|dm$
                        y entonces recuerda que tenemos de las propiedades de divisibilidad
                        que $|ab| \leq |dm|$, $d,m$ son siempre positivos, así que 
                        $|ab| \leq dm$.

                        Así que tenemos que $ab \geq md$, que es lo mismo que $|ab| \geq md$
                        y tenemos que  $|ab| \leq dm$. Por lo tanto $ab = dm$.


                        Esta identidad es endemoniadamente útil, prueba por ejemplo con:
                        $GCD(12,-30) \cdot LCM(12,-30) = |(-12)(30)|$

                    \end{SmallIndentation}


                \item Si $LCM(a,b) = |ab|$ implica que $(a,b) = 1$

                    % ======== DEMOSTRACION ========
                    \begin{SmallIndentation}[1em]
                        \textbf{Demostración}:

                        Si $LCM(a,b) = |ab|$ recuerda que $GCD(a,b) \cdot LCM(a,b) = |ab|$
                        Entonces $GCD(a,b) = \frac{|ab|}{LCM(a,b)}$ que ya dijimos que 
                        $GCD(a,b) = \frac{|ab|}{|ab|} = 1$.

                    \end{SmallIndentation}

            \end{itemize}



    % =====================================================
    % =======      ECUACIONES DIOFANTICAS         =========
    % =====================================================
    \clearpage
    \section{Ecuaciones Diofanticas}

        Diofantos fue un matemático que vivió en Alexandria al rededor de 250 a.c.
        Él fue el primero en estudiar soluciones a ecuaciones del tipo $ax+by=c$ en los enteros.

        Esa es la única razón por las que las llamamos así, esto es todo amiguitos.


        % ==================================
        % =========   FORMAL     ===========
        % ==================================
        \subsection*{Definición Formal}

            Una ecuación diofantinas es una ecuación del a forma $ax+by=c$ con
            $a,b,c \in \Integers$.

            Una solución de esta ecuación es un par de enteros $x_0, y_0$
            que satisfacen la ecuación.


        % ==================================
        % ===    COMO ES UNA SOLUCION  =====
        % ==================================
        \subsection{Soluciones}

            La ecuación $ax+by = c$ tiene solución si y sólo si, $GCD(a,b)|c$, es decir
            si $\frac{c}{GCD(a,b)}$.

            % ======== DEMOSTRACION ========
            \begin{SmallIndentation}[1em]
                \textbf{Demostración}:

                En efecto, habíamos visto que un corolario de la demostración la identidad
                de Bezut, es que
                $Combinaciones = \{ am+bn \Such m, n \in \Integers, \MiniSpace am+bn > 0 \}$.
                es precisamente el conjunto de múltiplos de $GCD(a,b)$.

                Ahora sabemos que $d$ es el menor elemento de ese conjunto, y más aún,
                que gracias a ese colorario que $d$ divide a cualquier elemento del conjunto.

            \end{SmallIndentation}


        % ===================================
        % == COMO SON TODAS LAS SOLUCIONES ==
        % ===================================
        \clearpage
        \subsection{Soluciones Generales}

            Supón que $x_0, y_0$ es \textbf{una} solución a la ecuación $ax+ by = c$,
            entonces todas las demás soluciones estarán dadas por:

            \begin{itemize}
                \item $x = x_0 + \dfrac{b}{d}t$
                \item $y = y_0 - \dfrac{a}{d}t$
            \end{itemize}

            Donde $t \in \Integers$ y $d=GCD(a,b)$


            % ======== DEMOSTRACION ========
            \begin{SmallIndentation}[1em]
                \textbf{Demostración}:

                    Recuerda, ya sabemos que $ax_0 + by_0 = c$, ahora pongamos 
                    las soluciones generales de regreso
                    $ax_0 + by_0 = c = ax + by$.

                    Podemos entonces ver que con Algebra llegaremos a que o bien
                    $ax_0+by_0 = ax+by$ o a algo mucho más interesante:

                    $a(x - x_0) = b(y_0 - y)$.

                    Ahora recuerda que ya habíamos probado que si $d = GCD(a,b)$.
                    La pareja de $\frac{a}{d}, \frac{b}{d}$ siempre son primos relativos.
                    Ya que escribir esas fracciones se ve feo pongamos que $r = \frac{a}{d}$
                    y $s = \frac{b}{d}$.

                    Pasa algo muy divertido si intentamos dividir entre $d$ todo esto:
                    $a(x - x_0) = b(y_0 - y)$ se convierte en
                    $\frac{a}{d}(x - x_0) = \frac{b}{d}(y_0 - y)$, que es lo mismo que:

                    $r(x - x_0) = s(y_0 - y)$

                    Bajo esa ecuación puede ver que $s(y_0 - y) = rq$, donde $q=(x-x_0)$
                    y $b=(y_0 - y)$ por lo tanto $r|sb$, ahora veamos que $GCD(r,s)=1$
                    y que ya sabemos que $r|sb$ podemos aplicar un teorema que ya vimos
                    antes que dice que \Quote{Supón $GCD(a,b) = 1$ y que $a|bc$,
                    entonces $a|c$}. Y afirmar con ello que $r|b$, es decir:

                    $r|(y_0 - y)$

                    Esto es lo mismo que decir que $(y_0 - y) = rt$, podemos despejar a
                    $y$ y tener que $y= y_0 - rt = y_0 - \frac{a}{GCD(a,b)}t$.

                    Ahora hagamos algo parecido con $x$, recordemos que $r(x-x_0)=s(y_0-y)$

                    Bajo esa ecuación puede ver que $sq = r(x-x_0)$, donde $q=(y_0-y)$
                    y $b=(x_0-x)$ por lo tanto $s|rb$, ahora veamos que $GCD(r,s)=1$
                    y que ya sabemos que $s|rb$ podemos aplicar un teorema que ya vimos
                    antes que dice que \Quote{Supón $GCD(a,b) = 1$ y que $a|bc$,
                    entonces $a|c$}. Y afirmar con ello que $s|b$, es decir:

                    $s|(x - x_0)$

                    Esto es lo mismo que decir que $(x-x_0) = st$, podemos despejar a
                    $x$ y tener que $x= x_0 + st = x_0 + \frac{b}{GCD(a,b)}t$.

            \end{SmallIndentation}






    % =====================================================
    % ======   TEOREMA FUNDAMENTAL DE ARITMETICA  =========
    % =====================================================
    \clearpage
    \section{Función Phi de Euler: $\phi$ }

        Para un número $n \in \Naturals$ tenemos que:

        \begin{equation}
            \phi(n) = 
                Card(
                    \{ x \in \Naturals \Such
                        GCD(n, x) = 1
                        \text{ y además } x \leq n
                    \}
                )
        \end{equation}
 
        Es decir, $\phi(n)$ es la cantidad de naturales que son menores o iguales a
        $n$ y que ademas son primos relativos.




        % ==================================
        % =====      EJEMPLO        ========
        % ==================================
        \clearpage
        \subsection{Ejemplo}
            
            Supón que $n=9$ entonces tenemos que:

            \begin{itemize}
                \item $GCD(1, 9) = 1$
                \item $GCD(2, 9) = 1$
                \item $GCD(3, 9) = 3$
                \item $GCD(4, 9) = 1$
                \item $GCD(5, 9) = 1$
                \item $GCD(6, 9) = 3$
                \item $GCD(7, 9) = 1$
                \item $GCD(8, 9) = 1$
                \item $GCD(9, 9) = 9$
            \end{itemize}

            Si te das cuenta los primos relativos de 9 son $1,2,4,5,7,8$
            por lo tanto $\phi(9) = 6$








        % ==================================
        % ======   IMPORTANTES     =========
        % ==================================
        \subsection{Proposiciones Importantes}

            \begin{itemize}

                \item Si $k > 1$ entonces $\phi(k) < k$
                
                    % ======== DEMOSTRACION ========
                    \begin{SmallIndentation}[1em]
                        \textbf{Demostración}:

                        Si $\phi(1)$ entonces $GCD(1,1) = 1$, por lo tanto
                        $\phi(1)=1$.

                        En otro caso sabemos que $GCD(k, k) = k$ por lo tanto
                        es imposible que $\phi(k)=k$, así que $\phi(k) < k$

                    \end{SmallIndentation}


                \item $\phi(p) = p - 1$ si y solo si $p$ es primo

                    % ======== DEMOSTRACION ========
                    \begin{SmallIndentation}[1em]
                        \textbf{Demostración}:

                        Si $p$ es un número primo, entonces cada entero menor
                        que $p$ es primo relativo con con $p$, así que 
                        $\phi(p) = p-1$.

                        Por otro lado si $p>1$ y $\phi(p) = p-1$ entonces $p$
                        tiene que ser primo pues de lo contrario $p$ tiene
                        un divisor $d$ tal que $1<d<p$ y entonces
                        $\phi(n) \leq n-2$.

                    \end{SmallIndentation}

            \end{itemize}





% ======================================================================================
% ===========================         NUMEROS PRIMOS          ==========================
% ======================================================================================
\chapter{Números Primos}
    \clearpage

     

    % =====================================================
    % ==========     ALGORITMO DE LA DIVISIÓN     =========
    % =====================================================
    \clearpage
    \section{Definición}

        % ==================================
        % =========   FORMAL     ===========
        % ==================================
        \subsection*{Definición Formal}

            Un número $p \in \Naturals$ es llamado número primo o simplemente
            primo, \textbf{si sus únicos divisores positivos son 1 y p}.

            Un entero mayor que 1 que no es primo es llamado compuesto.

         
    % ==================================
    % ======   IMPORTANTES     =========
    % ==================================
    \section{Proposiciones Importantes}

        \begin{itemize}

            \item \textbf{Teorema de Euclides}:
                Si p es un primo y $p|ab$ entonces eso implica que p divide mínimo
                a $a$ ó a $b$, es decir $p|a \lor p|b$

                % ======== DEMOSTRACION ========
                \begin{SmallIndentation}[1em]
                    \textbf{Demostración}:

                    Esta es más fácil de lo que te imaginas, recuerda que si
                    $GCD(a,b) = 1$ y que $a|bc$, entonces $a|c$.

                    Ahora supongamos que $GCD(p,a) = 1$, es decir, que son primos
                    relativos, y si sabemos aparte que $p|ab$ entonces $p|b$.

                    Por el otro lado si $GCD(p, a) \neq 1$ entonces existe un multiplo
                    común entre ellos y ya que $p$ es primo, tenemos que $a=kp$ por lo
                    tanto $p|a$.

                \end{SmallIndentation}


            \item Hay una cantidad infinita de primos ($|\Primes| = \infty$)

                % ======== DEMOSTRACION ========
                \begin{SmallIndentation}[1em]
                    \textbf{Demostración - Euclides Edition}:

                    Este también es un resultado muy famoso e importante, así
                    que veamoslo con detalle:

                    La demostración es por contradicción.
                    Supongamos que sólo hay un número finito, Sean estos 
                    $MiniPrimos = \{ p_1, p_2, \dots, p_n\}$.

                    Consideremos ahora el número $p'=p_1 \cdots p_n + 1$,
                    pongamos esto como $p' = \prod_{n=1}^{n} p_k +1$

                    Tenemos dos opciones, o $p$ es primo o $p$ no lo es.
                    (lo se, me merezco un Nobel).

                    Pero $p'$ no puede ser primo pues es más grande que
                    todos los primos en la lista, así que si $p'$ fuera
                    un primo indicaría que nuestra lista esta incompleta.

                    Si fuera compuesto entonces es divisible por algún primo.
                    Digamos que ese primo se llama $p_x$, ahora supongamos que 
                    esta en el conjunto, eso indica que $p_x|p'$, que es lo
                    mismo que poner $p_x|\prod_{n=1}^{n}p_k+1$ y ya que $p_x$ 
                    esta dentro del conjunto de $MiniPrimos$, entonces
                    $p_x|\prod_{n=1}^{n}p_k$. Si te das cuenta usando un teorema anterior
                    (Si $a|b$ y $a|b+c$ entonces $a|c$) tenemos que $p_x|1$ lo cual es
                    imposible pues implicaría que $1 = kp_x$ y eso simplemente no se puede.

                    Por lo tanto $p_x$ no puede estar en $MiniPrimos$, así que el conjunto
                    no esta completo.

                    Si te das cuenta, sin importar que $p'$ sea o no primo, la conclusión
                    siempre es la misma, el conjunto no esta completo, hay más primos.

                    Siempre hay más primos.

                \end{SmallIndentation}

        \end{itemize}





    % ==================================
    % ======   COMO ENCONTRARLOS   =====
    % ==================================
    \clearpage
    \section{Como Saber si $n \in \Primes$}
    
        Dado un entero particular, ¿Cómo podemos saber si es primo o no?

        Si el número es compuesto, ¿Cómo podemos encontrar un divisor no trivial?


        % ===================================
        % ====   CRIBA DE ESATOSTENES   =====
        % ===================================
        \subsection{Fuerza Bruta Inteligente}

            La primera idea es verificar si todos los enteros menores son
            divisores, si los únicos divisores son el 1 y el -1 entonces
            el número será primo.

            Este método es simple pero costoso en términos de cómputo. Sin
            embargo hay una propiedad que nos podría facilitar el cálculo.

            \begin{itemize}

                \item Todo número compuesto $n$ tiene un divisor $a$ tal que
                    $a \leq \sqrt{n}$

                    % ======== DEMOSTRACION ========
                    \begin{SmallIndentation}[1em]
                        \textbf{Demostración}:

                        En efecto, como n es compuesto, $n = ab$.

                        Si $a = b$, es decir si es un cuadrado perfecto entonces
                        $a = b = a^2 = \sqrt{n}$.

                        En caso contrario podemos suponer, que $a<b$, si multiplicamos
                        por $a$ tenemos que $a^2<ab$. Por lo tanto $a^2 < n$.
                        Por lo que $a < \sqrt{n}$.

                    \end{SmallIndentation}

                \end{itemize}



    % =====================================================
    % ======   TEOREMA FUNDAMENTAL DE ARITMETICA  =========
    % =====================================================
    \clearpage
    \section{Teorema Fundamental de la Aritmética}

        Un número $n \in \Naturals$ puede ser expresado como
        un producto de primos.

        Notese que dicha factorización es única si no cuentas el 
        orden.

        % ==================================
        % ======   DEMOSTRACION     ========
        % ==================================
        \begin{SmallIndentation}[1em]
            \textbf{Demostración}:\\


            \begin{Indentation}[0.75em]
                \textbf{Parte I: Producto de Primos:}
            \end{Indentation}

            \begin{Indentation}[1.25em]
                La demostración de la primera parte es mucho más sencilla
                de lo que parece:

                Suponemos que $n \in \Naturals$, si $n=1$ entonces $n$ es el
                producto de un conjunto vacío de primos.

                Si $n \in \Primes$, osea si $n$ es primo pues, pues ... Ya acabamos.

                Si $n$ no es primo entonces $n=ab$, y ahora en vez de enfocarnos en 
                $n$ lo hacemos en $a,b$.

                Por inducción tenemos que llegar a que $a,b$ es un primo o bien
                es el producto de dos naturales, y ahora analizamos a esos dos números...
                Si te das cuenta, es inducción y solo acaba cuando tanto $a$ como $b$ sean
                producto de primos.

                Veríamos que por el principio de buen orden tenemos la secuencia que se
                nos va formando $1 < a < b < n$ y si siguieramos y cambiaramos nombre
                por consistencia $1 <\cdots<n_2<n_1<n$ tiene que terminar, no puede ser
                una lista y por ende un proceso infinito.

                Por lo tanto $n$ es siempre producto de Primos.
            \end{Indentation}

            \begin{Indentation}[0.75em]
                \textbf{Parte II: Es Único:}
            \end{Indentation}

            \begin{Indentation}[1.25em]
                Supongamos dos secuencias de primos que al multiplicarlos nos dan
                a $n$, incluso supongamos que existe la posibilidad de que sea
                diferente la cantidad de primos, esto estaría escrito como:

                $n=p_1p_2p_3\dots p_r = q_1q_2q_3\dots q_s$ donde $r \leq s$

                Ahora sabemos que $n$ no es más que esas dos secuencias, por 
                lo tanto $p_1|n$, es decir $p_1|q_1q_2q_3\dots q_s$

                El Teorema de Euclides (este: Si p es un primo y $p|ab$ entonces
                eso implica que p divide mínimo a $a$ ó a $b$, es decir
                $p|a \lor p|b$) implica que o bien $p_1=q_1$ ó bien
                $p_1|q_2q_3\dots q_s$.

                Así que por inducción veremos que $p_1=q_i$ para alguna
                $1 \leq i \leq s$.

                Entonces vemos que podemos cancelar a esa $p_1$ y a $q_i$ y vemos
                que:

                $p_2p_3\dots p_r = q_1q_2q_3\dots q_{i-1}q_{i+1} \dots q_s$

                Repetimos este proceso de encontrar un compañero para alguna $p_x$
                r veces.

                Si $r < s$ entonces $q_{r+1} \dots q_s = 1$ no es posible,
                por lo tanto $r=s$ y el conjunto de $p_x$ y $q_y$ son exactamente
                el mismo, hemos terminado la prueba. ¡Yeah!



            \end{Indentation}
            
        \end{SmallIndentation}



        % ==================================
        % =====   FACTORIZACION PRIMA   ====
        % ==================================
        \clearpage
        \subsection{Factorización Prima}

            Como vimos podemos escribir cualquier $n \in \Naturals$ como:

            \begin{equation}
                n = p_1 p_2 p_3 \dots p_s
            \end{equation}

            Estos $p_i$ no tienen porque ser diferentes todos, así que 
            otra forma equivalente de escribirlos es como:

            \begin{equation}
                n = p_1^{\alpha_1}p_2^{\alpha_2} \dots p_s^{\alpha_s}
                \Space
                \text{ donde tenemos que }
                \alpha_i \geq 0 \text{, y también } p_1 < p_2 < \dots < p_r
            \end{equation}








% ======================================================================================
% ===================              ALGORITMOS COOL                  ====================
% ======================================================================================
\chapter{Algoritmos Útiles}
    \clearpage

    % =====================================================
    % ========        EXPONENCIACION BINARIA      =========
    % =====================================================
    \section{Exponenciación Binaria}

        Este es un algoritmo que te permite multiplicar más rápido, literalmente, ese es 
        su objetivo, se basa en que si tiene un número como $b^e$ y quieres calcularlo
        en vez de multiplicar $b$ $e$ veces, puedes ocupar la exponenciación binaria, 
        que se basa en la observación de que:

        \begin{equation}
        \begin{split}
            \text{Si n es inpar: }  &   b^e = b (b^2)^{\frac{e-1}{2}}   \\
            \text{Si n es par: }    &   b^e = b (b^2)^{\frac{e}{2}}
        \end{split}
        \end{equation}


        Ahora usando esto podemos crear 2 métodos alternos:

        % =======================================
        % ========     METODO 1         =========
        % =======================================
        \clearpage
        \subsection*{Método 1}

            Este método es bastante sencillo:

            \begin{SmallIndentation}[1em]
                
                \begin{itemize}
                    \item   Inicializa tu respuesta a ser 0
                    \item   Convierte el exponente en base 2

                    \item Para cada digito del exponente en base 2 (Iniciando con
                          el menos significativo):
                        
                        \begin{itemize}
                            \item Si es 1: Nueva Respuesta = Respuesta$^2$ * Base 
                            \item Si es 0: Nueva Respuesta = Respuesta$^2$ 
                        \end{itemize}
                \end{itemize}

            \end{SmallIndentation}


            


        % =======================================
        % ========        EJEMPLOS      =========
        % =======================================
        \subsection*{Ejemplo Método 1}



            Este ejemplo nos muestra como usar esta propiedad para elevar más rápido, por ejemplo para encontrar
            $x^{13}$ solo tenemos que seguir el algoritmo suponiendo que recuerdas que $13_{10} = 1101_{2}$:


            % ======== SOLUCIÓN ========
            \begin{SmallIndentation}[1em]
                \textbf{Solución}:

                \begin{alignat*}{3}
                    &\text{Inicializamos : }                    
                        && respuesta = 1                        &&  \text{Ahora mismo: } respuesta = x^0  \\
                    &\text{Como $1^{er}$ dígito es 1:  } 
                        && respuesta = respuesta^2 * x          &&  \text{Ahora mismo: } respuesta = x^1  \\
                    &\text{Como $2^{do}$ dígito es 1:  }
                        && respuesta = respuesta^2 * x          &&  \text{Ahora mismo: } respuesta = x^3  \\
                    &\text{Como $3^{er}$ dígito es 0:  }
                        && respuesta = respuesta^2              &&  \text{Ahora mismo: } respuesta = x^6  \\
                    &\text{Como $4^{er}$ dígito es 1:  } \Space 
                        && respuesta = respuesta^2 * x   \Space &&  \text{Ahora mismo: } respuesta = x^{13} \\
                \end{alignat*}

            \end{SmallIndentation}

            


        % =======================================
        % ========     METODO 2         =========
        % =======================================
        \clearpage
        \subsection*{Método 2}

            Este método es bastante sencillo:

            \begin{SmallIndentation}[1em]
                
                \begin{itemize}
                    \item   Inicializa tu respuesta a ser 0
                    \item   Inicializa tu auxiliar a ser b
                    \item   Convierte el exponente en base 2

                    \item Para cada digito del exponente en base 2 (Iniciando con
                          el menos significativo):
                        
                        \begin{itemize}
                            \item Si es 1:
                                    Nueva Respuesta = Respuesta * Auxiliar y Nuevo Auxiliar = Auxiliar$^2$
                            \item Si es 0:
                                    Nuevo Auxiliar = Auxiliar$^2$
                        \end{itemize}
                \end{itemize}

            \end{SmallIndentation}



        % =======================================
        % ========        EJEMPLOS      =========
        % =======================================
        \subsection*{Ejemplo Método 2}

            Este ejemplo nos muestra como usar esta propiedad para elevar más rápido, por ejemplo para encontrar
            $x^{13}$ solo tenemos que seguir el algoritmo suponiendo que recuerdas que $13_{10} = 1101_{2}$:


            % ======== SOLUCIÓN ========
            \begin{SmallIndentation}[1em]
                \textbf{Solución}:

                \begin{alignat*}{3}
                    &\text{Inicializamos : }
                        && res = 1 \text{ y } aux = x         &&  \text{Ahora: } res = x^0 \text{ y } aux = x    \\
                    &\text{$1^{er}$ dígito es 1: }
                        && res=res*x \text{ y } aux=aux^2     &&  \text{Ahora: } res = x^1 \text{ y } aux = x^2  \\
                    &\text{$2^{do}$ dígito es 0: }
                        && aux=aux^2                          &&  \text{Ahora: } res = x^1 \text{ y } aux = x^4  \\
                    &\text{$3^{er}$ dígito es 1: }
                        && res=res*x \text{ y } aux=aux^2     &&  \text{Ahora: } res = x^5 \text{ y } aux = x^8  \\
                    &\text{$4^{er}$ dígito es 1: }     \Space
                        && res=res*x\text{ y }aux=aux^2\Space &&  \text{Ahora: } res = x^{13} \text{ y } aux = x^{16}  
                \end{alignat*}

            \end{SmallIndentation}







% ======================================================================================
% ===================        CONGRUENCIA & ENTEROS MODULO N         ====================
% ======================================================================================
\chapter{Teoría de Congruencias}
    \clearpage

    % =====================================================
    % ========         CONGRUENCIA MODULO N       =========
    % =====================================================
    \section{Congruencia Módulo N}

        % ==================================
        % =========   FORMAL     ===========
        % ==================================
        \subsection*{Definición Formal}

            Si tenemos dos elementos $a, b \in \Integers$ y $n \in \Naturals$
            entonces decirmos que $a$ es \textbf{congruente} a $b$ \textbf{modulo n}
            que escribimos como:
            \begin{equation}
                a \equiv b \pmod{n}
            \end{equation}

            Si y solo si:
            \begin{equation}
                n | (a - b)
            \end{equation}


            La idea principal de este nombre se da porque 2 enteros arbitrarios
            $a,b \in \Integers$ son congruentes módulo $n$, esto es
            $a \equiv b \pmod{n}$, si y sólo si $a$ y $b$ dejan el mismo
            residuo al ser divididos por $n$. Esto lo demostraremos en las
            siguientes páginas.



        % ==================================
        % ==   RELACION DE EQUIVALENCIA   ==
        % ==================================
        \subsection{Relación de Equivalencia}

            La notación $\equiv$ es usada porque las caracteristicas de la
            congruencia son muy muy parecidos a los de la igualdad ($=$), 
            más exigentemente es porque es una relación de equivalencia.


            \begin{itemize}
                \item
                    $ a \equiv a \pmod{n}$

                    % ======== DEMOSTRACION ========
                    \begin{SmallIndentation}[1em]
                        \textbf{Demostración}:

                        Sabemos que $n|0$ por lo tanto $n|a-a$, por lo tanto
                        $ a \equiv a \pmod{n}$

                    \end{SmallIndentation}

                \item
                    Si $a \equiv b \pmod{n}$ entonces $b \equiv a \pmod{n}$

                    % ======== DEMOSTRACION ========
                    \begin{SmallIndentation}[1em]
                        \textbf{Demostración}:

                        Si $a \equiv b \pmod{n}$ entonces $n | a - b$, por lo tanto
                        $n | -(a - b)$, es decir $n|b-a$, por lo tanto $b \equiv a \pmod{n}$

                    \end{SmallIndentation}


                \item
                    Si $a \equiv b \pmod{n}$ y $b \equiv c \pmod{n}$ entonces
                    $a \equiv c \pmod{n}$

                    % ======== DEMOSTRACION ========
                    \begin{SmallIndentation}[1em]
                        \textbf{Demostración}:

                        Si $a \equiv b \pmod{n}$ entonces $n | a - b$ y si
                        $b \equiv c \pmod{n}$ entonces $n | b - c$ (y recuerda
                        que si $b|a$ y $b|c$ entonces $b|a \pm c$) por lo tanto
                        $n | (a-b)+(b-c)$, es decir $n|a-b+b-c$, es decir $n|a-c$
                        por lo tanto $a \equiv c \pmod{n}$

                    \end{SmallIndentation}

            \end{itemize}



        % ==================================
        % ======    PROPIEDADES      =======
        % ==================================
        \clearpage
        \subsection{Propiedades}

            \begin{itemize}

                \item $a \equiv b \pmod{n}$ si y solo si $b \equiv a \pmod{n}$

                    % ======== DEMOSTRACION ========
                    \begin{SmallIndentation}[1em]
                        \textbf{Demostración}:

                        Si $a \equiv b \pmod{n}$ entonces $n | a - b$, por lo tanto
                        $n | -(a - b)$, es decir $n|b-a$, por lo tanto $b \equiv a \pmod{n}$

                        Y Si $b \equiv a \pmod{n}$ entonces $n | b - a$, por lo tanto
                        $n | -(b - a)$, es decir $n|a-b$, por lo tanto $a \equiv b \pmod{n}$

                    \end{SmallIndentation}

                \item Para cualesquiera dos enteros $a,b$ son congruentes módulo 1

                    % ======== DEMOSTRACION ========
                    \begin{SmallIndentation}[1em]
                        \textbf{Demostración}:

                        Esto es muy obvio pues $1|a-b$ (que es una proposición siempre
                        verdadera) entonces $a \equiv b \pmod{1}$

                    \end{SmallIndentation}

                \item $a \equiv r \pmod{n}$ si y solo si podemos escribir a $a$ como $a=nq+r$
                    para alguna $q$.

                    Es decir, todo entero es congruente a su residuo $r$ al ser dividido por $n$
                    (módulo $n$).

                    Otra forma común de encontrarlo es que
                    $a=nq+b \lequal a \equiv b \pmod{n}$ 


                    % ======== DEMOSTRACION ========
                    \begin{SmallIndentation}[1em]
                        \textbf{Demostración}:

                        $a \equiv r \pmod{n}$ si y solo si $n|a-r$ que es lo mismo
                        que decir $a-r = kn = nq$ que es lo mismo que decir $a=nq+r$.

                    \end{SmallIndentation}

                \item $a \equiv b \pmod{n}$, si y sólo si $a$ y $b$ dejan el mismo
                    residuo al ser divididos por $n$.

                    % ======== DEMOSTRACION ========
                    \begin{SmallIndentation}[1em]
                        \textbf{Demostración}:

                        $a \equiv b \pmod{n}$ si y solo si $a=nk+b$ para alguna $k$ (es
                        la proposición de arriba), despejemos $b=a-kn$.

                        Ahora apliquemos el algoritmo de la división $a=nq+r$, con $0<r<n$
                        sustituimos y tenemos que $b=nq+r -kn$ que es lo mismo que decir
                        $b=n(q-k)+r$ con lo que podemos ver que dejan el mismo residuo al 
                        aplicar el algoritmo de la división con $n$.

                    \end{SmallIndentation}


                \item Si $a \equiv b \pmod{n}$ y $c \equiv d \pmod{n}$ entonces
                        $a+c \equiv b+d \pmod{n}$

                    % ======== DEMOSTRACION ========
                    \begin{SmallIndentation}[1em]
                        \textbf{Demostración}:

                        Podemos escribir que $a=nq_1+b$ y $c=nq_2+d$ si las sumamos tenemos que:
                        $a+c=nq_1 + nq_2 + b+d$ esto es lo mismo que $(a+c)=n(q_1+q_2)+(b+d)$ tenemos que
                        $(a+c) - (b+d) = n(q_1+q_2)$, por lo tanto $n|(a+c) - (b+d)$, es decir
                        $a+c \equiv b+d \pmod{n}$.

                    \end{SmallIndentation}

                \item Si $a \equiv b \pmod{n}$ y $c \equiv d \pmod{n}$ entonces
                        $ac \equiv bd \pmod{n}$

                    % ======== DEMOSTRACION ========
                    \begin{SmallIndentation}[1em]
                        \textbf{Demostración}:

                        Podemos escribir que $a=nq_1+b$ y $c=nq_2+d$ si las multiplicamos tenemos que:
                        $ac=(nq_1+b)(nq_2+d)$ esto es lo mismo que
                        $(ac)=n^2q_1q_2 + dnq_1 + bnq_2 + bd$, por lo que tenemos que
                        $(ac)=n(nq_1q_2+dq_1+bq_2) +bd$, por lo tanto $(ac) - (bd)=n(nq_1q_2+dq_1+bq_2)$
                        es decir $n|(ac) - (bd)$, es decir $ac \equiv bd \pmod{n}$.

                    \end{SmallIndentation}

            \end{itemize}




        % ==================================
        % =====      A % B = MODULO   ======
        % ==================================
        \clearpage
        \subsection{Modulo: $A \% B$}

            Recuerda las 2 proposición demostrada allá arriba:

            \begin{itemize}
                \item $a,b \in \Integers$ son congruentes módulo $n$,
                    esto es $a \equiv b \pmod{n}$, si y sólo si $a$ y $b$
                    dejan el mismo residuo al ser divididos por $n$.

                \item $a \equiv r \pmod{n}$ si y solo si podemos escribir
                    a $a$ como $a=nq+r$ para alguna $q$.
             \end{itemize} 

            Por esta razón el residuo de un número $a$ cuando es divido por $n$
            (que es el mismo número el residuo que deja $b$ al dividirlo entre $n$)
            lo solemos llamar $a \% b$.


            Por lo tanto $a \% b = r$ donde $a =bq+r$ y $0 \leq r < b$.




    % =====================================================
    % ========            APLICACIONES            =========
    % =====================================================
    \clearpage
    \section{Aplicaciones}


        % ==================================
        % ==   EXPONENCIACION MODULAR   ====
        % ==================================
        \subsection{Exponenciación Modular: $b^e \equiv s \pmod {n}$}

            Es muy común usar congruencias para encontrar el residuo $s$ de 
            un númerotote (generalmente denotado de la forma $b^e$) al
            divirlo entre alguna $n$, donde $0 \leq c < n$.

            Usamos $b$ de base, $e$ de expontente y $s$ de solución.

            La idea se basa en la propiedad que ya demostramos de congruencias:
            \Quote{$a \equiv b \pmod{n}$ si y solo si ambos $a,b$ dejan el mismo
            residio al ser dividido por $n$}.

            Este proceso, de encontrar $s$ dado un número $b^e$ y un módulo $n$
            es bastante fácil y rápido, incluso cuando el número es endemoniadamente
            grande, pero el proceso inverso, encontrar $e$ dado una base $b$ y un
            módulo $n$. Esto lo hace perfecto para la criptografía.




            % =======================================
            % ======    METODO EXP BINARIA      =====
            % =======================================
            \subsection*{Método 1: Exponenciación Binaria}

                Este método es bastante sencillo y se basa en la exponenciación binaria, digo
                literalmente es el mismo algoritmo, lo único diferente es que hablamos de congruencias

                \begin{SmallIndentation}[1em]
                    
                    \begin{itemize}
                        \item   Inicializa tu respuesta a ser 0
                        \item   Inicializa tu auxiliar a ser b
                        \item   Convierte el exponente en base 2

                        \item Para cada digito del exponente en base 2 (Iniciando con
                              el menos significativo):
                            
                            \begin{itemize}
                                \item Si es 1:
                                        Nueva Respuesta = Respuesta * Auxiliar (mod n) y Nuevo Auxiliar = Auxiliar$^2$ (mod n)
                                \item Si es 0:
                                        Nuevo Auxiliar = Auxiliar$^2$ (mod n)
                            \end{itemize}
                    \end{itemize}

                \end{SmallIndentation}


            % =======================================
            % =====     METODO 2: ALTERNO       =====
            % =======================================
            \clearpage
            \subsection*{Método 2: Alterno}

                Este método es bastante sencillo:

                \begin{SmallIndentation}[1em]
                    
                    \begin{itemize}
                        \item Aplica el algoritmo de la división y llega a $b=nq+r$, de aquí tienes
                            la proposición $b \equiv r \pmod{n}$

                        \item Empieza a elevar proposición anterior hasta llegar a:

                            $b^k \equiv r^k \pmod{n}$ donde $r^k \equiv 1 \pmod{n}$


                        \item Ya que sabemos que $1^n=1$ entonces ya nos podremos acerca mucho más,
                            ¿Pero cuanto?

                            Aplica el algoritmo de la división y llega a $e = kq' + r'$, por lo
                            tanto puedes escribir la proposición $b^{kq'} \equiv 1 \pmod{n}$

                        \item Finalmente puedes también decir que $b^{r'} \equiv r^{r'} \pmod{n}$,
                            por lo tanto si multiplicas las ultimas dos congruencias tenemos que
                            $b^{kq'}b^{r'} \equiv (1)(r^{r'})$ esto es lo mismo que
                            $b^{kq'+r'} \equiv r^{r'} \pmod{n}$

                            Por lo tanto $b^e \equiv r^{r'} \pmod{n}$. Donde $r^{r'} \pmod {n}$ es 
                            nuestra respuesta.
                    \end{itemize}

                \end{SmallIndentation}


            % =======================================
            % ========        EJEMPLOS      =========
            % =======================================
            \subsubsection*{Ejemplo Método 2}

                    \begin{itemize}
                        \item \large{Ejemplo 1:}
                            Encontrar el residuo de dividir $17^{341}$ entre 5

                            % ======== SOLUCIÓN ========
                            \begin{SmallIndentation}[1em]
                                \textbf{Solución}:
                                \begin{alignat*}{3}
                                    &\text{Sabemos que: }  
                                    && 17 \equiv 2                  && \pmod{5} \\
                                    &\text{Por lo que: }
                                    && 17^2 \equiv 2^2 = 4          && \pmod{5} \\
                                    &\text{y al cuadrado da: }
                                    && 17^4 \equiv 4^2 = 16         && \pmod{5} \\
                                    &\text{y recuerda que: }
                                    && 16 \equiv 1                  && \pmod{5} \\
                                    &\text{por lo tanto: }
                                    && 17^4 \equiv 1                && \pmod{5} \\
                                    &\text{y ya que $1^k=1$: }
                                    && (17^4)^{85} = 7^{4*85}\equiv 1^{85} =1   && \pmod{5} \\
                                    &\text{que es lo mismo que: }
                                    && 17^{340} \equiv 1            && \pmod{5} \\
                                    &\text{y multiplicando por la $1^a$ congruencia: }\Space
                                    && (17^{340})(17) \equiv 1(2)   && \pmod{5} \\
                                    &\text{que es lo mismo que: }
                                    && 17^{341} \equiv 2            && \pmod{5} 
                                \end{alignat*}

                                Por lo tanto $17^{341}$ y $2$ dejan el mismo residuo al divirlos entre 5
                            \end{SmallIndentation}

                    \end{itemize}


        % ====================================
        % == CONGRUENCIAS Y DIVISIBILIDAD   ==
        % ====================================
        \clearpage
        \subsection{Encontrar si $n$ es divisible entre $m$ con Congruencias}

            \begin{itemize}

                \item Un número $n \in \Integers$ es divisble entre 3 si y solo si
                    la suma de digitos (en base 10) de $n$ es divisible entre 3.

                    % ======== DEMOSTRACION ========
                    \begin{SmallIndentation}[1em]
                        \textbf{Demostración}:

                        Antes que nada, recuerda que a $n$ lo puedes escribir como
                        $n = a(10^0) + b(10^1) + c(10^2) + \dots$ donde
                        $a, b, c, \dots \in \Naturals$ son los dígitos de $n$.

                        Ahora, también recuerda que $10 \equiv 1 \pmod{3}$.

                        Ahora $3|n$ si y solo si $n \equiv 0 \pmod{3}$ y esto pasa si
                        y solo si $a(10^0) + b(10^1) + c(10^2) + \dots \equiv 0 \pmod{3}$
                        (esto es una propiedad que vimos antes) y como recuerdas
                        ($10 \equiv 1 \pmod{3}$) tenemos que esto ocurre si y solo si:
                        $a + b + c + \dots \equiv 0 \pmod{3}$, esto es lo mismo que
                        $3|a+b+c+\dots$.

                        Es decir, un número $n \in \Integers$ es divisble entre 3 si y solo si
                        la suma de digitos de $n$ es divisible entre 3.

                    \end{SmallIndentation}

            \end{itemize}









% ======================================================================================
% ===================          ANILLOS, CAMPOS Y GRUPOS             ====================
% ======================================================================================
\chapter{Grupos, Anillos y Campos}
    \clearpage


    % =====================================================
    % ==========              GRUPOS              =========
    % =====================================================
    \clearpage
    \section{Grupo}

        % ==================================
        % =========   FORMAL     ===========
        % ==================================
        \subsection*{Definición Formal}

            Un grupo es una combinación de tres elementos que van de la mano:
            \begin{itemize}
                \item
                    \textbf{Conjunto Base:}

                    \begin{SmallIndentation}[1em]
                        Un conjunto llamado $G$ que no este vacío (daaa!)
                    \end{SmallIndentation}


                \item
                    \textbf{Relación Maestra:}

                    \begin{SmallIndentation}[1em]
                        
                        Una relación entre $R: (G \times  G) \to G$, es decir,
                        es una relación que recibe dos elementos de $G$ (o más
                        específicos un par ordenado) y te regresa un nuevo
                        elemento de $G$.

                        Vamos a suponer que esta relación tendrá la misma notación
                        que la multiplicación para que se vea mas normal:
                        $(a)(b) = c \MiniSpace$ donde $a, b, c \in G$

                        Por lo tanto este grupo será cerrado con respecto a esa
                        operación.

                        Esta relación tiene que cumplir que sea asociativa, es
                        decir $\forall a, b, c \in G, \MiniSpace (ab)c = a(bc)$

                    \end{SmallIndentation}

                

                \item
                    \textbf{Elemento Identidad para $R$:}

                    \begin{SmallIndentation}[1em]
                        Tendrá que existir un solo elemento llamado $1 \in G$
                        que cumpla con las siguientes caracteristicas: 
                        
                        \begin{itemize}
                            \item \textbf{Es neutro con respecto a esa operación:}
                                $\forall a \in G, \MiniSpace 1a = a1 = a$

                            \item \textbf{Existen inversos:}
                                $\forall a \in G, \MiniSpace \exists b \in G, \MiniSpace ab=ba=1$
                        \end{itemize}
                    \end{SmallIndentation}

            \end{itemize}



        % ==================================
        % =====     GRUPO ABELIANO     =====
        % ==================================
        \subsection{Grupo Abeliano}

            Un Grupo Abeliano es aquel grupo donde tenemos que:

            $\forall a, b \in G, \MiniSpace ab = ba$



    % =====================================================
    % ==========              ANILLOS             =========
    % =====================================================
    \clearpage
    \section{Anillo}

        % ==================================
        % =========   FORMAL     ===========
        % ==================================
        \subsection*{Definición Formal}

            Un Anillo es un conjunto $G$ que viene equipado con dos relaciones
            $R_1$ y $R_2$, (la que me referiré con las notaciones de la suma
            y la multiplicación).

            Donde se cumple que:

            \begin{itemize}
                \item \textbf{$G$ es un grupo con respecto a $+$}

                \item \textbf{$G$ es un grupo con respecto a $*$}

                \item
                    \textbf{Propiedad Distrivutiva}
                    
                    \begin{SmallIndentation}[1em]
                        Se cumple la propiedad especial en la que:
                        $\forall a, b, c \in G, \MiniSpace a(b+c)= ab +ac$
                    \end{SmallIndentation}

            \end{itemize}









       
\end{document}
