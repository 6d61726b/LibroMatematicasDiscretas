% ****************************************************************************************
% ************************      TAREA DE ALGEBRA      ************************************
% ****************************************************************************************

% =======================================================
% =======         HEADER FOR DOCUMENT        ============
% =======================================================
    % *********   DOCUMENT ITSELF   **************
    \documentclass[12pt, fleqn]{article}                             %Type of docuemtn and size of font and left eq
    \usepackage[margin=1.2in]{geometry}                             %Margins and Geometry pacakge
    \usepackage{ifthen}                                             %Allow simple programming
    \usepackage{hyperref}                                           %Create MetaData for a PDF and LINKS!
    \usepackage{pdfpages}                                           %Create MetaData for a PDF and LINKS!
    \hypersetup{pageanchor=false}                                   %Solve 'double page 1' warnings in build
    \setlength{\parindent}{0pt}                                     %Eliminate ugly indentation
    \author{Oscar Andrés Rosas}                                     %Who I am

    % *********   LANGUAJE AND UFT-8   *********
    \usepackage[spanish]{babel}                                     %Please use spanish
    \usepackage[utf8]{inputenc}                                     %Please use spanish - UFT
    \usepackage[T1]{fontenc}                                        %Please use spanish
    \usepackage{textcmds}                                           %Allow us to use quoutes
    \usepackage{changepage}                                         %Allow us to use identate paragraphs
    \usepackage{lipsum}                                             %Allow to put dummy text

    % *********   MATH AND HIS STYLE  *********
    \usepackage{ntheorem, amsmath, amssymb, amsfonts}               %All fucking math, I want all!
    \usepackage{mathrsfs, mathtools, empheq}                        %All fucking math, I want all!
    \usepackage{centernot}                                          %Allow me to negate a symbol
    \decimalpoint                                                   %Use decimal point

    % *********   GRAPHICS AND IMAGES *********
    \usepackage{graphicx}                                           %Allow to create graphics
    \usepackage{wrapfig}                                            %Allow to create images
    \graphicspath{ {Graphics/} }                                    %Where are the images :D

    % *********   LISTS AND TABLES ***********
    \usepackage{listings}                                           %We will be using code here
    \usepackage[inline]{enumitem}                                   %We will need to enumarate
    \usepackage{tasks}                                              %Horizontal lists
    \usepackage{longtable}                                          %Lets make tables awesome
    \usepackage{booktabs}                                           %Lets make tables awesome
    \usepackage{tabularx}                                           %Lets make tables awesome
    \usepackage{multirow}                                           %Lets make tables awesome
    \usepackage{multicol}                                           %Create multicolumns

    % *********   HEADERS AND FOOTERS ********
    \usepackage{fancyhdr}                                           %Lets make awesome headers/footers
    \pagestyle{fancy}                                               %Lets make awesome headers/footers
    \setlength{\headheight}{16pt}                                   %Top line
    \setlength{\parskip}{0.5em}                                     %Top line
    \renewcommand{\footrulewidth}{0.5pt}                            %Bottom line

    \lhead{                                                         %Left Header
        \hyperlink{section.\arabic{section}}                        %Make a link to the current chapter
        {\normalsize{\textsc{\nouppercase{\leftmark}}}}             %And fot it put the name
    }

    \rhead{                                                         %Right Header
        \hyperlink{section.\arabic{section}.\arabic{subsection}}    %Make a link to the current chapter
            {\footnotesize{\textsc{\nouppercase{\rightmark}}}}      %And fot it put the name
    }

    \rfoot{\textsc{\small{\hyperref[sec:Index]{Ve al Índice}}}}     %This will always be a footer  

    \fancyfoot[L]{                                                  %Algoritm for a changing footer
        \ifthenelse{\isodd{\value{page}}}                           %IF ODD PAGE:
            {\href{https://compilandoconocimiento.com/yo/}          %DO THIS:
                {\footnotesize                                      %Send the page
                    {\textsc{Oscar Andrés Rosas}}}}                 %Send the page
            {\href{https://compilandoconocimiento.com}              %ELSE DO THIS: 
                {\footnotesize                                      %Send the author
                    {\textsc{Compilando Conocimiento}}}}            %Send the author
    }
    
    
    
% ========================================
% ===========   COMMANDS    ==============
% ========================================

    % =====  GENERAL TEXT  ==========
    \newcommand \Quote {\qq}                                        %Use: \Quote to use quotes
    \newcommand \Over {\overline}                                   %Use: \Bar to use just for short
    \newcommand \ForceNewLine {$\Space$\\}                          %Use it in theorems for example
    
    \newenvironment{Indentation}[1][0.75em]                         %Use: \begin{Inde...}[Num]...\end{Inde...}
    {\begin{adjustwidth}{#1}{}}                                     %If you dont put nothing i will use 0.75 em
    {\end{adjustwidth}}                                             %This indentate a paragraph
    \newenvironment{SmallIndentation}[1][0.75em]                    %Use: The same that we upper one, just 
    {\begin{adjustwidth}{#1}{}\begin{footnotesize}}                 %footnotesize size of letter by default
    {\end{footnotesize}\end{adjustwidth}}                           %that's it


    % =====  GENERAL MATH  ==========
    \DeclareMathOperator \Space {\quad}                             %Use: \Space for a cool mega space
    \DeclareMathOperator \MiniSpace {\;}                            %Use: \Space for a cool mini space
    \newcommand \Such {\MiniSpace|\MiniSpace}                       %Use: \Such like in sets
    \newcommand \Also {\MiniSpace \text{y} \MiniSpace}              %Use: \Also so it's look cool
    \newcommand \Remember[1]{\Space\text{\scriptsize{#1}}}          %Use: \Remember so it's look cool

    \newtheorem{Theorem}{Teorema}[section]                          %Use: \begin{Theorem}[Name]\label{Nombre}...
    \newtheorem{Corollary}{Colorario}[Theorem]                      %Use: \begin{Corollary}[Name]\label{Nombre}...
    \newtheorem{Lemma}[Theorem]{Lemma}                              %Use: \begin{Lemma}[Name]\label{Nombre}...
    \newtheorem{Definition}{Definición}[section]                    %Use: \begin{Definition}[Name]\label{Nombre}...

    \newcommand{\Set}[1]{\left\{ \MiniSpace #1 \MiniSpace \right\}} %Use: \Set {Info}
    \newcommand{\Brackets}[1]{\left[ #1 \right]}                    %Use: \Brackets {Info} 
    \newcommand{\Wrap}[1]{\left( #1 \right)}                        %Use: \Wrap {Info} 
    \newcommand{\pfrac}[2]{\Wrap{\dfrac{#1}{#2}}}                   %Use: Put fractions in parentesis

    \newenvironment{MultiLineEquation}[1]                           %Use: To create MultiLine equations
        {\begin{equation}\begin{alignedat}{#1}}                     %Use: \begin{Multi..}{Num. de Columnas}
        {\end{alignedat}\end{equation}}                             %And.. that's it!
    \newenvironment{MultiLineEquation*}[1]                          %Use: To create MultiLine equations
        {\begin{equation*}\begin{alignedat}{#1}}                    %Use: \begin{Multi..}{Num. de Columnas}
        {\end{alignedat}\end{equation*}}                            %And.. that's it!


    % =====  LOGIC  ==================
    \DeclareMathOperator \doublearrow {\leftrightarrow}             %Use: \doublearrow for a double arrow
    \newcommand \lequal {\MiniSpace \Leftrightarrow \MiniSpace}     %Use: \lequal for a double arrow
    \newcommand \linfire {\MiniSpace \Rightarrow \MiniSpace}        %Use: \lequal for a double arrow
    \newcommand \longto {\longrightarrow}                           %Use: \longto for a long arrow

    % =====  NUMBER THEORY  ==========
    \DeclareMathOperator \Naturals  {\mathbb{N}}                     %Use: \Naturals por Notation
    \DeclareMathOperator \Primes    {\mathbb{P}}                     %Use: \Naturals por Notation
    \DeclareMathOperator \Integers  {\mathbb{Z}}                     %Use: \Integers por Notation
    \DeclareMathOperator \Racionals {\mathbb{Q}}                     %Use: \Racionals por Notation
    \DeclareMathOperator \Reals     {\mathbb{R}}                     %Use: \Reals por Notation
    \DeclareMathOperator \Complexs  {\mathbb{C}}                     %Use: \Complex por Notation

    % === LINEAL ALGEBRA & VECTORS ===
    \DeclareMathOperator \LinealTransformation {\mathcal{T}}        %Use: \LinealTransformation for a cool T
    \newcommand{\Mag}[1]{\left| #1 \right|}                         %Use: \Mag {Info} 

    \newcommand{\pVector}[1]{                                       %Use: \pVector {Matrix Notation} use parentesis
        \ensuremath{\begin{pmatrix}#1\end{pmatrix}}                 %Example: \pVector{a\\b\\c} or \pVector{a&b&c} 
    }
    \newcommand{\lVector}[1]{                                       %Use: \lVector {Matrix Notation} use a abs 
        \ensuremath{\begin{vmatrix}#1\end{vmatrix}}                 %Example: \lVector{a\\b\\c} or \lVector{a&b&c} 
    }
    \newcommand{\bVector}[1]{                                       %Use: \bVector {Matrix Notation} use a brackets 
        \ensuremath{\begin{bmatrix}#1\end{bmatrix}}                 %Example: \bVector{a\\b\\c} or \bVector{a&b&c} 
    }
    \newcommand{\Vector}[1]{                                        %Use: \Vector {Matrix Notation} no parentesis
        \ensuremath{\begin{matrix}#1\end{matrix}}                   %Example: \Vector{a\\b\\c} or \Vector{a&b&c}
    }

    % MATRIX
    \makeatletter                                                   %Example: \begin{matrix}[cc|c]
    \renewcommand*\env@matrix[1][*\c@MaxMatrixCols c] {             %WTF! IS THIS
        \hskip -\arraycolsep                                        %WTF! IS THIS
        \let\@ifnextchar\new@ifnextchar                             %WTF! IS THIS
        \array{#1}                                                  %WTF! IS THIS
    }                                                               %WTF! IS THIS
    \makeatother                                                    %WTF! IS THIS

    % TRIGONOMETRIC FUNCTIONS
    \newcommand{\Cos}[1]{\cos\Wrap{#1}}                             %Simple wrappers
    \newcommand{\Sin}[1]{\sin\Wrap{#1}}                             %Simple wrappers

    % === COMPLEX ANALYSIS ===
    \newcommand \Cis[1]  {\Cos{#1} + i \Sin{#1}}                    %Use: \Cis for cos(x) + i sin(x)
    \newcommand \pCis[1] {\Wrap{\Cis{#1}}}                          %Use: \pCis for the same ut parantesis
    \newcommand \bCis[1] {\Brackets{\Cis{#1}}}                      %Use: \bCis for the same to Brackets

    % === CALCULUS ===
    \newcommand \MiniDerivate[1][x] {\dfrac{d}{d #1}}               %Use: \MiniDerivate for simple use
    \newcommand \Derivate[2]                                        %Complete Derivate -- [f(x)][x]
        {\dfrac{d \; #1}{d #2}}                                     %Use: \Partial for simple use
    
    \newcommand \MiniUpperDerivate[2]                               %Mini Derivate High Orden Derivate -- [x][1]
        {\dfrac{d^{#2}}{d#1^{#2}}}                                  %Mini Derivate High Orden Derivate
    \newcommand \UpperDerivate[3]                                   %Complete High Orden Derivate -- [f(x)][x][1]
        {\dfrac{d^{#3} \; #1}{d#2^{#3}}}                            %Use: \UpperDerivate for simple use
    
    \newcommand \MiniPartial[1][x] {\dfrac{\partial}{\partial #1}} %Use: \MiniDerivate for simple use
    \newcommand \Partial[2]                                        %Complete Derivate -- [f(x)][x]
        {\dfrac{\partial \; #1}{\partial #2}}                      %Use: \Partial for simple use
    
    \newcommand \MiniUpperPartial[2]                                %Mini Derivate High Orden Derivate -- [x][1] 
        {\dfrac{\partial^{#2}}{\partial #1^{#2}}}                   %Mini Derivate High Orden Derivate
    \newcommand \UpperPartial[3]                                    %Complete High Orden Derivate -- [f(x)][x][1]
        {\dfrac{\partial^{#3} \; #1}{\partial#2^{#3}}}              %Use: \UpperDerivate for simple use


    % =====  GENERAL COLOR  =========
    \definecolor{IndigoMD}{HTML}{3F51B5}                            %Use: Color :D
    \definecolor{DeepPurpleMD}{HTML}{673AB7}                        %Use: Color :D
    \definecolor{TealMD}{HTML}{009688}                              %Use: Color :D        
    \definecolor{BlueGrey800MD}{HTML}{37474F}                       %Use: Color :D
    \definecolor{BlueGrey100MD}{HTML}{CFD8DC}                       %Use: Color :D
    \definecolor{IndigoMD}{HTML}{3F51B5}                            %Use: Color :D
    \definecolor{Green100MD}{HTML}{DCEDC8}                          %Use: Color :D

    \newenvironment{ColorText}[1]{                                  %Use: \begin{ColorText}
        \leavevmode\color{#1}\ignorespaces}                         %That's is!


    % =====  CODE EDITOR =========
    \lstdefinestyle{CompilandoStyle} {                              %This is Code Style
        backgroundcolor=\color{BlueGrey800MD},                      %Background Color  
        basicstyle=\tiny\color{white},                              %Font color
        commentstyle=\color{BlueGrey100MD},                         %Comment color
        stringstyle=\color{TealMD},                                 %String color
        keywordstyle=\color{Green100MD},                            %keywords color
        numberstyle=\tiny\color{TealMD},                            %Size of a number
        frame=shadowbox,                                            %Adds a frame around the code
        breakatwhitespace=true,                                     %Style                       
        breaklines=true,                                            %Style                   
        keepspaces=true,                                            %Style                   
        numbers=left,                                               %Style                   
        numbersep=10pt,                                             %Style 
        xleftmargin=\parindent,                                     %Style 
        tabsize=4                                                   %Style 
    }
 
    \lstset{style=CompilandoStyle}                                  %Use this style

% =====================================================
% ============        COVER PAGE       ================
% =====================================================
\begin{document}
\begin{titlepage}

    \center
    % ============ UNIVERSITY NAME AND DATA =========
    \textsc{\Large Álgebra Superior I}\\[0.5cm] 
    \textsc{\large Grupo 4020}\\[1.5cm]

    % ============ NAME OF THE DOCUMENT  ============
    \rule{\linewidth}{0.5mm} \\[1.0cm]
        { \huge \bfseries Tarea Exámen: Relaciones de Equivalencia}\\[1.0cm] 
    \rule{\linewidth}{0.5mm} \\[2.0cm]
     
    % ============  MY INFORMATION  =================
    \begin{minipage}{0.4\textwidth}
        \begin{flushleft} \large
            \textbf{\textsc{Alumno:}}\\
            Rosas Hernandez Oscar Andres
        \end{flushleft}
    \end{minipage}
    ~
    \begin{minipage}{0.4\textwidth}
        \begin{flushright} \large
            \textbf{\textsc{Profesor: }}\\
            Rodrígo Domínguez López
        \end{flushright}
    \end{minipage}\\[3,5cm]

    % ====== SEMI TITLE ==========
    {\LARGE Tarea Exámen}\\[4cm] 
    
    
    % ====== DATE ================
    {\large Lunes 30 de Octubre}\\[2cm] 

    \vfill

\end{titlepage}

% =====================================================
% ========                INDICE              =========
% =====================================================
\tableofcontents{}
\label{sec:Index}

\clearpage




% ======================================================================================
% ==================================     TAREA     =====================================
% ======================================================================================


% =====================================================
% ================     EJERCICIO 5   ==================
% =====================================================
\section{Ejercicio 5}

    Ya se demostro que la relación: $R: \Naturals^2 \times \Naturals^2$ que esta dada por la regla
    $(a, b) \; R \; (c, d)$ si y solo si $a+d = b+c$ es de equivalencia

    Además es la forma en la que construimos a los enteros desde un punto de vista conjuntivista.

    Demuestra que: Existe un representante de la forma $(n, 0)$ o $(0, n)$ o $(n,n)$ en cada
    clase de equivalencia

    % ======== DEMOSTRACION ========
    \begin{SmallIndentation}[1em]
        \textbf{Demostración}:

        Antes que nada recuerda que la resta en los naturales del estilo $m-n$ solo esta definida
        para $m \geq n$ y se define como el $m-n = k$ donde $k \in \Naturals$ tal que $k+n=m$.

        Considera la clase de equivalencia $[(a, b)]_R$ esta contiene por definición a $(a,b)$.

        Ahora veamos por casos, considera primero que el caso en que $a=b$, si esto pasa esta clase
        de equivalencia contienen al famoso $(0,0)$ pues $a+0=b+0$. En esta clase estará cualquier
        elemento del estilo $(n, n)$ pues $a+n=b+n$ es equivalente a $a+n=a+n$ que es equivalente a $n=n$.
        A este algun día lo llamaremos el cero de los enteros.

        Ahora considera el par $(a,b)$ donde $a<b$, entonces podemos pensar en el natural $b-a$, considera
        entonces el par ordenado $(0, b-a)$, vemos que esta dentro de la misma clase de equivalencia pues
        $a+b-a=b+0$ se puede reducir a $b=b$ lo cual es claramente siempre cierto, por lo tanto
        para la clase de equivalencia arbitraria $[(a, b)]_R$ con $a<b$ tenemos que existe el representante
        $(0, n)$ con $n$ definida como el natural $b-a$. A este algún lo llamaremos el -n en los enteros.

        Ahora considera el par $(a,b)$ donde $a>b$, entonces podemos pensar en el natural $a-b$, considera
        entonces el par ordenado $(a-b, 0)$, vemos que esta dentro de la misma clase de equivalencia pues
        $a+0=b+a-b$ se puede reducir a $a=a$ lo cual es claramente siempre cierto, por lo tanto
        para la clase de equivalencia arbitraria $[(a, b)]_R$ con $a>b$ tenemos que existe el representante
        $(0, n)$ con $n$ definida como el natural $a-b$. A este algún lo llamaremos el n en los enteros.

    \end{SmallIndentation}
                




% =====================================================
% ================     EJERCICIO 7   ==================
% =====================================================
\clearpage
\section{Ejercicio 7}

    Ya se demostro que la relación: $R: \Reals \times \Reals$ que esta dada por la regla
    $a \; R \; b$ si y solo si $a-b \in \Integers$ es una relación de equivalencia, además
    $[0,1)$ es un conjunto de todos los represetantes de las clases de equivalencia

    % ======== DEMOSTRACION ========
    \begin{SmallIndentation}[1em]
        \textbf{Demostración}:
        
        Es Reflexiva pues dado cualquier real $a$ tenemos que $a-a=0$ y $0 \in \Integers$, por lo tanto
        $\forall a \in \Reals \; (a, a) \in R$.

        Es Simetrica pues si $(a,b) \in R$ entonces $a-b \in \Integers$, pero sea $n = a-b$ entonces tanto
        $n \in \Integers$ como $-n \in \Integers$, por lo tanto $b-a \in \Integers$, por lo tanto
        $(b, a) \in R$, es decir de forma general $\forall (a,b) \in R \; (b,a) \in R$. 
    

        Finalmente podemos ver que para $(a,b) \in R$ y $(b,c) \in R$ tenemos por la definición de 
        la misma relación $a-b \in \Integers$ y $b-c \in \Integers$ por lo tanto sea $n=a-b$ y $m=b-c$
        entonces veamos al entero $n+m$ este se puede ver como $a-b+b-c$ es decir $m+n=a-c$ y para cualquiera
        dos enteros, su suma sigue en los enteros, por lo tanto $a-c \in \Integers$, por lo tanto
        $(a,c) \in R$, es decir de forma general $\forall (a,b) \in R$ y $(b,c) \in R$ tenemos que $(a,c) \in R$. 

        Finalmente para ver que $[0,1)$ es un conjunto de todos los represetantes de las clases de equivalencia
        basta con ver que para un entero arbitrario $a$ existe un elemento en $b \in [0,1)$ tal que $a \; R \; b$

        Esto lo podemos demostrar por casos, sea $a \in \Reals$:
        \begin{itemize}

            \item
                Si $a \in \Integers$ entonces $a \; R \; 0$ pues $a-0 \in \Integers$, y ya que $0 \in [0,1)$
                logramos encontrar un representante dentro del conjunto.

            \item
                Si $a$ no esta en los enteros, podemos describirla como $a=b+k$ con $b \in \Integers$ y
                $k \in \Reals$ donde $b$ es el entero inmediatamente anterior, ya que la separacióne entre
                enteros es de un real, $0 < k < 1$.
                Finalmente podemos ver que $a \; R \; k$ pues $a=b+k-k \in \Integers$ por lo tanto encotramos
                una $k$ tal que $0 < k < 1$ y que esta en la misma clase de equivalencia, por lo tanto
                es un representante de la misma

        \end{itemize}

        Entonces sin importar si $a$ es un entero o no, podemos encontrar un elemento dentro de $[0,1)$
        para cada clase de equivalencia, por lo tanto ese conjunto es un conjunto de representantes.

    \end{SmallIndentation}
        





% =====================================================
% ================     EJERCICIO 8   ==================
% =====================================================
\clearpage
\section{Ejercicio 8}


    Sea $X$ y $Y$ conjuntos ajenos. Si $\Set{X_i}_{i \in I}$ es una partición de $X$
    y $\Set{Y_j}_{j \in J}$ es una partición de $Y$, entonces $\Set{X_i}_{i \in I} \cup \Set{Y_j}_{j \in J}$
    es una partición de $X \cup Y$.

    % ======== DEMOSTRACION ========
    \begin{SmallIndentation}[1em]
        \textbf{Demostración}:
        
        Ya que son con conjuntos disconjuntos $X \cap Y = \emptyset$.

        Sea $Z_k \in \Set{X_i}_{i \in I} \cup \Set{Y_j}_{j \in J}$

        Ahora para podemos decir que $\Set{X_i}_{i \in I} \cup \Set{Y_j}_{j \in J}$ es una partición de $X \cup Y$
        si y solo si:

        \begin{itemize}
            \item $\forall k \in I \cup J \; Z_k \subseteq \Set{X_i}_{i \in I} \cup \Set{Y_j}_{j \in J}$

                Este sale directo, pues sea $Z_k \in \Set{X_i}_{i \in I} \cup \Set{Y_j}_{j \in J}$
                entonces o bien $Z_k \subseteq X$ o $Z_k \subseteq Y$, vayamos por casos:

                Si $Z_k \in \Set{X_i}_{i \in I}$ entonces por definición de partición sobre X
                $Z_k \subseteq X \subseteq X \cup Y$ por lo tanto $Z_k \in X \cup Y$

                Si $Z_k \in \Set{Y_j}_{j \in J}$ entonces por definición de partición sobre Y
                $Z_k \subseteq Y \subseteq Y \cup X = X \cup Y$ por lo tanto $Z_k \in X \cup Y$

            \item $\forall k \in I \cup J \; Z_k \neq \emptyset$

                Si $Z_k \in \Set{X_i}_{i \in I}$ por definición de partición niguna de las $X_i$ puede
                ser vacia, por lo tanto $Z_k$ no será vacía.

                O bien $Z_k \in \Set{Y_j}_{j \in J}$ por definición de partición niguna de las $Y_i$ puede
                ser vacia, por lo tanto $Z_k$ no será vacía.

            \item Si $z \in Z_k$ entonces $z$ no pertenece a $Z_{k'}$ $\forall k' \in I \cup J$ donde $k' \neq k$

                Esta proposición nos dice que no existen elementos que pertenzcan a más de una partición.

                Sea $z \in Z_k$ un elemento cualquiera dentro de $X \cup Y$.

                Si $Z_k \in \Set{X_i}_{i \in I}$ entonces por definición de partición sobre X 
                si $i \neq j$ entonces $X_i \cap X_j = \emptyset$, por lo tanto z no esta en nigun otro subconjunto
                de $X$ además de $Z_k$ y ya que $X$ y $Y$ son ajenos no existe elementos de X en Y, por lo tanto
                si $z \in Z_k$, z no pertenecerá a ninguna $Z_{k'}$ con $k' \neq k$.

                Si $Z_k \in \Set{X_j}_{j \in J}$ entonces por definición de partición sobre Y
                si $i \neq j$ entonces $Y_i \cap Y_j = \emptyset$, por lo tanto z no esta en nigun otro subconjunto
                de $Y$ además de $Z_k$ y ya que $X$ y $Y$ son ajenos no existe elementos de Y en X, por lo tanto
                si $z \in Z_k$, z no pertenecerá a ninguna $Z_{k'}$ con $k' \neq k$.

            \item $\cup_{k \in I \cup J} Z_k = X \cup Y$

                Sea $Z_k \in \Set{X_i}_{i \in I} \cup \Set{Y_j}_{j \in J}$
                \begin{MultiLineEquation*}{3}
                    \cup_{k \in I \cup J} Z_k
                        &= \cup_{i \in I} \Set{X_i} \;\; \cup \;\; \cup_{j \in J} \Set{Y_i}   \\
                        &= X \cup Y                                                 
                \end{MultiLineEquation*}
        \end{itemize}
    
    \end{SmallIndentation}
        




\end{document}