% ****************************************************************************************
% ************************     	LOGICA MATEMATICA   	  ********************************
% ****************************************************************************************


% =======================================================
% =======         HEADER FOR DOCUMENT        ============
% =======================================================
    % *********   DOCUMENT ITSELF   **************
    \documentclass[12pt]{report}                                    %Type of docuemtn and size of font
    \usepackage[margin=1.2in]{geometry}                             %Margins and Geometry pacakge
    \usepackage{ifthen}                                             %Allow simple programming
    \usepackage{hyperref}                                           %Create MetaData for a PDF and LINKS!
    \setlength{\parindent}{0pt}                                     %Eliminate ugly indentation
    \author{Oscar Andrés Rosas}                                     %Who I am

    % *********   LANGUAJE AND UFT-8   *********
    \usepackage[spanish]{babel}                                     %Please use spanish
    \usepackage[utf8]{inputenc}                                     %Please use spanish - UFT
    \usepackage[T1]{fontenc}                                        %Please use spanish

    % *********   MATH AND HIS STYLE  *********
    \usepackage{amsthm, amssymb, amsfonts, mathrsfs}                %Make math beautiful
    \usepackage[fleqn]{amsmath}                                     %Please make equations left
    \decimalpoint                                                   %Use decimal point

    % *********   GRAPHICS AND IMAGES *********
    \usepackage{graphicx}                                           %Allow to create graphics
    \usepackage{wrapfig}                                            %Allow to create images
    \graphicspath{ {Graphics/} }                                    %Where are the images :D

    % *********   LISTS AND TABLES ***********
    \usepackage{listings}                                           %We will be using code here
    \usepackage[inline]{enumitem}                                   %We will need to enumarate
    \usepackage{tasks}                                              %Horizontal lists
    \usepackage{longtable}                                          %Lets make tables awesome
    \usepackage{booktabs}                                           %Lets make tables awesome
    \usepackage{tabularx}                                           %Lets make tables awesome
    \usepackage{multirow}                                           %Lets make tables awesome

    % *********   HEADERS AND FOOTERS ********
    \usepackage{fancyhdr}                                           %Lets make awesome headers/footers
    \pagestyle{fancy}                                               %Lets make awesome headers/footers
    \setlength{\headheight}{16pt}                                   %Top line
    \setlength{\parskip}{0.5em}                                     %Top line
    \renewcommand{\footrulewidth}{0.5pt}                            %Bottom line

    \lhead{                                                         %Left Header
        \hyperlink{chapter.\arabic{chapter}}                        %Make a link to the current chapter
        {\normalsize{\textsc{\nouppercase{\leftmark}}}}             %And fot it put the name
    }

    \rhead{                                                         %Right Header
        \hyperlink{section.\arabic{chapter}.\arabic{section}}       %Make a link to the current chapter
            {\footnotesize{\textsc{\nouppercase{\rightmark}}}}      %And fot it put the name
    }

    \rfoot{\textsc{\small{\hyperref[sec:Index]{Ve al Índice}}}}    %This will always be a footer  

    \fancyfoot[L]{                                                  %Algoritm for a changing footer
        \ifthenelse{\isodd{\value{page}}}                           %IF ODD PAGE:
            {\href{https://compilandoconocimiento.com/yo/}          %DO THIS:
                {\footnotesize                                      %Send the page
                    {\textsc{Oscar Andrés Rosas}}}}                 %Send the page
            {\href{https://compilandoconocimiento.com}              %ELSE DO THIS: 
                {\footnotesize                                      %Send the author
                    {\textsc{Compilando Conocimiento}}}}            %Send the author
    }
    
    
    
% ========================================
% ===========   COMMANDS    ==============
% ========================================

    % =====  GENERAL MATH  ==========
    \DeclareMathOperator \Space {\quad}                             %Use: \Space for a cool mega space
    \DeclareMathOperator \MiniSpace {\;}                            %Use: \Space for a cool mini space
    \newcommand \Such {\MiniSpace|\MiniSpace}                       %Use: \Such like in sets

    % =====  LOGIC  ==================
    \DeclareMathOperator \doublearrow {\leftrightarrow}             %Use: \doublearrow for a double arrow
    \newcommand \lequal {\MiniSpace \Leftrightarrow \MiniSpace}     %Use: \lequal for a double arrow

    % =====  NUMBER THEORY  ==========
    \DeclareMathOperator \Naturals {\mathbb{N}}                     %Use: \Naturals por Notation
    \DeclareMathOperator \Integers {\mathbb{Z}}                     %Use: \Integers por Notation
    \DeclareMathOperator \Racionals{\mathbb{Q}}                     %Use: \Racionals por Notation
    \DeclareMathOperator \Reals {\mathbb{R}}                        %Use: \Reals por Notation
    \DeclareMathOperator \Complexs {\mathbb{C}}                     %Use: \Complex por Notation

    % === LINEAL ALGEBRA & VECTORS ===
    \DeclareMathOperator \LinealTransformation {\mathcal{T}}        %Use: \LinealTransformation for a cool T

    \newcommand{\pVector}[1]{                                       %Use: \pVector {Matrix Notation} use parentesis
        \ensuremath{\begin{pmatrix}#1\end{pmatrix}}                 %Example: \pVector{a\\b\\c} or \pVector{a&b&c} 
    }
    \newcommand{\lVector}[1]{                                       %Use: \lVector {Matrix Notation} use a abs 
        \ensuremath{\begin{vmatrix}#1\end{vmatrix}}                 %Example: \lVector{a\\b\\c} or \lVector{a&b&c} 
    }
    \newcommand{\Vector}[1]{                                        %Use: \Vector {Matrix Notation} no parentesis
        \ensuremath{\begin{matrix}#1\end{matrix}}                   %Example: \Vector{a\\b\\c} or \Vector{a&b&c}
    }



% =====================================================
% ============     	  COVER PAGE	   ================
% =====================================================
\begin{document}
\begin{titlepage}

	\center
	% ============ UNIVERSITY NAME AND DATA =========
	\textbf{\textsc{\Large Proyecto Compilando Conocimiento}}\\[1.0cm] 
	\textsc{\Large Matemáticas Discretas}\\[1.0cm] 

	% ============ NAME OF THE DOCUMENT  ============
	\rule{\linewidth}{0.5mm} \\[1.0cm]
		{ \huge \bfseries Lógica Matemática}\\[1.0cm] 
	\rule{\linewidth}{0.5mm} \\[2.0cm]
	
	% ====== SEMI TITLE ==========
	{\LARGE Una Pequeña Introducción}\\[7cm] 
	
	% ============  MY INFORMATION  =================
	\begin{center} \large
	\textbf{\textsc{Autor:}}\\
	Rosas Hernandez Oscar Andres
	\end{center}

	\vfill

\end{titlepage}

% =====================================================
% ========                INDICE              =========
% =====================================================
\tableofcontents{}
\label{sec:Index}

\clearpage

% ======================================================================================
% =============================       PRINCIPIOS BASICOS      ==========================
% ======================================================================================
\chapter{Proposiciones y Conectores}
    \clearpage

    % =====================================================
    % ============           DEFINICION            ========
    % =====================================================
    \section{Proposiciones}

        La lógica es una forma sistemática de pensar que nos permite deducir nueva información desde la
        información que ya conocemos.

        Recuerda que la lógica es un proceso de deducir la información correctamente,
        no sólo deducir la información correcta.

        La lógica trabajo con algo llamado proposiciones, son como las funciones para
        cálculo, o los lenguajes de programación para informática o los libros para la literatura.

        Así que empecemos por ahí ... ¿Qué son?


        % =====================================
        % =========   ¿QUE SON?     ===========
        % =====================================
        \subsection*{Definición}
            
            \textbf{Son proposiciones las frases que pueden adquirir un valor de verdadero o falso.}
            
            O dicho de manera formal:

            \textbf{Es una oración aseverativa de la que tiene sentido decir que es verdadera o falsa}.\\

            Y cuando digo frase, me refiero a:
            \begin{itemize}
                \item Secuencia finita de signos con significado y sentido de ser calificado como verdadero o falso.
                        (es decir una simple expresión matemática).

                \item Expresión lingüística susceptible de ser calificada de verdadera o falsa.
                        (es decir una frase aseverativa).
            \end{itemize}


        % =====================================
        % ====   SENTENCIAS ABIERTAS    =======
        % =====================================
        \subsection*{Sentencias Abiertas}
            Existen cosas que son parecidas a las proposiciones, pero no lo son exactamente, son cosas como:

            $P(x)$: $x$ es un número par.

            Puesto que la validez de $P(x)$ depende que número sea $x$, $P(x)$no es no totalmente cierta ni
            totalmente falsa, por lo tanto no es una proposición.

            Una oración como esta, cuya verdad depende del valor de una o más variables,
            se llama sentencias abierta.





            % ==========================
            % =====   EJEMPLOS   =======
            % ==========================
            \clearpage
            \subsection*{Ejemplo}


                Por ejemplo son proposiciones frases como:
                \begin{itemize}
                    \item $2 + 3 = 4$
                    \item Hay solamente 325 personas en Marte
                    \item $\forall x, y \in \Naturals$ se tiene que $\MiniSpace x+y \in \Reals$
                    \item Hoy es lunes
                    \item $f(x+y) = f(x) + f(y)$
                    \item Si x = 2 entonces 2x = 4
                \end{itemize}

                Pero no son cosas como:
                \begin{itemize}
                    \item ¡Ojalá no llueva hoy!
                    \item Haz la tarea
                    \item Este enunciado es falso
                    \item Tomar una siesta
                \end{itemize}



        % =====================================
        % ======   CLASIFICACION    ===========
        % =====================================
        \clearpage
        \subsection{Teoremas, Colorario y Tautológias}
            
            \subsubsection*{Clasificación de Propiedades}

                \begin{itemize}
                    \item \textbf{Tautología}: Cuando para todos los valores posibles de un conjunto
                            de proposiciones siempre será verdadero el conjunto.

                    \item \textbf{Contradicción}: Cuando para todos los valores posibles de un conjunto
                    de proposiciones esta será siempre falso.

                    \item \textbf{Contingencia}: Una proposición “común” son básicamente todas las que
                    no son ni tautologías ni contradicciones.
                \end{itemize}

            \subsubsection*{Notación}

                Además a los matemáticas les encanta demostrar todo y cuando digo todo, es TODO, así que
                aquí te dejo las diferencias entre varias palabras que se parecen:

                \begin{itemize}
                    \item \textbf{Proposición}: Enunciado que encierra un valor de verdad.

                    \item \textbf{Axioma}: Principio tan claro y evidente que no necesita demostración.

                    \item \textbf{Corolario}: Proposición demostrado que provoca una afirmación.

                    \item \textbf{Demostración}: Razonamiento por el cuál se da prueba de la
                        exactitud de una proposición.

                    \item \textbf{Lema}: Proposición que es necesaria demostrar antes de
                    establecer un teorema.

                \end{itemize}


    % =====================================================
    % ============           CONECTORES            ========
    % =====================================================
    \clearpage
    \section{Conectores Lógicos}

        Los conectores nos permiten 'concatenar' proposiciones o crear proposiciones mas avanzadas. 
        Veamos primero como solemos mostrarlos:

        \begin{longtable}{p{35mm} || p{30mm} || p{80mm}}

            % ==== HEADERS ============
            \textbf{\large Conector}
            &
            \textbf{\large Nombres}
            &
            \textbf{\large Símbolos}
            \\[1.5ex]
            \hline\hline
            & & \\                                                                    
            \endhead                                                     

            \large y  &  \large{p $\land$ q}                                    &

            \begin{minipage}[t]{\textwidth}\begin{itemize}
                \item \textbf{Conjunción de} p \textbf{y de} q
            \end{itemize}\end{minipage}                                         \\[1.5ex]
            
            \hline & & \\ \large o  &  \large{p $\lor$ q}                       &
            
            \begin{minipage}[t]{\textwidth}\begin{itemize}
                \item \textbf{Disyunción de} p \textbf{y de} q
            \end{itemize}\end{minipage}                                         \\[1.5ex]


            \hline & & \\ \large no  &  \large{$\lnot$ q}                       &

            \begin{minipage}[t]{\textwidth}\begin{itemize}
                \item \textbf{Negación de} P
            \end{itemize}\end{minipage}                                         \\[1.5ex]

            \hline & & \\ \large implica  &  \large{p $\to$ q}                  & 

            \begin{minipage}[t]{\textwidth}\begin{itemize}
                \small{
                \item p \textbf{implica} q
                \item \textbf{Si} p\textbf{, entonces} q
                \item q \textbf{si} p
                \item \textbf{Sólo si} q \textbf{entonces} p
                \item p \textbf{sólo si} q
                \item \textbf{Cuando} p\textbf{,} q
                \item \textbf{Siempre que} q\textbf{,} p
                \item q \textbf{siempre que} p
                \item p \textbf{es una condición suficiente para} q
                \item q \textbf{es una condición necesaria para} p
                \item \textbf{Es necesario que} q \textbf{para} p
                \item \textbf{Es suficiente que} p \textbf{para que} q
                }\\
            \end{itemize}\end{minipage}                                         \\[1.5ex]

            \hline & & \\ \large si y solo si  &  \large{p $\doublearrow$ q}    &

            \begin{minipage}[t]{\textwidth}\begin{itemize}
                \item p \textbf{ssi} q
                \item p \textbf{es equivalente a} q
            \end{itemize}\end{minipage}                                         \\
 
        \end{longtable}

        % =====================================
        % ======     NEGACION     ===========
        % =====================================
        \clearpage
        \subsection{Negación}

            Es un operador que opera sobre dos valores de verdad. Devuelve el 
            inverso del valor de verdad de la proposición que le pases.

            La conjunción es una de las compuertas más basicas. \\

            \begin{tabular}{ |c|c|c| } 
                \hline &&\\
                \large{Nombre} & $p$ & $\lnot p$ \\[0.5em]
                \hline
                \multirow{2}{5em}{Negación}
                & $F$ & $V$  \\ \cline{2-3}
                & $V$ & $F$  \\ \cline{2-3}
                \hline
            \end{tabular}

        % =====================================
        % ======     CONJUNCION     ===========
        % =====================================
        \subsection{Conjunción}

            Es un operador que opera sobre dos valores de verdad. Devuelve verdadero
            \textbf{solo} cuando ambas son verdaderas, y falso en cualquier
            otra combinación.

            La conjunción es una de las compuertas más basicas. \\

            \begin{tabular}{ |c|c|c|c|c| } 
                \hline &&&\\
                \large{Nombre} & $p$ & $q$ & $p \land q$ \\[0.5em]
                \hline
                \multirow{4}{5em}{Conjunción}
                & $F$ & $F$ & $F$ \\ \cline{2-4}
                & $F$ & $V$ & $F$ \\ \cline{2-4}
                & $V$ & $F$ & $F$ \\ \cline{2-4}
                & $V$ & $V$ & $V$ \\ 
                \hline
            \end{tabular}


        % =====================================
        % ======     DISYUNCION     ===========
        % =====================================
        \subsection{Disyunción}

            Es un operador que opera sobre dos valores de verdad. Devuelve falso
            \textbf{solo} cuando ambas son falsas, y verdadero en cualquier
            otra combinación.

            La conjunción es una de las compuertas más basicas. \\

            \begin{tabular}{ |c|c|c|c|c| } 
                \hline &&&\\
                \large{Nombre} & $p$ & $q$ & $p \lor q$ \\[0.5em]
                \hline
                \multirow{4}{5em}{Disyunción}
                & $F$ & $F$ & $F$ \\ \cline{2-4}
                & $F$ & $V$ & $V$ \\ \cline{2-4}
                & $V$ & $F$ & $V$ \\ \cline{2-4}
                & $V$ & $V$ & $V$ \\ 
                \hline
            \end{tabular}





        % =====================================
        % ======     IMPLICACION    ===========
        % =====================================
        \clearpage
        \subsection{Implicación}

            Devuelve falso \textbf{solo} cuando la primera premisa es verdadera, pero la 
            segunda es falsa, y verdadero en cualquier otra combinación.

            Ve a $P \to Q$ como una promesa de que siempre que $P$ es verdadera, $Q$ será verdadera también.
            Sólo hay una manera de romper esta promesa y que es si $P$ sea verdad y $Q$ es falso.\\


            \begin{tabular}{ |c|c|c|c|c| } 
                \hline &&&\\
                \large{Nombre} & $p$ & $q$ & $p \to q$ \\[0.5em]
                \hline
                \multirow{4}{5em}{Disyunción}
                & $F$ & $F$ & $V$ \\ \cline{2-4}
                & $F$ & $V$ & $V$ \\ \cline{2-4}
                & $V$ & $F$ & $F$ \\ \cline{2-4}
                & $V$ & $V$ & $V$ \\ 
                \hline
            \end{tabular}\\[1.0em]

            \subsubsection*{Ideas Importantes}

                La implicación es creo yo la más importante de todas, y no es porque sea básica, 
                es más: $p \to q$ es totalmente equivalente a $\lnot p \lor q$.

                Usando la implicación hay algunas cosas famosas que deberías saber:

                \begin{itemize}
                    \item \textbf{Contrapositiva del Condicional}
                            Esta equivalencia es muy importante, pues es muy usada para las demostraciones
                            (no te preocupes Timmy, ya entenderas después).
                            \begin{equation*}
                                p \to q \lequal \lnot q \to \lnot p
                            \end{equation*}

                    \item \textbf{Implicaciones Famosas}
                            No se a quién se le ocurrio ponerles nombres, pero creo que te combiene
                            que las conozcas.

                            \begin{tabular}{ |c|c|c| } 
                                \hline &&\\
                                \large{Nombre} & \large{Forma} & \large{Es equivalente con...}      \\[0.5em]
                                \hline \hline
                                
                                \textbf{Condicional}    & $p \to q$             & Contrapositiva    \\ \hline
                                \textbf{Contrapositiva} & $\lnot q \to \lnot p$ & Condicional       \\ \hline\hline 
                                
                                \textbf{Recíproca}      & $q \to p$             & Inversa           \\ \hline
                                \textbf{Inversa}        & $\lnot p \to \lnot q$ & Recíproca         \\ \hline
                            \end{tabular}
                \end{itemize}










\end{document}
